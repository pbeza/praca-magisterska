\documentclass{mini}
\usepackage[utf8]{inputenc}

\usepackage[polish]{babel}
\usepackage{subfiles}
\usepackage[style=numeric,backend=biber]{biblatex}
\addbibresource{bibliography.bib}

\title{Protokół zarządzania stacjami komputerowymi pod kontrolą systemu Linux}
\author{\href{mailto:P.Beza@student.mini.pw.edu.pl}{Patryk Bęza}}
\album{237533}
\supervisor{\href{M.Kozlowski@mini.pw.edu.pl}{dr inż. Marek Kozłowski}}
\discipline{informatyka}
\monthyear{luty 2017}
\date{\today}

\begin{document}
\maketitle
\makeatletter
{
    \renewcommand{\arraystretch}{2.5}
    \centerline{
        \begin{tabular}{>{\bfseries}l>{\itshape}l}
	Autor:            & \@author\\
	Numer albumu:     & \@album\\
        Promotor:         & \@supervisor\\
        Tytuł:            & \parbox{6.5cm}{\@title}\\
        Uczelnia:         & Politechnika Warszawska\\
        Wydział:          & Matematyki i Nauk Informacyjnych (MiNI)\\
        Kierunek:         & Informatyka\\
        Rok akademicki:   & 2016/2017\\
	Semestr:          & Zimowy\\
        Dziekan Wydziału: & prof. nzw. dr hab. Irmina Herburt\\
        \end{tabular}
    }
}
\makeatother
\tableofcontents
%
% Ze względu na custom'owy styl zastosowany w glossaries zamiast '\\' używamy '\\&' (bo słownik to tabelka dwukolumnowa).
% Na końcu każdego wpisu w słowniku dodaje się automatycznie znak kropki '.' i nowej linii '\\'.
%

\newglossaryentry{gnulinux}
{
	name={GNU/Linux},
	description={System operacyjny \texttt{GNU} napisany przez \emph{Richarda Stallmana}, z jądrem \texttt{Linux}, napisanym przez \emph{Linusa Torvalds'a}~\cite{gnulinux}}
}
\newglossaryentry{kernel}
{
	name={kernel},
	description={Jądro \gls{OpenCL}.\\&Program uruchamiany w środowisku OpenCL na urządzeniu docelowym, a kompilowany na urządzeniu host}
}
\newglossaryentry{UTF8}
{
	name={UTF-8},
	description={\textit{UCS Transformation Format 8-bit}.\\&System kodowania \emph{Unicode}, wykorzystujący od 8 do 32 bitów do zakodowania pojedynczego znaku, w pełni kompatybilny z \gls{ASCII}}
}
\newacronym{CPU}{CPU}{\textit{Central Processing Unit}.\\&Procesor}
\newacronym{API}{API}{\textit{Application Programming Interface}.\\&Interfejs Programowania Aplikacji}
\newacronym{BOM}{BOM}{\textit{Byte Order Mark}.\\&Znacznik kolejności bajtów}
\newacronym{IDE}{IDE}{\textit{Integrated Development Environment}.\\&Zintegrowane Środowisko Programistyczne}

\printglossary[style=clong,title=Słownik pojęć]

\newcounter{savepage}
\setcounter{savepage}{\thepage}

\subfile{chapters/abstract}

\setcounter{page}{\thesavepage}
\stepcounter{page}

\subfile{chapters/intro}
\subfile{chapters/ch01}
\subfile{chapters/ch02}

%\bibliographystyle{plplain}
%\bibliography{bibliography}
\printbibliography

\end{document}
