%
% Ze względu na custom'owy styl zastosowany w glossaries zamiast '\\' używamy '\\&' (bo słownik to tabelka dwukolumnowa).
% Na końcu każdego wpisu w słowniku dodaje się automatycznie znak kropki '.' i nowej linii '\\'.
%

\newglossaryentry{gnulinux}
{
	name={GNU/Linux},
	description={System operacyjny \texttt{GNU} napisany przez \emph{Richarda Stallmana}, z jądrem \texttt{Linux}, napisanym przez \emph{Linusa Torvalds'a}~\cite{gnulinux}}
}
\newglossaryentry{kernel}
{
	name={kernel},
	description={Jądro \gls{OpenCL}.\\&Program uruchamiany w środowisku OpenCL na urządzeniu docelowym, a kompilowany na urządzeniu host}
}
\newglossaryentry{UTF8}
{
	name={UTF-8},
	description={\emph{UCS Transformation Format 8-bit}.\\&System kodowania \emph{Unicode}, wykorzystujący od 8 do 32 bitów do zakodowania pojedynczego znaku, w pełni kompatybilny z \gls{ASCII}}
}
\newglossaryentry{PWD}
{
	name={\texttt{\$PWD}},
	description={\emph{\textbf{P}athname of the current \textbf{W}orking \textbf{D}irectory}\\&Zmienna środowiskowa systemu \gls{gnulinux}, zawierająca ścieżkę aktualnego katalogu roboczego}
}
\newacronym{CPU}{CPU}{\emph{Central Processing Unit}.\\&Procesor}
\newacronym{API}{API}{\emph{Application Programming Interface}.\\&Interfejs Programowania Aplikacji}
\newacronym{BOM}{BOM}{\emph{Byte Order Mark}.\\&Znacznik kolejności bajtów}
\newacronym{IDE}{IDE}{\emph{Integrated Development Environment}.\\&Zintegrowane Środowisko Programistyczne}
