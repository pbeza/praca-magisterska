% Uwaga: Na końcu każdego wpisu w słowniku dodaje się automatycznie znak kropki '.' i~nowej linii '\\'.

\hyphenation{ARPANET}
\hyphenation{Ansible}
\hyphenation{Apache}
\hyphenation{Burgessa}
\hyphenation{Capistrano}
\hyphenation{GNU/Linux}
\hyphenation{Laboratory}
\hyphenation{Linux}
\hyphenation{Makefile}
\hyphenation{POSIX}
\hyphenation{Perfect} \hyphenation{Forward} \hyphenation{Secrecy}
\hyphenation{PowerShell}
\hyphenation{Pythonowe}
\hyphenation{Rijandel}
\hyphenation{copyleft}
\hyphenation{handler}
\hyphenation{replay} \hyphenation{attack}
\hyphenation{signal}
\hyphenation{socket}


\newglossaryentry{gnu}
{
	name={GNU},
	description={Akronim rekurencyjny: \emph{GNU’s not Unix}. Uniksopodobny system operacyjny, zapoczątkowany przez Richarda Stallmana w~styczniu~1984~roku, złożony wyłącznie z~\glslink{wolne-oprogramowanie}{wolnego oprogramowania}. W~ramach projektu \gls{gnu} powstały m.in.~\emph{gcc}, \emph{glibc}, \emph{bash}, \emph{emacs} i~jądro systemu operacyjnego --- \emph{GNU~Hurd} --- jednak jego popularność jest dużo mniejsza od~jądra \gls{linux}. System operacyjny \gls{gnu} z~jądrem \gls{linux} nazywamy dystrybucją \gls{gnulinux}~\cite{gnu,gnulinux,gnu-faq,wiki:linux-naming-controversy}.}
}
\newglossaryentry{linux}
{
	name={Linux},
	description={Uniksopodobne jądro systemu operacyjnego będące \glslink{wolne-oprogramowanie}{wolnym oprogramowaniem}, zapoczątkowane przez Linusa Torvaldsa w~1991~roku~\cite{linux-kernel}. Często mylnie lub~potocznie utożsamiany z~określeniem \gls{gnulinux}~\cite{gnu-faq,wiki:linux-naming-controversy}.}
}
\newglossaryentry{unix}
{
	name={Unix},
	description={Rodzina systemów operacyjnych, mających wspólny początek w~postaci systemu \gls{unix} powstałego w~1969~roku w~\emph{AT\&T} (obecnie \emph{Bell~Labs}). Korzystając z~kodu \glslink{unix}{Unix'a} powstało wiele pokrewnych systemów operacyjnych, m.in.~BSD, Solaris, Sun~OS, System~V, AIX, HP-UX, Mac~OS~X. Projekt \gls{gnu} był od~początku projektowany jako ulepszony \gls{unix}, niebazujący na~jego kodzie źródłowym~\cite{gnu-initial-msg}.}
}
\newglossaryentry{gnulinux}
{
	name={GNU/Linux},
	description={System operacyjny \gls{gnu} zapoczątkowany przez Richarda Stallmana, z~jądrem \gls{linux}, zapoczątkowanym przez Linusa Torvaldsa~\cite{gnulinux}. Często mylnie lub~potocznie utożsamiany z~określeniem \gls{linux}~\cite{gnu-faq,wiki:linux-naming-controversy}.}
}
\newglossaryentry{kernel}
{
	name={kernel},
	description={Jądro systemu \gls{gnulinux}. Centralna część systemu operacyjnego, mająca nad~nim całkowitą kontrolę. W~szczególności, jądro systemu operacyjnego jest odpowiedzialne za~zarządzanie pamięcią, komunikacją między urządzeniami, obsługę sterowników i~za~inne, niskopoziomowe zadania.}
}
\newglossaryentry{UTF8}
{
	name={UTF-8},
	description={\emph{UCS Transformation Format 8-bit}. System kodowania \emph{Unicode}, wykorzystujący od~8~do 32~bitów do~zakodowania pojedynczego znaku, w~pełni kompatybilny z~\gls{ASCII}.}
}
\newglossaryentry{PWD}
{
	name={PWD},
	description={\emph{Pathname of the current Working Directory}. Zmienna środowiskowa systemu \gls{gnulinux} przechowująca ścieżkę aktualnego katalogu roboczego.}
}
\newglossaryentry{wolne-oprogramowanie}
{
	name={wolne oprogramowanie},
	description = {\emph{Free Software}. Oprogramowanie publikowane na~licencji typu \glslink{copyleft}{copyleft}, które zezwala użytkownikom je~uruchamiać, powielać, badać, zmieniać i~ulepszać~\cite{free-software}.}
}
\newglossaryentry{gpl}
{
	name={licencja GPL},
	description={\emph{GNU General Public License}. Licencja \glslink{wolne-oprogramowanie}{wolnego oprogramowania} stworzona w~1989~roku przez Richarda Stallmana, gwarantująca użytkownikowi możliwość uruchamiania, kopiowania i~modyfikowania oprogramowania, pod~warunkiem, że~zmodyfikowana wersja zostanie również opublikowana na~warunkach tej~licencji. Najnowszą wersją \glslink{gpl}{GPL} jest \glslink{gpl}{GPL~3}, opublikowana 29~czerwca~2007~roku~\cite{gpl3,wiki:licenses-comparison}.}
}
\newglossaryentry{lgpl}
{
	name={licencja LGPL},
	description={\emph{GNU Lesser General Public License}. Licencja typu \glslink{copyleft}{copyleft}, bazująca na~licencji \glslink{gpl}{GPL} --- różni się od~niej tylko tym, że~wykorzystanie \glslink{lgpl}{LGPL} dopuszcza korzystanie z~oprogramowania objętego licencją w~programach prawnie zastrzeżonych~\cite{lgpl3,why-not-lgpl,wiki:licenses-comparison}.}
}
\newglossaryentry{mit-license}
{
	name={licencja MIT},
	description={Licencja \emph{Massachusetts Institute of Technology}. Licencja typu \glslink{copyleft}{copyleft}, kompatybilna z~wieloma innymi \glslink{wolne-oprogramowanie}{wolnymi} licencjami --- np.~z~licencją~\glslink{gpl}{GPL}~\cite{mit}. W~projekcie~\emph{Fedora} i~na~\emph{GitHub} jest to~najczęściej używana licencja~\cite{mit-popularity-fedora,mit-popularity-github,wiki:licenses-comparison}.}
}
\newglossaryentry{apache2.0-license}
{
	name={licencja Apache~2.0},
	description={Licencja typu \glslink{copyleft}{copyleft}~opublikowana w~styczniu~2004~roku, zezwalająca użycie oprogramowania objętą tą~licencją również na~użytek zamkniętego oprogramowania komercyjnego~\cite{apache2.0,wiki:licenses-comparison}. Jest kompatybilna z~licencją \glslink{gpl}{GPL~3}, ale nie z~jej poprzednimi wersjami.}
}
\newglossaryentry{bsd-license}
{
	name={licencje BSD},
	description={Rodzina licencji typu \glslink{copyleft}{copyleft}, zezwalająca na~rozprowadzanie oprogramowania bez postaci źródłowej oraz~na~włączenia do~zamkniętego oprogramowania, pod~warunkiem załączenia do~oprogramowania informacji o~autorach oryginalnego kodu i~treści licencji~\cite{bsd,wiki:licenses-comparison}.}
}
\newglossaryentry{copyleft}
{
	name={licencje copyleft},
	description={Licencje umożliwiające uczynienie programu \glslink{wolne-oprogramowanie}{wolnym oprogramowaniem} i~nakazanie, by~wszystkie jego zmienione i~poszerzone wersje również takimi były~\cite{copyleft,wiki:licenses-comparison}. Nazwa \emph{copyleft} jest odwróceniem znaczenia słowa \emph{copyright}.}
}
\newglossaryentry{ssh}
{
	name={SSH},
	description={\emph{Secure Shell}. Protokół sieciowy działający w~trybie klient--serwer, zapewniający bezpieczne zdalne logowanie i~inne bezpieczne usługi sieciowe~\cite{rfc:ssh}. Serwer \gls{ssh} działa domyślnie na~porcie~22.}
}
\newglossaryentry{tcpip}
{
	name={TCP/IP},
	description={Model warstwowej struktury protokołów komunikacyjnych bazujący na~modelu \emph{Department-of-Defence~(DoD)}~\cite{rfc:tcpip-dod}, powstały w~\gls{darpa}. Protokół TCP~miał pierwotnie odpowiadać za~zadania, które dziś realizuje protokół IP. Z~czasem okazało się, że~wygodniej jest rozdzielić te~protokoły i~protokół IP został wydzielony z~pierwotnego zakresu odpowiedzialności TCP (dlatego też~pierwsza wersja~IP to~IPv4, a~nie~IPv1)~\cite{tcpguide-tcpip-history}. Protokół IP odpowiada głównie za~logiczną adresację i~fragmentację datagramów. Pierwsza wersja protokołu IP~została zaproponowana w~\glslink{rfc}{RFC760} --- została ona rok później zastąpiona przez \glslink{rfc}{RFC791} wraz z~dalszymi poprawkami~\cite{rfc:ip-rfc760,rfc:ip-rfc791}. Pierwsza wizja protokołu TCP~została opisana w~\glslink{rfc}{RFC675}, a~kolejne wersja w~\glslink{rfc}{RFC793}, którą później również poszerzano kolejnymi dokumentami~\gls{rfc}~\cite{rfc:tcp-rfc675,rfc:ip-rfc793}.}
}
\newglossaryentry{signal-handler}
{
	name={signal handler},
	description={Funkcja, która zostaje wywołana przez system operacyjny gdy~program otrzymuje od~systemu niemaskowany, nieignorowany sygnał~\cite{signal-handler}.}
}
\newglossaryentry{fifo}
{
	name={FIFO},
	description={\emph{First In, First Out}. Łącze nazwane. W~ogólności \gls{fifo} to~sposób organizacji i~manipulacji danymi, w~szczególności danymi na~stosie. W~kontekście niniejszej pracy pisząc o~\gls{fifo}, mowa o~pliku specjalnym, podobnym w~działaniu do~łączy nienazwanych~(potoków) --- różni się od~nich tym, że~dostęp do~\gls{fifo} odbywa się przez system plików~\cite{fifo-manual}.}
}
\newglossaryentry{socket}
{
	name={socket},
	description={W~\glslink{unix-like-system}{systemach *niksowych}~plik specjalny reprezentujący dwukierunkowy punkt końcowy połączenia --- najczęściej sieciowego (wyjątek stanowią \emph{Unix sockets})~\cite{socket-definition-oracle}.}
}
\newglossaryentry{posix}
{
	name={POSIX},
	description={\emph{Portable Operating System Interface for Unix}. Rodzina standardów opublikowana przez \emph{IEEE Computer Society}, mająca na~celu zapewnienie zgodności pomiędzy systemami operacyjnymi m.in.~na~poziomie~API. W~czasie pisania niniejszej pracy, najnowsza wersja standardu \gls{posix} została opublikowana w~kwietniu~2013~roku~\cite{posix-ieee,posix-opengroup}.}
}
\newglossaryentry{unix-like-system}
{
	name={system \mbox{*niksowy}},
	description={System operacyjny zbliżony budową do~systemu \gls{unix}, np.~\gls{gnulinux}, BSD, Solaris, HP-UX, AIX~itp.~\cite{wiki:unix-like}}
}
\newglossaryentry{darpa}
{
	name={DARPA},
	description={\emph{Defense Advanced Research Projects Agency}. Amerykańska agencja rządowa działająca w~strukturach Departamentu Obrony Stanów Zjednoczonych powstała w~1958~roku, zajmująca się~rozwojem technologii mającej potencjał militarny. Organizacja zasłużyła się dla rozwoju informatyki m.in.~finansując rozwój projektu ARPANET, w~ramach którego zostało opracowanych wiele protokołów, na~czele z~\gls{tcpip}. W~ostatnich latach intensywnie wspiera rozwój sztucznej inteligencji, np.~w robotach i~autonomicznych samochodach~\cite{darpa-grandchallange,darpa-robotics-challange,darpa-robotics-challange-ieee}.}
}
\newglossaryentry{demon}
{
	name={demon},
	description={\emph{Daemon}. Program komputerowy działający w~tle. Nazwy \glslink{demon}{demonów} w~systemach \glslink{unix-like-system}{*niksowych} zwyczajowo kończą się literą~\textbf{d}.}
}
\newglossaryentry{rfc}
{
	name={RFC},
	description={\emph{Request for Comments}. Zbiór technicznych dokumentacji związanych z~protokołami sieciowymi i~sieciami komputerowymi, w~szczególności z~Internetem, publikowany przez \glslink{ietf}{Internet Engineering Task Force~(IETF)}~\cite{rfc-editor}. Każdy dokument \gls{rfc} ma~unikatowy numer identyfikacyjny używany w~odniesieniach. Błędy w~\gls{rfc} są~publikowane w~erratach, a~większe zmiany następują przez opublikowanie nowego \gls{rfc}, anulującego poprzednik. Idea dokumentów \gls{rfc} pojawiła się jako część projektu ARPANET w~\gls{darpa}.}
}
\newglossaryentry{rsa}
{
	name={RSA},
	description={\emph{Rivest-Shamir-Adleman}. Asymetryczny algorytm kryptograficzny z~kluczem publicznym, zaprojektowany w~1977~roku przez Rona Rivesta, Adiego Shamira i~Leonarda Adlemana~\cite{wiki:rsa,rsa}. Był pierwszym algorytmem, który może być stosowany zarówno do~szyfrowania jak i~podpisów cyfrowych.}
}
\newglossaryentry{dh}
{
	name={Diffie-Hellman},
	description={Algorytm Diffiego-Hellmana umożliwia dwóm stronom połączenia na~ustalenie tajnego klucza w~niezabezpieczonej sieci~\cite{mimuw-ssl-w04}. Efemeryczna wersja tego protokołu jest używana w~celu osiągnięcia własności \glslink{pfs}{Perfect Forward Secrecy}.}
}
\newglossaryentry{pfs}
{
	name={PFS},
	description={\emph{Perfect Forward Secrecy}. Własność bezpiecznego protokołu komunikacji, która zapewnia, że~skompromitowanie klucza długoterminowego, takiego jak np.~klucza prywatnego w~\gls{rsa}, nie skompromituje kluczy sesyjnych~\cite{wiki:pfs}. W~konsekwencji własność ta~zabezpiecza przed odszyfrowaniem zaszyfrowanej komunikacji przez atakującego, nawet jeśli atakujący wejdzie w~posiadanie kluczy prywatnych rozmówców.}
}
\newglossaryentry{openssl}
{
	name={OpenSSL},
	description={Popularna, wieloplatformowa biblioteka kryptograficzna napisana w~języku~C, opublikowana na~licencji typu \glslink{copyleft}{copyleft}~\cite{openssl}.}
}
\newglossaryentry{ssl/tls}
{
	name={SSL/TLS},
	description={Protokół kryptograficzny zapewniający bezpieczeństwo transmisji danych w~sieciach komputerowych. Często utożsamiany z~nazwą SSL ze~względu na~to, że~protokół TLS jest następcą SSL, a~nazwa SSL została przez lata rozpowszechniona. Przykładowymi implementacjami \gls{ssl/tls} są biblioteki: \gls{openssl}, GnuTLS, LibreSSL i~BoringSSL --- dwie ostatnie z~nich bazują na~kodzie źródłowym \gls{openssl}.}
}
\newglossaryentry{ca}
{
	name={CA},
	description={\emph{Certificate Authority}. Urząd lub~centrum certyfikacji, czyli podmiot wystawiający certyfikaty cyfrowe, potwierdzajce własność klucza publicznego przez wskazanie podmiotu certyfikatu.}
}
\newglossaryentry{aes}
{
	name={AES},
	description={\emph{Advanced Encryption Standard}, znany też~pod pierwotną nazwą \emph{Rijandel}. Symetryczny szyfr blokowy przyjęty przez \emph{National Institute of Standards and Technology~(NIST)} jako standard FIPS-197 w~wyniku konkursu ogłoszonego w~1997~roku.~\cite{aes-fips197}}
}
\newglossaryentry{ietf}
{
	name={IETF},
	description={\emph{Internet Engineering Task Force}. Międzynarodowa organizacja powołana w~1986~roku przez rząd Stanów Zjednoczonych, obecnie prowadzona pod~auspicjami stowarzyszenia \emph{Internet Society (ISOC)}, zajmująca się standardyzacją techniczną Internetu, w~szczególności protokołami protokołami sieciowymi. Efektem prac~\gls{ietf} są~dokumenty~\gls{rfc}.}
}
\newglossaryentry{iana}
{
	name={IANA},
	description={Organizacja niegdyś należąca do~\gls{ietf}, której zadaniem jest zarządzanie systemem numeracji i~nazewnictwa w~kontekście Internetu i~protokołów sieciowych. W~szczególności IANA~odpowiada za~nadawanie znaczenia zakresom dresów IP, przyporządkowywaniem usług do~portów sieciowych~itp. Twórcą IANA był dr~Jonathan Bruce Postel.}
}
%\newacronym{cpu}{CPU}{\emph{Central Processing Unit}. Procesor}
%\newacronym{api}{API}{\emph{Application Programming Interface}}
%\newacronym{bom}{BOM}{\emph{Byte Order Mark}. Znacznik kolejności bajtów}
%\newacronym{ide}{IDE}{\emph{Integrated Development Environment}. Zintegrowane Środowisko Programistyczne}
