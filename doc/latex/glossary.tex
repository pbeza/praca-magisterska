%
% Ze względu na custom'owy styl zastosowany w glossaries zamiast '\\' używamy '\\&' (bo słownik to tabelka dwukolumnowa).
% Na końcu każdego wpisu w słowniku dodaje się automatycznie znak kropki '.' i nowej linii '\\'.
%

\newglossaryentry{gnu}
{
	name={GNU},
	description={\emph{GNU’s not Unix}.\\&Uniksopodobny system operacyjny, zapoczątkowany przez Richarda Stallmana w styczniu~1984~roku, złożony wyłącznie z~\glslink{wolne-oprogramowanie}{wolnego oprogramowania}. W~ramach projektu \gls{gnu} powstały m.in.~\emph{gcc}, \emph{glibc}, \emph{bash}, \emph{emacs} i~jądro systemu operacyjnego --- \emph{GNU~Hurd} --- jednak jego popularność jest dużo mniejsza od~jądra \gls{linux}. System operacyjny \gls{gnu} z~jądrem \gls{linux} nazywamy dystrybucją \gls{gnulinux}~\cite{gnu,gnulinux,gnu-faq,wiki:linux-naming-controversy}}
}
\newglossaryentry{linux}
{
	name={Linux},
	description={Uniksopodobne jądro systemu operacyjnego będące \glslink{wolne-oprogramowanie}{wolnym oprogramowaniem}, zapoczątkowane przez Linusa Torvaldsa w~1991~roku~\cite{linux-kernel}. Często mylnie lub~potocznie utożsamiany z~określeniem \gls{gnulinux}~\cite{gnu-faq,wiki:linux-naming-controversy}}
}
\newglossaryentry{unix}
{
	name={Unix},
	description={Rodzina systemów operacyjnych, mających wspólny początek w~postaci systemu \gls{unix} powstałego w~1969~roku w~\emph{AT\&T} (obecnie \emph{Bell~Labs}). Korzystając z~kodu \glslink{unix}{Unix'a} powstało wiele pokrewnych systemów operacyjnych, m.in.~\emph{BSD}, \emph{Solaris}, \emph{Sun~OS}, \emph{System~V}, \emph{AIX}, \emph{HP-UX}, \emph{Mac~OS~X}. Projekt \gls{gnu} był od~początku projektowany jako ulepszony \gls{unix}, ale nie bazujący na~jego kodzie źródłowym~\cite{gnu-initial-msg}}
}
\hyphenation{GNU/Linux}
\newglossaryentry{gnulinux}
{
	name={GNU/Linux},
	description={System operacyjny \gls{gnu} zapoczątkowany przez Richarda Stallmana, z~jądrem \gls{linux}, zapoczątkowanym przez Linusa Torvaldsa~\cite{gnulinux}. Często mylnie lub~potocznie utożsamiany z~określeniem \gls{linux}~\cite{gnu-faq,wiki:linux-naming-controversy}}
}
\newglossaryentry{kernel}
{
	name={kernel},
	description={Jądro systemu \gls{gnulinux}.\\&Centralna część systemu operacyjnego, mająca całkowitą kontrolę nad nim. W~szczególności, jądro systemu operacyjnego jest odpowiedzialne za~zarządzanie pamięcią, komunikacją między urządzeniami, obsługę sterowników i~za~inne, niskopoziomowe zadania}
}
\newglossaryentry{UTF8}
{
	name={UTF-8},
	description={\emph{UCS Transformation Format 8-bit}.\\&System kodowania \emph{Unicode}, wykorzystujący od 8~do 32~bitów do zakodowania pojedynczego znaku, w~pełni kompatybilny z~\gls{ASCII}}
}
\newglossaryentry{PWD}
{
	name={\texttt{\$PWD}},
	description={\emph{\textbf{P}athname of the current \textbf{W}orking \textbf{D}irectory}.\\&Zmienna środowiskowa systemu \gls{gnulinux} przechowująca ścieżkę aktualnego katalogu roboczego}
}
\newglossaryentry{wolne-oprogramowanie}
{
	name = {Wolne oprogramowanie},
	description = {\emph{Free Software}.\\&Oprogramowanie publikowane na licencji typu \glslink{copyleft}{copyleft}, które zezwala użytkownikom je~uruchamiać, powielać, badać, zmieniać i~ulepszać~\cite{free-software}}
}
\newglossaryentry{gpl}
{
	name={licencja GPL},
	description={\emph{GNU General Public License}.\\&Licencja \glslink{wolne-oprogramowanie}{wolnego oprogramowania} stworzona w~1989~roku przez Richarda Stallmana, gwarantująca użytkownikowi możliwość uruchamiania, kopiowania i~modyfikowania oprogramowania, pod~warunkiem, że~zmodyfikowana wersja zostanie również opublikowana na~warunkach tej~licencji. Najnowszą wersją \glslink{gpl}{GPL} jest \emph{GPL~3}, wydana 29~czerwca~2007~roku~\cite{gpl3,wiki:licenses-comparison}}
}
\newglossaryentry{lgpl}
{
	name={licencja LGPL},
	description={\emph{GNU Lesser General Public License}.\\&Licencja typu \glslink{copyleft}{copyleft}, bazująca na~licencji \glslink{gpl}{GPL} --- różni się od niej tylko tym, że~wykorzystanie \glslink{lgpl}{LGPL} dopuszcza korzystanie z~biblioteki w~programach prawnie zastrzeżonych~\cite{lgpl3,why-not-lgpl,wiki:licenses-comparison}}
}
\newglossaryentry{mit-license}
{
	name={licencja MIT},
	description={Licencja \emph{Massachusetts Institute of Technology}.\\&Licencja typu \glslink{copyleft}{copyleft}, kompatybilna z~wieloma innymi \glslink{wolne-oprogramowanie}{wolnymi} licencjami --- np.~z~licencją~\glslink{gpl}{GPL}~\cite{mit}. W~projekcie~\emph{Fedora} i~na~\emph{GitHub} jest to najczęściej używana licencja~\cite{mit-popularity-fedora,mit-popularity-github,wiki:licenses-comparison}}
}
\newglossaryentry{apache2.0-license}
{
	name={licencja Apache~2.0},
	description={Licencja typu \glslink{copyleft}{copyleft}~opublikowana w~styczniu~2004~roku, zezwalająca użycie oprogramowania objętą tą licencją również na użytek zamkniętego oprogramowania komercyjnego~\cite{apache2.0,wiki:licenses-comparison}. Jest kompatybilna z licencją \glslink{gpl}{GPL~3}, ale nie z jej poprzednimi wersjami}
}
\newglossaryentry{bsd-license}
{
	name={licencje BSD},
	description={Rodzina licencji typu \glslink{copyleft}{copyleft}, zezwalająca na~rozprowadzanie oprogramowania bez postaci źródłowej oraz na~włączenia do~zamkniętego oprogramowania, pod warunkiem załączenia do~oprogramowania informacji o~autorach oryginalnego kodu i~treści licencji~\cite{bsd,wiki:licenses-comparison}}
}
\newglossaryentry{copyleft}
{
	name={licencje copyleft},
	description={Licencje umożliwiające uczynienie programu \glslink{wolne-oprogramowanie}{wolnym oprogramowaniem} i~nakazanie, by~wszystkie jego zmienione i~poszerzone wersje również takimi były~\cite{copyleft,wiki:licenses-comparison}}
}
\newglossaryentry{ssh}
{
	name={SSH},
	description={\emph{Secure Shell}.\\&Protokół sieciowy działający w~trybie klient-serwer, zapewniający bezpieczne zdalne logowanie i~inne bezpieczne usługi sieciowe~\cite{rfc:ssh}. Serwer \gls{ssh} działa domyślnie na~porcie~22}
}
\newglossaryentry{tcpip}
{
	name={TCP/IP},
	description={Model warstwowej struktury protokołów komunikacyjnych bazujący na~modelu \emph{Department-of-Defence~(DoD)}~\cite{rfc:tcpip-dod}}
}
\newglossaryentry{signal-handler}
{
	name={signal handler},
	description={Funkcja, która zostaje wywołana przez system operacyjny gdy~program otrzymuje od~systemu niezamaskowany sygnał~\cite{signal-handler}}
}
\newglossaryentry{fifo}
{
	name={FIFO},
	description={\emph{First In, First Out} --- \emph{łącze nazwane}.\\&W~ogólności \gls{fifo} to~sposób organizacji i~manipulacji danymi, w~szczególności danymi na~stosie. W~kontekście niniejszej pracy pisząc o~\gls{fifo}, mamy na~myśli plik specjalny podobny w~działaniu do~łączy nienazwanych~(potoków) --- różni się od~nich tym, że~dostęp do~\gls{fifo} odbywa się przez system plików~\cite{fifo-manual}}
}
\newglossaryentry{socket}
{
	name={socket},
	description={W~\glslink{unix-like-system}{systemach *niksowych}~plik specjalny reprezentujący dwukierunkowy punkt końcowy połączenia --- najczęściej sieciowego (wyjątek stanowią \emph{Unix sockets})~\cite{socket-definition-oracle}}
}
\newglossaryentry{posix}
{
	name={\mbox{POSIX}},
	description={\emph{Portable Operating System Interface for Unix}.\\&Rodzina standardów opublikowana przez \emph{IEEE Computer Society}, mająca na~celu zapewnienie zgodności pomiędzy systemami operacyjnymi m.in.~na~poziomie \acrshort{api}. W~czasie pisania niniejszej pracy, najnowsza wersja standardu \gls{posix} została opublikowana w~kwietniu~2013~roku~\cite{posix-ieee,posix-opengroup}}
}
\newglossaryentry{unix-like-system}
{
	name={system *niksowy},
	description={System operacyjny zbliżony budową do systemu \gls{unix}, np.~\gls{gnulinux}, \emph{bsd}, \emph{Solaris}, \emph{HP-UX}, \emph{AIX}~itp.~\cite{wiki:unix-like}}
}
\newacronym{cpu}{CPU}{\emph{Central Processing Unit}.\\&Procesor}
\newacronym{api}{API}{\emph{Application Programming Interface}}
\newacronym{bom}{BOM}{\emph{Byte Order Mark}.\\&Znacznik kolejności bajtów}
\newacronym{ide}{IDE}{\emph{Integrated Development Environment}.\\&Zintegrowane Środowisko Programistyczne}
