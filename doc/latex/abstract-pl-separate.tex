\documentclass[11pt,a4paper]{article}
\usepackage[english,main=polish]{babel}
\usepackage{graphicx}
\usepackage[hidelinks]{hyperref}
\usepackage[parfill]{parskip}[2001/04/09]
\usepackage[utf8]{inputenc}
\usepackage[T1]{fontenc}
\usepackage{mathptmx}
\usepackage{hyphenat}

\hyphenation{Linux}

\linespread{1.15}
\setlength{\parindent}{0cm}

\pagestyle{empty} % no page numbers!
\usepackage[left=28mm, right=28mm, top=13mm, bottom=18mm, noheadfoot]{geometry}

\begin{document}

\begin{titlepage}
	\let\endtitlepage\relax
	\center
	\begin{minipage}{2.3cm}
		\hspace*{-0.8cm}\includegraphics[width=2.3cm]{img/mini}
	\end{minipage}
	\hfill
	\begin{minipage}{2.3cm}
		\hspace*{0.8cm}\includegraphics[width=2.3cm]{img/pw}
	\end{minipage}\par
	\global\let\newpagegood\newpage
	\global\let\newpage\relax
\end{titlepage}

\thispagestyle{empty}
\title{\textbf{Protokół zarządzania stacjami komputerowymi\\pod~kontrolą systemu Linux}}
\author{\texorpdfstring{%
	\footnotesize
	\begin{minipage}{.5\textwidth}
	\centering
	\emph{Autor:} Patryk Bęza\\[-1pt]
	\texttt{\href{mailto:P.Beza@student.mini.pw.edu.pl}{P.Beza@student.mini.pw.edu.pl}}
	\end{minipage}%
	\begin{minipage}{.5\textwidth}
	\centering
	\emph{Promotor:} dr inż.~Marek Kozłowski\\[-1pt]
	\texttt{\href{mailto:M.Kozlowski@mini.pw.edu.pl}{M.Kozlowski@mini.pw.edu.pl}}
\end{minipage}}{The Author}}

%\small Wydział Matematyki i~Nauk Informacyjnych\\[-8pt]
%\small Politechnika Warszawska\\[-8pt]
%\small ul.~Koszykowa 75, 00-662~Warszawa

\date{}

\maketitle
\providecommand{\keywords}[1]{\textbf{\textit{Słowa kluczowe ---}} #1}
\keywords{\emph{\href{https://en.wikipedia.org/wiki/Software_configuration_management}{Software Configuration Management~(SCM)}}, \emph{\href{https://en.wikipedia.org/wiki/Infrastructure_as_Code}{Infrastructure as Code~(IaC)}}, \emph{\href{https://en.wikipedia.org/wiki/Linux}{Linux}}, \emph{\href{https://en.wikipedia.org/wiki/Communications_protocol}{Communications Protocol}}}
\global\let\newpage\newpagegood
\thispagestyle{empty}

Niniejsza praca pt.~,,Protokół zarządzania stacjami komputerowymi pod~kontrolą systemu Linux'' dotyczy problemu automatyzacji jednakowej lub~podobnej konfiguracji grupy komputerów. Jej~celem jest zaprojektowanie i~zaimplementowanie alternatywnego, w~zamyśle wygodniejszego od~istniejących, sposobu przygotowania i~propagowania zmian konfiguracji systemu wzorcowego --- w~skrócie nazywanego serwerem --- do~stacji klienckich.% Główną cechą odróżniającą opracowany protokół od~konkurencyjnych rozwiązań jest uproszczona metoda generowania zmian zachodzących na~serwerze, tzn.~na~maszynie zawierającej wzorzec oprogramowania.

Istniejące aplikacje typu SCM (\emph{Software Configuration Manager}) rozwiązujące przedstawiony problem --- takie jak np.~omówione w~pracy \emph{Puppet}, \emph{Chef}, \emph{Ansible} i~\emph{Salt} --- korzystają dedykowanych języków metakonfiguracji (\emph{Domain Specific Language}), w~których można deklaratywnie opisać oczekiwaną konfigurację klientów, tzn.~nie specyfikując konkretnych poleceń do~wykonania, tylko odwołując się do~pożądanego stanu np.~pliku, pakietu lub~użytkownika. Dopiero na~podstawie takiego opisu, będącego warstwą abstrakcji nad~właściwą konfiguracją, klient podejmuje działania mające na~celu dostosowanie się do~wymagań w~nim zapisanych. Stworzona warstwa abstrakcji pozwala na~obsługę różnych dystrybucji i~systemów operacyjnych, jednak powstanie takiej możliwości jest okupione utratą czasu jaki administrator musi poświęcić na~naukę języka wybranego narzędzia oraz~na~wyrażenie w~nim docelowych ustawień kontrolowanych przez siebie komputerów.

Rozwiązanie przedstawione w~ramach tej~pracy nie definiuje własnego, pośredniego języka metakonfiguracji, przez co, z~jednej strony maszyna klienta i~serwera muszą mieć zainstalowane dystrybucje z~co~najmniej identycznym drzewami najważniejszych katalogów (FHS --- \emph{Filesystem Hierarchy Standard}) oraz takimi samymi menadżerami pakietów. Z~drugiej strony, przyjęty niekonwencjonalny pomysł implementacji pozwolił na~stworzenie aplikacji, która ułatwia i~minimalizuje pracę administratora nad~iteracyjnym dostosowywaniem zarządzanych przez siebie systemów, w~konsekwencji zmniejszając ryzyko popełnienia błędów konfiguracji. Sposób działania zaprojektowanego oprogramowania po~stronie serwera składa się z~trzech, wykonywanych na~przemian, kroków --- modyfikowaniu systemu wzorcowego tak jakby był klientem, skanowaniu go w~celu wygenerowania migawki jego stanu oraz automatycznym utworzeniu obrazu zmian konfiguracji stacji wzorcowej na~podstawie migawek, dającym się zastosować na~maszynach klienckich. Przygotowana implementacja pozwala na~wykorzystanie szablonów plików zawierających zmienne (\emph{placeholdery}), które podczas zastosowywania obrazu zmian zostają zamienione na~konkretne wartości --- np.~na~nazwę sieciową komputera lub~wybraną zmienną środowiskową. Wymiana obrazów zmian może zachodzić przez sieć zarówno między serwerem i~klientami, jak i~między samymi klientami.

\end{document}
