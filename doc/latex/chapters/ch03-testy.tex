\documentclass[thesis]{subfiles}

\begin{document}

\chapter{Testy}
\label{chapter:testy}

Przeprowadzone testy przeprowadzono na~kilku komputerach klasy~PC połączonych siecią lokalną \emph{ethernet}. Niniejszy rozdział opisuje przeprowadzone różnego rodzaju testy, zarówno z~wykorzystaniem infrastruktury sieciowej, jak i~bez niej. W~szczególności opisano testy funkcjonalne, wydajnościowe, wycieków zasobów (w~szczególności pamięci).

%------------------------------------------------------------------------------

\section{Testy wycieków pamięci}

Do~testów kodu pod~kątem sprawdzenia wycieków pamięci wykorzystano program \texttt{valgrind} w~najnowszej dostępnej wersji~\texttt{3.12.0}~\cite{valgrind}.

%------------------------------------------------------------------------------
%
%\section{Monitorowanie wykorzystania zasobów}
%
%\noindent Poniżej przedstawiono testy użycia zasobów, takich jak:
%\begin{itemize}
%	\item pamięci,
%	\item plików,
%	\item \glslink{socket}{socketów},
%	\item \gls{fifo} (potok nazwany).
%\end{itemize}
%
%------------------------------------------------------------------------------

\section{Testy funkcjonalne}

TODO

\end{document}
