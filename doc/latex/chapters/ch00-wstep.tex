\documentclass[thesis]{subfiles}

% Umożliwienia odwoływania się do polskiego tytułu pracy

\makeatletter
\let\inserttitle\@title
\let\inserttitleaux\@titleaux
\makeatother

\begin{document}

\chapter{Wstęp}
\label{chapter:intro}

\section{Temat pracy}

\noindent Tematem niniejszej pracy jest:
\begin{displayquote}
\inserttitle.
\end{displayquote}
Wersja angielska tytułu pracy:
\begin{displayquote}
\inserttitleaux.
\end{displayquote}

\section{Cel i zakres pracy}
\label{cel-i-zakres}

\noindent Celem niniejszej pracy jest:
\blockcquote{formularz-zgloszenia-pracy}{Zaprojektowanie i~zaimplementowanie protokołu umożliwiającego propagowanie zmian w~systemie plików, pakietów oraz elementów konfiguracji do~stacji roboczych pod kontrolą systemu operacyjnego \glslink{gnulinux}{Linux/GNU} lub~innego systemu \emph{*nix}, a~także elastycznego standardu opisu zmian oraz narzędzi dla~ich rejestrowania i~dostosowywania. Proponowaną architekturą jest model klient--serwer. Konkretny obraz zmian może być przeznaczony dla~konkretnej dystrybucji, jednakże stworzone rozwiązania praktyczne powinny dawać się w~prosty sposób dostosowywać do~popularnych dystrybucji systemu \glslink{gnulinux}{Linux/GNU} oraz systemów \emph{*nix}.}
W~skrócie, cel niniejszej pracy to zaprojektowanie i~zaimplementowanie systemu synchronizowania oprogramowania, instalowanego na~grupie komputerów pracujących pod kontrolą systemu operacyjnego~\glslink{gnulinux}{GNU/Linux}.

\section{Motywacja}

Motywacją do~napisania pracy magisterskiej na~przedstawiony temat jest problem, z~którym zmagają się~administratorzy głównie średnich i dużych sieci komputerowych na~całym świecie, niezależnie od~używanego przez nich systemu operacyjnego --- problem automatycznego dystrybuowania ujednoliconego oprogramowania i~jego konfiguracji na~komputery połączone siecią komputerową~\cite{so-problem-intro}. W~ogólności nie~muszą to~być komputery takie jak serwery czy stacje robocze --- choć to~głównie z~nimi administratorzy komputerowi mają do~czynienia. Omawiany problem może dotyczyć również niektórych połączonych w~sieć systemów wbudowanych nie będących systemami typu~\mbox{\emph{bare~metal}}, tzn.~pracujących pod kontrolą systemu operacyjnego.

W praktyce problem dystrybuowania jednolitego oprogramowania jest szczególnie odczuwalny przez administratorów, którzy zarządzają dużym zbiorem komputerów --- serwerów lub~stacji roboczych --- z~których każdy musi mieć ten~sam lub podobny zestaw zainstalowanych programów i~konfiguracji dedykowanej dla~zainstalowanego oprogramowania. Popularne sposoby radzenia sobie z~tym problemem~(patrz rozdział~\ref{sec:istniejace-rozwiazania}) są~uciążliwe w~praktyce, a~istniejące rozwiązania w~pełni automatyzujące cały proces synchronizacji oprogramowania i~konfiguracji, są stosunkowo młode i~słabo rozpowszechnione wśród większości administratorów komputerowych.

\section{Zastosowania rozwiązania problemu}

Zastosowania rozwiązania problemu dystrybuowania jednolitego oprogramowania możemy znaleźć wszędzie tam gdzie istnieje grupa ludzi, procesów, maszyn lub~innych podmiotów, z~których każdy wykorzystuje to~samo lub podobnie skonfigurowane oprogramowanie na~urządzeniu wyposażonym w~system operacyjny, podłączonym do~sieci komputerowej, najlepiej opartej o~\gls{tcpip}. W~szczególności mogą to~być niektóre: grupy serwerów pracujących w~podobnym środowisku, np.~maszyny developerskie, testowe i~produkcyjne, serwery działające w~ramach systemu rozproszonego, laboratoria komputerowe na~uczelniach, w szkołach, bibliotekach, biura, komputery administracji publicznej, w~szczególności w~urzędach, bankomaty, biletomaty, maszyny vendingowe, komputery kasjerów w~hipermarketach~itp.

\section{Istniejące rozwiązania}
\label{sec:istniejace-rozwiazania}

W~niniejszym rozdziale przedstawiono przegląd istniejących rozwiązań dla~problemu dystrybuowania oprogramowania na~wiele komputerów połączonych siecią komputerową.

\subsection{\texttt{Puppet}}

\section{Uwagi licencyjne}

Niniejsza praca wpisuje się w~tematykę \glslink{wolne-oprogramowanie}{wolnego oprogramowania} i~w~związku z~tym wszystkie narzędzia, biblioteki, programy, kody źródłowe, grafiki wykorzystane w czasie implementacji i~dokumentacji niniejszego projektu są~licencjonowane na~warunkach co~najmniej jednej z~licencji typu~\glslink{copyleft}{copyleft}, takiej jak np.~\gls{gpl}, \gls{lgpl}, \gls{mit-license}, \gls{bsd-license}, \gls{apache2.0-license}, o~ile nie zaznaczono inaczej.

W~związku z~obowiązującym \emph{art.~15a~ustawy z~dnia 4~lutego~1994r. o~prawie autorskim i~prawach pokrewnych}~\cite{papp}, kod źródłowy aplikacji powstałej w~ramach niniejszej pracy, może zostać opublikowany, np.~na~warunkach licencji~\glslink{gpl}{GNU~GPL~3}, po~upływie co~najmniej 6~miesięcy od~momentu obrony autora niniejszej pracy, pod warunkiem, że~Uczelnia, tj.~Politechnika Warszawska, sama nie~opublikuje jej przez ten~czas. Jeśli warunek ten~będzie spełniony, aplikacja zostanie opublikowana w~ramach repozytorium na~serwisie~\hreftt{https://github.com/}{github.com}.

\newpage

\section{Środowisko projektu}

\section{Zawartość pracy}

\noindent Niniejsza praca magisterska składa się z~dwóch głównych części:
\begin{enumerate}
	\item Niniejszej dokumentacji napisanej w \LaTeX,
	\item Aplikacji, w~tym wersji skompilowanej, kodu źródłowego i~dokumentacji wygenerowanej z~komentarzy w~kodzie.
\end{enumerate}
Aplikacja składa się~również z~dwóch części, co~jest naturalne biorąc pod uwagę przyjętą architekturę klient-serwer~(por.~rozdział~\ref{cel-i-zakres}):
\begin{enumerate}
	\item Serwera --- wzorca oprogramowania i~konfiguracji dla~klientów,
	\item Klienta --- okresowo synchronizującego oprogramowanie i~konfigurację z serwerem.
\end{enumerate}

\section{Wykorzystane źródła wiedzy}

W~czasie prac nad implementacją projektu, korzystano z~wielu pomocnych dokumentacji, materiałów dydaktycznych i~innych źródeł wiedzy takich jak np.~programistyczne dyskusyjne, fora programistyczne, np.~\hreftt{http://stackoverflow.com/}{stackoverflow.com}, \hreftt{http://superuser.com/}{superuser.com}.

\noindent Na~szczególne wyróżnienie zasługują źródła, z~których korzystano najwięcej:
\begin{enumerate}
	\item dokumentacje:
	\begin{itemize}
		\item dokumentacja TLDP~\cite{tldp},
		\item dokumentacja deweloperów dla wybranych dystrybucji/systemów~\cite{archlinux-wiki,gentoo-wiki},
		\item dokumentacja POSIX~\cite{posix},
		\item dokumentacja wybranego języka programowania i innych użytych narzędzi~\cite{glibc-doc},
		\item (opcjonalnie) dokumenty IETF (RFC)~\cite{rfc-editor},
		\item dokumenty W3C~\cite{w3c},
		\item \hreftt{http://www.tcpipguide.com/}{tcpipguide.com}, czyli szczegółowy przegląd protokołu TCP/IP i~pokrewnych protokołów, opublikowany również w~formie obszernej\footnote{Książka ma~1616 stron.} książki autorstwa \mbox{Charlesa~M.~Kozieroka} pt.~\textit{The~TCP/IP Guide}\footnote{ISBN:~\href{https://www.nostarch.com/tcpip.htm}{9781593270476}. Wydawnictwo \emph{No Starch Press}.}.
	\end{itemize}
	\item materiały dydaktyczne do~przedmiotów prowadzonych na~Wydziale~MiNI~PW:
	\begin{itemize}
		\item ,,TCP/IP'' --- przedmiot prowadzony przez dra~inż.~Marka~Kozłowskiego~\cite{kozlowski},
		\item ,,UNIX'' --- przedmiot prowadzony przez mgra~inż.~Marcina~Borkowskiego~\cite{borkowski}.
	\end{itemize}
\end{enumerate}
Kompletny spis bibliografii, cytowanej w~niniejszej dokumentacji, znajduje się~na~jej końcu.

\newpage

\end{document}
