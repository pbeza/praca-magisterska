\documentclass[thesis]{subfiles}

% Umożliwienia odwoływania się do polskiego tytułu pracy

\makeatletter
\let\inserttitle\@title
\let\inserttitleaux\@titleaux
\makeatother

\begin{document}

\chapter{Wstęp}

%------------------------------------------------------------------------------

\section{Temat pracy}

\noindent Tematem niniejszej pracy jest:
\begin{displayquote}
\inserttitle.
\end{displayquote}
Angielska wersja tytułu pracy:
\begin{displayquote}
\inserttitleaux.
\end{displayquote}
Wybór tematu pracy jest podyktowany zainteresowaniami autora pracy dotyczącymi m.in.~sytemów \glslink{unix-like-system}{*nixowych}, protokołów sieciowych, ich bezpieczeństwa oraz~problemem zautomatyzowanego dystrybuowania oprogramowania, który dotyka wiele firm, o~czym może świadzyć lista firm, uniwersytetów i~instytucji, które korzystają z~istniejących rozwiązań tego problemu przedstawionych w~rozdziale \ref{ch:istniejace-rozwiazania}. Problem ten dotyknął również Wydział~MiNI~PW, co~stało się bezpośrednią motywacją do~alternatywnego rozwiązania tego problemu.

%------------------------------------------------------------------------------

\section{Cel i~zakres pracy}
\label{sec:cel-i-zakres}

\noindent Celem niniejszej pracy jest:
\blockcquote{formularz-zgloszenia-pracy}{Zaprojektowanie i~zaimplementowanie protokołu umożliwiającego propagowanie zmian w~systemie plików, pakietów oraz~elementów konfiguracji do~stacji roboczych pod~kontrolą systemu operacyjnego \glslink{gnulinux}{Linux/GNU} lub~innego systemu \glslink{unix-like-system}{*nix}, a~także elastycznego standardu opisu zmian oraz~narzędzi dla~ich rejestrowania i~dostosowywania. Proponowaną architekturą jest model klient--serwer. Konkretny obraz zmian może być przeznaczony dla~konkretnej dystrybucji, jednakże stworzone rozwiązania praktyczne powinny dawać się w~prosty sposób dostosowywać do~popularnych dystrybucji systemu \glslink{gnulinux}{Linux/GNU} oraz~systemów \glslink{unix-like-system}{*nix}.}
W~uproszczeniu, celem niniejszej pracy jest zaprojektowanie i~zaimplementowanie narzędzia dla~systemów \glslink{unix-like-system}{*niksowych} do~zarządzania synchronizacją oprogramowania i~jego konfiguracji na~grupie komputerów podłączonych do~sieci opartej na~protokole~\gls{tcpip}. Oprogramowanie takie często jest nazywane oprogramowaniem typu \hrefemph{https://en.wikipedia.org/wiki/Software_configuration_management}{Software Configuration Management~(SCM)}~\cite{wiki:scm}, a~sposób jego działania nazywany jest \hrefemph{https://en.wikipedia.org/wiki/Infrastructure_as_Code}{Infrastructure as Code~(IaC)}.

%------------------------------------------------------------------------------

\section{Motywacja}

Motywacją do~napisania pracy magisterskiej na~przedstawiony w~rozdziale~\ref{sec:cel-i-zakres} temat, jest problem, z~którym zmagają się~administratorzy głównie średnich i~dużych sieci komputerowych na~całym świecie, niezależnie od~używanego przez nich systemu operacyjnego --- problem automatycznego dystrybuowania ujednoliconego oprogramowania i~jego konfiguracji na~komputery połączone siecią komputerową~\cite{so-problem-intro}. W~ogólności nie~muszą to~być komputery takie jak serwery czy stacje robocze --- choć to~głównie z~nimi administratorzy komputerowi mają do~czynienia. Omawiany problem może dotyczyć również niektórych systemów wbudowanych nie będących systemami typu~\mbox{\hrefemph{https://en.wikipedia.org/wiki/Bare_machine}{bare~metal}}, tzn.~pracujących pod~kontrolą systemu operacyjnego. Przykładem takich urządzeń są~niektóre biletomaty, bankomaty, maszyny \emph{vendingowe} i~wyświetlacze np.~na lotniskach, peronach i~przystankach komunikacji miejskiej.

W praktyce problem dystrybuowania jednolitego oprogramowania jest szczególnie odczuwalny przez administratorów, którzy zarządzają dużą ilością komputerów --- serwerów lub~stacji roboczych --- z~których każdy musi mieć ten~sam lub~podobny zestaw zainstalowanych programów i~konfiguracji dedykowanej dla~zainstalowanego oprogramowania. Popularne sposoby radzenia sobie z~tym problemem są~w~praktyce uciążliwe, a~istniejące rozwiązania w~pełni automatyzujące cały proces synchronizacji oprogramowania i~konfiguracji~(patrz rozdział~\ref{ch:istniejace-rozwiazania}), są stosunkowo młode i~słabo rozpowszechnione wśród większości administratorów komputerowych.

%------------------------------------------------------------------------------

\section{Zastosowania}

Zastosowania rozwiązania problemu dystrybuowania jednolitego oprogramowania można znaleźć wszędzie tam gdzie istnieje grupa ludzi, procesów lub~komputerów, z~których każdy wykorzystuje to~samo lub~podobnie skonfigurowane oprogramowanie na~urządzeniu wyposażonym w~system operacyjny, podłączonym do~sieci komputerowej. W~szczególności mogą to~być grupy serwerów pracujących w~podobnym środowisku, np.~maszyny \emph{developerskie}, testowe i~produkcyjne, \emph{load-balancery}, serwery działające w~ramach systemu rozproszonego, laboratoria komputerowe na~uczelniach, w~szkołach i~bibliotekach, biura, komputery administracji publicznej (w~szczególności w~urzędach), bankomaty, biletomaty, automaty sprzedające~(tzn.~automaty \emph{vendingowe}), komputery osobiste osób korzystających z~kilku(nastu) komputerów, komputery kasjerów w~hipermarketach, komputery obsługujące wyświetlacze informacyjne np.~na~przystankach komunikacji miejskiej, peronach, galeriach handlowych~itp.

%------------------------------------------------------------------------------

\section{Zawartość pracy}

\noindent Niniejsza praca magisterska składa się z~dwóch części:\mynobreakpar
\begin{enumerate}
	\item Niniejszej dokumentacji napisanej w~\LaTeX, opisującej problem, istniejące rozwiązania i~projekt protokołu dla~przedstawionego problemu,
	\item Aplikacji napisanej w~języku Python oraz~towarzyszącej jej dokumentacji wygenerowanej z~komentarzy w~kodzie źródłowym.
\end{enumerate}
Aplikacja również składa~się z~dwóch części, co~jest naturalne dla~przyjętej architektury klient--serwer~(por.~rozdział~\ref{sec:cel-i-zakres}):\mynobreakpar
\begin{enumerate}
	\item Serwera --- wzorca oprogramowania i~konfiguracji dla~klientów,
	\item Klienta --- okresowo synchronizującego oprogramowanie i~konfigurację z~serwerem.
\end{enumerate}

W~czasie prac nad~implementacją projektu, korzystano z~pomocnych dokumentacji, materiałów dydaktycznych i~innych źródeł wiedzy, takich jak np.~listy dyskusyjne i~programistyczne fora \hreftt{http://stackoverflow.com/}{stackoverflow.com} i~\hreftt{http://superuser.com/}{superuser.com}.

\noindent Na~szczególne wyróżnienie zasługują następujące źródła:\mynobreakpar
\begin{enumerate}
	\item dokumentacje:\mynobreakpar
	\begin{enumerate}
		\item dokumentacja POSIX~\cite{posix},
		\item dokumentacja deweloperów dla wybranych dystrybucji/systemów~\cite{archlinux-wiki,gentoo-wiki},
		\item dokumentacja Python i~narzędzi towarzyszących~\cite{python-doc},%~\cite{glibc-doc},
		\item dokumenty \gls{ietf}~(\gls{rfc})~\cite{rfc-editor},
		\item dokumentacja TLDP~\cite{tldp},
		\item dokumenty W3C~\cite{w3c},
		\item \hreftt{http://www.tcpipguide.com/}{tcpipguide.com}, czyli szczegółowy przegląd protokołu TCP/IP i~pokrewnych protokołów, opublikowany również w~formie obszernej\footnote{Książka ma~1616 stron. ISBN:~\href{https://www.nostarch.com/tcpip.htm}{9781593270476}. Wydawnictwo: \emph{No Starch Press}.} książki autorstwa \href{https://www.linkedin.com/in/charles-kozierok-708112/}{\mbox{Charlesa} \mbox{Kozieroka}} pt.~\emph{The~TCP/IP Guide}.
		\item dokumentacja biblioteki OpenSSL~\cite{openssl-doc},
%		\item dokumentacja biblioteki libconfig~\cite{libconfig-doc}.
	\end{enumerate}
	\item materiały dydaktyczne do~przedmiotów prowadzonych na~Wydziale~MiNI~PW:\mynobreakpar
	\begin{enumerate}
		\item ,,TCP/IP'' --- przedmiot prowadzony przez dra~inż.~Marka~Kozłowskiego~\cite{kozlowski},
		\item ,,UNIX'' --- przedmiot prowadzony przez mgra~inż.~Marcina~Borkowskiego~\cite{borkowski}.
	\end{enumerate}
\end{enumerate}
Kompletny spis odwołań cytowanych w~niniejszej dokumentacji, znajduje się~na~jej końcu. %\hyperref[bibliography-page]{końcu}.

%------------------------------------------------------------------------------

\section{Uwagi licencyjne}

Niniejsza praca wpisuje się w~tematykę \glslink{wolne-oprogramowanie}{wolnego oprogramowania} i~dlatego wszystkie narzędzia, biblioteki, programy, kody źródłowe i~grafiki użyte w~czasie implementacji i~dokumentacji niniejszego projektu są~licencjonowane na~warunkach licencji typu~\glslink{copyleft}{copyleft} jeśli nie zaznaczono inaczej.

%Lista wykorzystanych bibliotek i~odpowiadających im~licencji, na~prawach której zostały opublikowane:
%\begin{itemize}
%\item \textbf{libconfig} --- licencja \glslink{lgpl}{\emph{LGPL}}~\cite{libconfig-webpage},
%\item \textbf{OpenSSL} --- podwójna licencja \emph{SSLeay License} i~\emph{OpenSSL License}~\cite{openssl-license}.
%\end{itemize}
%
%W~związku z~obowiązującym \emph{art.~15a~ustawy z~dnia 4~lutego~1994r. o~prawie autorskim i~prawach pokrewnych}~\cite{papp}, kod źródłowy aplikacji powstałej w~ramach niniejszej pracy, może zostać opublikowany na~warunkach licencji wybranej przez autora pracy po~upływie co~najmniej 6~miesięcy od~momentu obrony autora niniejszej pracy, pod~warunkiem, że~Uczelnia, tj.~Politechnika Warszawska, sama nie~opublikuje jej przez ten~czas. Jeśli warunek ten~będzie spełniony, aplikacja zostanie opublikowana w~repozytorium \hreftt{https://github.com/}{github.com} na warunkach zmodyfikowanej licencji \glslink{gpl}{GNU~GPL~3} z~dodanym wyjątkiem dla biblioteki OpenSSL. Bez~dodania wyjątku, podwójna licencja OpenSSL jest niekompatybilna z~rodziną licencji~\glslink{gpl}{GPL}~\cite{openssl-license-incompatibility,openssl-license-incompatibility-2}. Pełny tekst licencji znajduje się w~pliku tekstowym \path{LICENSE}, dołączonym do~projektu.

\end{document}
