\documentclass[thesis]{subfiles}

% Umożliwienia odwoływania się do polskiego tytułu pracy

\makeatletter
\let\inserttitle\@title
\let\inserttitleaux\@titleaux
\makeatother

\begin{document}

\chapter{Wstęp}
\label{chapter:intro}

%------------------------------------------------------------------------------

\section{Temat pracy}

\noindent Tematem niniejszej pracy jest:
\begin{displayquote}
\inserttitle.
\end{displayquote}
Wersja angielska tytułu pracy:
\begin{displayquote}
\inserttitleaux.
\end{displayquote}

%------------------------------------------------------------------------------

\section{Cel i zakres pracy}
\label{cel-i-zakres}

\noindent Celem niniejszej pracy jest:
\blockcquote{formularz-zgloszenia-pracy}{Zaprojektowanie i~zaimplementowanie protokołu umożliwiającego propagowanie zmian w~systemie plików, pakietów oraz elementów konfiguracji do~stacji roboczych pod kontrolą systemu operacyjnego \glslink{gnulinux}{Linux/GNU} lub~innego systemu \glslink{unix-like-system}{*nix}, a~także elastycznego standardu opisu zmian oraz narzędzi dla~ich rejestrowania i~dostosowywania. Proponowaną architekturą jest model klient--serwer. Konkretny obraz zmian może być przeznaczony dla~konkretnej dystrybucji, jednakże stworzone rozwiązania praktyczne powinny dawać się w~prosty sposób dostosowywać do~popularnych dystrybucji systemu \glslink{gnulinux}{Linux/GNU} oraz systemów \glslink{unix-like-system}{*nix}.}
W~uproszczeniu, celem niniejszej pracy jest zaprojektowanie i~zaimplementowanie narzędzia dla~systemów \glslink{unix-like-system}{*niksowych} do~zarządzania synchronizacją oprogramowania i~jego konfiguracji na~grupie komputerów. Oprogramowanie takie często nazywa się~oprogramowaniem typu \emph{Software Configuration Management~(SCM})~\cite{wiki:scm}.

%------------------------------------------------------------------------------

\section{Motywacja}

Motywacją do~napisania pracy magisterskiej na~przedstawiony temat jest problem, z~którym zmagają się~administratorzy głównie średnich i dużych sieci komputerowych na~całym świecie, niezależnie od~używanego przez nich systemu operacyjnego --- problem automatycznego dystrybuowania ujednoliconego oprogramowania i~jego konfiguracji na~komputery połączone siecią komputerową~\cite{so-problem-intro}. W~ogólności nie~muszą to~być komputery takie jak serwery czy stacje robocze --- choć to~głównie z~nimi administratorzy komputerowi mają do~czynienia. Omawiany problem może dotyczyć również niektórych połączonych w~sieć systemów wbudowanych nie będących systemami typu~\mbox{\emph{bare~metal}}, tzn.~pracujących pod kontrolą systemu operacyjnego.

W praktyce problem dystrybuowania jednolitego oprogramowania jest szczególnie odczuwalny przez administratorów, którzy zarządzają dużym zbiorem komputerów --- serwerów lub~stacji roboczych --- z~których każdy musi mieć ten~sam lub podobny zestaw zainstalowanych programów i~konfiguracji dedykowanej dla~zainstalowanego oprogramowania. Popularne sposoby radzenia sobie z~tym problemem są~uciążliwe w~praktyce, a~istniejące rozwiązania w~pełni automatyzujące cały proces synchronizacji oprogramowania i~konfiguracji~(patrz rozdział~\ref{sec:istniejace-rozwiazania}), są stosunkowo młode i~słabo rozpowszechnione wśród większości administratorów komputerowych.

%------------------------------------------------------------------------------

\section{Zastosowania}

Zastosowania rozwiązania problemu dystrybuowania jednolitego oprogramowania możemy znaleźć wszędzie tam gdzie istnieje grupa ludzi, procesów, maszyn lub~innych podmiotów, z~których każdy wykorzystuje to~samo lub podobnie skonfigurowane oprogramowanie na~urządzeniu wyposażonym w~system operacyjny, podłączonym do~sieci komputerowej opartej o~\gls{tcpip}. W~szczególności mogą to~być grupy serwerów pracujących w~podobnym środowisku, np.~maszyny developerskie, testowe i~produkcyjne, \emph{load-balancery}, serwery działające w~ramach systemu rozproszonego, laboratoria komputerowe na~uczelniach, w szkołach, bibliotekach, biura, komputery administracji publicznej, w~szczególności w~urzędach, bankomaty, biletomaty, automatach sprzedających~(automaty vendingowe), komputery kasjerów w~hipermarketach, komputery osobiste osób korzystających z~kilku(nastu) komputerów.

%------------------------------------------------------------------------------

\section{Istniejące rozwiązania}
\label{sec:istniejace-rozwiazania}

W~niniejszym rozdziale przedstawiono przegląd czterech wybranych, prawdopodobnie najpopularniejszych obecnie rozwiązań dla~problemu dystrybuowania oprogramowania na~sieć komputerów~\cite{leading-scms,wiki:scm-comparison}.

Wymienione w~tym rozdziale aplikacje, poza możliwością synchronizacji zainstalowanego oprogramowania i~konfiguracji, pozwalają również na~inne manipulacje systemem operacyjnym, np.~zapewnienie istnienia zadanego użytkownika w~systemie czy zapewnienie działanie zadanego \glslink{demon}{demona}. Takie oprogramowanie jest nazywane często w~skrócie~\emph{Software configuration management (SCM)} implementującym podejście \emph{Infrastructure as~Code}~\cite{wiki:iac,wiki:scm}.

%---------------------------------------

\subsection{Puppet}

Jednym z~najpopularniejszych istniejących rozwiązań problemu dystrybuowania oprogramowania jest projekt Puppet, który jest dedykowany nie tylko dla systemów~\glslink{unix-like-system}{*niksowych}, ale~również dla~systemów \emph{Windows} i~\emph{Mac~OS}~\cite{puppet,puppet-github,puppet-clients}.

Puppet został napisany w~języku programowania \texttt{Ruby}, a~jego rozwój jest wspierany i~wykorzystywany przez wiele czołowych firm i~organizacji w~branży~IT, takich jak: \emph{Google}, \emph{NASA}, \emph{Verizon}, \emph{Sony}, \emph{Disney}, \emph{Uber}, \emph{gettyimages}, \emph{CERN} i~wiele innych~\cite{puppet,puppet-google,puppet-cern,puppet-chef-disney}. Pierwsza wersja Puppet została opublikowana w~2005~roku na~licencji~\glslink{gpl}{GPL}. Od~wersji 2.0~Puppet jest licencjonowany na~warunkach licencji~\glslink{apache2.0-license}{Apache~2.0}. Istnieje również komercyjna wersja projektu, znana pod nazwą \emph{Puppet Enterprise}.

Konfiguracja Puppet polega na~napisaniu reguł tekstowych, niezależnych od~systemu operacyjnego (\emph{Domain Specific Language}), przechowywanych zwyczajowo w~plikach z~rozszerzeniem~\texttt{.pp}, zwanych manifestami. Reguły są~zapisane w~postaci deklaratywnej, a~nie imperatywnej, tzn.~opisują oczekiwany stan systemu, a~nie sposób przeprowadzenia zmian w~systemie. Struktura manifestu jest w~postaci tablicy asocjacyjnej (struktura danych \hreftt{https://docs.ruby-lang.org/en/trunk/Hash.html}{Hash}) z~języka programowania \texttt{Ruby}. Przykładowe manifesty przedstawiono na~listingach \ref{puppet-openssh-example}, \ref{puppet-motd-example} i~\ref{puppet-httpd-example}~\cite{puppet-examples}. Poprawność składni w~manifeście \path{example.pp} można zweryfikować komendą:
\begin{center}
\texttt{puppet parser validate example.pp}
\end{center}

Aby~wprowadzić zmiany opisane manifestem \texttt{example.pp}, wystarczy uruchomić komendę:
\begin{center}
\texttt{puppet apply example.pp}
\end{center}

Pomocna może być flaga \texttt{-{}-noop}, która uruchomia tryb \emph{dry run}, tzn.~tryb w~którym nie są dokonywane żadne zmiany w~systemie, a~na standardowym wyjściu są drukowane zmiany jakie by~zaszły, gdyby nie użyć tej flagi.

\begin{lstlisting}[numbers=none,caption={Manifest Puppet gwarantujący, że pakiet \texttt{openssh-client}, dostarczający klienta SSH, jest zainstalowany},label=puppet-openssh-example]
package { 'openssh-client':
  ensure => present,
}
\end{lstlisting}

\begin{lstlisting}[numbers=none,caption={Manifest Puppet gwarantujący istnienie pliku \hreftt{https://en.wikipedia.org/wiki/Motd_(Unix)}{/etc/motd} z~komunikatem \texttt{Hello!}, który wyświetla się~użytkownikom systemu po~zalogowaniu},label=puppet-motd-example]
file { 'motd':
  path => '/etc/motd',
  content => 'Hello!',
}
\end{lstlisting}

\begin{lstlisting}[numbers=none,caption={Manifest Puppet gwarantujący, że \gls{demon} \hreftt{https://httpd.apache.org/}{httpd} jest włączony i~działa},label=puppet-httpd-example]
service { 'httpd':
  ensure => running,
  enable => true,
}
\end{lstlisting}

W~nomenklaturze Puppeta istnieje co~najmniej kilka słów kluczowych, które odpowiadają abstrakcyjnym typom danych reprezentującym klientów i~ich oprogramowanie. Najważniejsze z~nich to:
\begin{itemize}
\item \texttt{class} --- Nazwany blok kodu Puppet przechowywany w~module. Służy do~konfiguracji dużej części funkcjonalności aplikacji. Konfiguracja węzła pozwala zadeklarować parametryzowane użycie wielu różnych klas. Deklaracja klasy może być co~najwyżej jedna.
\item \texttt{module} --- Zbiór klas konfigurujący całą aplikację. Społeczność Puppet dostarcza na~stronie internetowej \hreftt{https://forge.puppet.com/}{forge.puppet.com} wielu użytecznych modułów do~różnych rodzajów złożonego oprogramowania. Najpopularniejsze z~nich są~weryfikowane przez programistów Puppet i~otrzymują status \emph{Puppet Supported} lub~\emph{Puppet Approved}.
\item \texttt{node} --- Reprezentacja jednego lub wielu klientów.
\item wbudowane typy zasobów, takie jak: \texttt{cron}, \texttt{exec}, \texttt{file}, \texttt{group}, \texttt{host}, \texttt{maillist}, \texttt{mount}, \texttt{package}, \texttt{router}, \texttt{service}, \texttt{ssh\_authorized\_key}, \texttt{user}, \texttt{vlan}, \texttt{zone}, \texttt{zone\_pool} i~wiele innych. Listę tą można rozszerzać o~własne typy danych np.~przez ładowanie modułów~\cite{puppet-resources-types}.
\end{itemize}
W~praktyce pożądane jest automatyczne odpytywanie serwera przez klienta. Konfiguracja serwera odpowiadająca za~zdefiniowanie klas oprogramowania dla~poszczególnych klientów jest przechowywana w~pliku zwyczajowo nazywanym \path{site.pp}. Klient Puppeta domyślnie odpytuje serwer o~zmiany oprogramowania co~30~minut.

W~przykładach~\ref{puppet-openssh-example}, \ref{puppet-motd-example}, \ref{puppet-httpd-example} zostały trzy różne zasoby, odpowiednio: paczka oprogramowania~(\texttt{package}), plik~(\texttt{file}) oraz~serwis~(\texttt{service}). Puppet udostępnia również inne rodzaje zasobów takie jak: użytkownik~(\texttt{user}), grupa~(\texttt{group}), zdarzenie cykliczne~(\texttt{cron}), wykonanie komendy~(\texttt{exec}), wpis w~pliku \path{/etc/hosts}~(\texttt{host})\footnote{Lub odpowiedniku pliku \path{/etc/hosts} w~systemach nie unixowych.}. Pełną listę wbudowanych typów zasobów można uzyskać wywołując komendę \texttt{puppet describe --list}. Aby~dowiedzieć się~więcej o~wybranym typie zasobu \texttt{X}, wystarczy wydać polecenie:
\begin{center}
\texttt{puppet describe~X}
\end{center}

Puppet składa się~z~wielu narzędzi towarzyszących. Jednym z~nich jest \texttt{facter}, który na~żądanie zbiera informacje o~systemie i~wyświetla je w~jednym z~możliwych formatów: \emph{JSON}, \emph{YAML} lub~\emph{plaintext}, który jest formatem domyślnym. Wywołanie komendy \texttt{facter} (bez dodatkowych parametrów) wyświetla wszystkie dane o~systemie, jakie zebrało to~narzędzie.

Dla~nowych użytkowników Puppeta została przygotowana przez opiekunów projektu maszyna wirtualna z~samouczkiem, tak, aby użytkownik nie~był skazany na~naukę Puppeta na~własnym systemie~\cite{puppet-vm-tutorial}. W~przypadku błędnej konfiguracji Puppeta, stan maszyny wirtualnej z~samouczkiem może być~łatwo przywrócony, co może nie~być takie proste w~przypadku eksperymentowania na~własnym systemie.

%---------------------------------------

\subsection{Chef}

Chef jest oprogramowaniem napisanym w~językach programowania \texttt{Ruby} (tak jak Puppet) i~\texttt{Erlang} (zastosowanym tylko do~API w~części serwerowej). Projekt został zapoczątkowany przez \href{https://blog.chef.io/author/adam/}{\emph{Adama Jacoba}} na~potrzeby firmy consultingowej, ale~po~zaprezentowaniu go~\href{https://en.wikipedia.org/wiki/Jesse_Robbins}{\emph{Jesse Robinsowi}}, został został opublikowany na~zasadach licencji \glslink{apache2.0-license}{Apache~2.0} i~zaadaptowany do~użycia w~firmie \emph{Amazon}. Pierwsza wersja Chef została opublikowana w~2009~roku.

Chef działa w~architekturze klient-serwer. Serwer może być uruchomiony na~systemach operacyjnych: \emph{CentOS}, \emph{Oracle Linux}, \emph{Red Hat Enterprise Linux} i~\emph{Ubuntu}. Serwer można również próbować uruchomić na~systemach \emph{Mac~OS~X} i~\emph{Microsoft Windows} --- dokumentacja Chef wyszczególnia te~systemy jako nieprzetestowane pod kątem współpracy z~Chef~\cite{chef-supported-platforms}.

Klientami mogą być systemy działające pod kontrolą m.in.: \emph{AIX}, \emph{CentOS}, \emph{FreeBSD}, \emph{Mac~OS~X}, \emph{Oracle Linux}, \emph{Red Hat Enterprise Linux}, \emph{Solaris}, \emph{Ubuntu}, \emph{Microsoft Windows}, \emph{Cisco NX-OS} i~\emph{Cisco IOS XR}~\cite{chef-supported-platforms}. Lista wspieranych systemów operacyjnych dla~maszyn działających w~roli klienta jest większa, ale są~one określone jako \emph{Secondary Platforms} i~\emph{Other Platforms} w~przeciwieństwie do wymienionych, które są~określane jako~\emph{Foundational Platforms}.

Użytkownikami Chefa, tak jak w~przypadku Puppeta, jest wiele dużych firm i~uniwersytetów, takich jak: \emph{Airbnb}, \emph{Disney}, \emph{Mozilla}, \emph{Facebook}, \emph{HP Public Cloud}, \emph{Prezi}, \emph{Uniwersytet Minnesoty}~\cite{chef-customers,puppet-chef-disney}.

Odpowiednikiem plików konfiguracyjnych, które w~Puppet nazywane są~\emph{manifestami}, są~pliki nazywane w~nomenklaturze Chef \emph{przepisami}~(ang.~\emph{recipe}). Przykładowe przepisy przedstawiono w~przykładach~\ref{chef-apache-example}, \ref{chef-user-example} i~\ref{chef-service-example}. Więcej przykładów można znaleźć w~bogatej w~przykładowe przepisy dokumentacji Chef~\cite{chef-examples}.

\begin{lstlisting}[numbers=none,caption={Przepis Chef gwarantujący, że serwer Apache zostanie zainstalowany niezależnie od~tego jak~się~nazywa paczka oprogramowania go~dostarczająca},label=chef-apache-example]
package 'Install Apache' do
  case node[:platform]
  when 'redhat', 'centos'
    package_name 'httpd'
  when 'ubuntu', 'debian'
    package_name 'apache2'
  end
end
\end{lstlisting}

\begin{lstlisting}[numbers=none,caption={Przepis Chef gwarantujący, że użytkownik \texttt{xyz} istnieje w~systemie i~ma zadane ustawienia konta},label=chef-user-example]
user 'xyz' do
  comment 'A random user'
  uid '1234'
  gid '1234'
  home '/home/random'
  shell '/bin/bash'
  password '$1$JJsvHslasdfjVEroftprNn4JHtDi'
end
\end{lstlisting}

\begin{lstlisting}[numbers=none,caption={Przepis Chef gwarantujący, że serwis \texttt{example\_service} jest uruchomiony},label=chef-service-example]
service 'example_service' do
  action :start
end
\end{lstlisting}

%---------------------------------------

\subsection{Ansible}

Ansible jest rozwiązaniem do~zarządzania oprogramowaniem różniącym się od~Puppet i~Chef tym, że~do~konfiguracji klientów \glslink{unix-like-system}{*nixowych} używa~SSH, a~do~klientów Windows WinRM i~PowerShell, dzięki czemu klienci nie~muszą uruchamiać żadnych dodatkowych procesów działających w~tle. Autorzy projektu Ansible podkreślają, że~w~przeciwieństwie do~konkurencji, Ansible nie jest skomplikowany, przez co~jego szybkie opanowanie jest łatwe.

Ansible może zostać uruchomiony na~wielu różnych systemach~\glslink{unix-like-system}{*nixowych}, takich systemach jak~np.~\emph{Red~Hat}, \emph{Debian}, \emph{CentOS}, \emph{OS~X}, \emph{BSD}. System Windows może działać tylko w~roli klienta. Konfiguracja Ansible jest przechowywana w~plikach tekstowych~YAML.

Ansible został opublikowany na~licencji~\glslink{gpl}{GPL} w~wersji~3, a~jego pierwsza wersja została wydana w~2012~roku. Projekt jest napisany w~języku \texttt{Python}~\cite{ansible,ansible-github}.

%---------------------------------------

\subsection{Salt}

Salt, nazywany też SaltStack, to~otwartoźdłowe oprogramowanie napisane w~Pythonie i~opublikowane na~licencji \glslink{apache2.0-license}{Apache~2.0} mające podobną funkcjonalność do~opisanych dotychczas rozwiązań problemu synchronizacji oprogramowania. Pierwsza wersja Salt powstała w~2011~roku, a~więc jest to~najmłodsze rozwiązanie wśród przedstawionych~\cite{saltstack}.

%------------------------------------------------------------------------------

\section{Środowisko projektu}

%------------------------------------------------------------------------------

\section{Zawartość pracy}

\noindent Niniejsza praca magisterska składa się z~dwóch głównych części:
\begin{enumerate}
	\item Niniejszej dokumentacji napisanej w \LaTeX,
	\item Aplikacji, w~tym wersji skompilowanej, kodu źródłowego i~dokumentacji wygenerowanej z~komentarzy w~kodzie.
\end{enumerate}
Aplikacja składa się~również z~dwóch części, co~jest naturalne biorąc pod uwagę przyjętą architekturę klient-serwer~(por.~rozdział~\ref{cel-i-zakres}):
\begin{enumerate}
	\item Serwera --- wzorca oprogramowania i~konfiguracji dla~klientów,
	\item Klienta --- okresowo synchronizującego oprogramowanie i~konfigurację z serwerem.
\end{enumerate}

%------------------------------------------------------------------------------

\section{Źródła wiedzy}

W~czasie prac nad implementacją projektu, korzystano z~wielu pomocnych dokumentacji, materiałów dydaktycznych i~innych źródeł wiedzy takich jak np.~programistyczne dyskusyjne, fora programistyczne, np.~\hreftt{http://stackoverflow.com/}{stackoverflow.com}, \hreftt{http://superuser.com/}{superuser.com}.

\noindent Na~szczególne wyróżnienie zasługują źródła, z~których korzystano najwięcej:
\begin{enumerate}
	\item dokumentacje:
	\begin{enumerate}
		\item dokumentacja TLDP~\cite{tldp},
		\item dokumentacja deweloperów dla wybranych dystrybucji/systemów~\cite{archlinux-wiki,gentoo-wiki},
		\item dokumentacja POSIX~\cite{posix},
		\item dokumentacja wybranego języka programowania i innych użytych narzędzi~\cite{glibc-doc},
		\item (opcjonalnie) dokumenty IETF (\gls{rfc})~\cite{rfc-editor},
		\item dokumenty W3C~\cite{w3c},
		\item \hreftt{http://www.tcpipguide.com/}{tcpipguide.com}, czyli szczegółowy przegląd protokołu TCP/IP i~pokrewnych protokołów, opublikowany również w~formie obszernej\footnote{Książka ma~1616 stron.} książki autorstwa \mbox{Charlesa~M.~Kozieroka} pt.~\textit{The~TCP/IP Guide}\footnote{ISBN:~\href{https://www.nostarch.com/tcpip.htm}{9781593270476}. Wydawnictwo \emph{No Starch Press}.}.
		\item dokumentacja biblioteki OpenSSL~\cite{openssl-doc},
		\item dokumentacja biblioteki libconfig~\cite{libconfig-doc}.
	\end{enumerate}
	\item materiały dydaktyczne do~przedmiotów prowadzonych na~Wydziale~MiNI~PW:
	\begin{enumerate}
		\item ,,TCP/IP'' --- przedmiot prowadzony przez dra~inż.~Marka~Kozłowskiego~\cite{kozlowski},
		\item ,,UNIX'' --- przedmiot prowadzony przez mgra~inż.~Marcina~Borkowskiego~\cite{borkowski}.
	\end{enumerate}
\end{enumerate}
Kompletny spis bibliografii, cytowanej w~niniejszej dokumentacji, znajduje się~na~jej końcu.

%------------------------------------------------------------------------------

\section{Uwagi licencyjne}

Niniejsza praca wpisuje się w~tematykę \glslink{wolne-oprogramowanie}{wolnego oprogramowania} i~dlatego wszystkie narzędzia, biblioteki, programy, kody źródłowe i~grafiki użyte w~czasie implementacji i~dokumentacji niniejszego projektu są~licencjonowane na~warunkach licencji typu~\glslink{copyleft}{copyleft} jeśli nie zaznaczono inaczej.

Lista wykorzystanych bibliotek i~odpowiadających im~licencji, na~prawach której zostały opublikowane:
\begin{itemize}
\item \textbf{libconfig} --- licencja \glslink{lgpl}{\emph{LGPL}}~\cite{libconfig-webpage},
\item \textbf{OpenSSL} --- podwójna licencja \emph{SSLeay License} i~\emph{OpenSSL License}~\cite{openssl-license}.
\end{itemize}

W~związku z~obowiązującym \emph{art.~15a~ustawy z~dnia 4~lutego~1994r. o~prawie autorskim i~prawach pokrewnych}~\cite{papp}, kod źródłowy aplikacji powstałej w~ramach niniejszej pracy, może zostać opublikowany na~warunkach licencji wybranej przez autora pracy po~upływie co~najmniej 6~miesięcy od~momentu obrony autora niniejszej pracy, pod warunkiem, że~Uczelnia, tj.~Politechnika Warszawska, sama nie~opublikuje jej przez ten~czas. Jeśli warunek ten~będzie spełniony, aplikacja zostanie opublikowana w~repozytorium \hreftt{https://github.com/}{github.com} na warunkach zmodyfikowanej licencji \glslink{gpl}{GNU~GPL~3} z~dodanym wyjątkiem dla biblioteki OpenSSL. Bez~dodania wyjątku, podwójna licencja OpenSSL jest niekompatybilna z~rodziną licencji~\glslink{gpl}{GPL}~\cite{openssl-license-incompatibility,openssl-license-incompatibility-2}. Pełny tekst licencji znajduje się w~pliku tekstowym \path{LICENSE}, dołączonym do~projektu.

\newpage

\end{document}
