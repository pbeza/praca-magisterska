\documentclass[thesis]{subfiles}

\providecommand{\keywords}[1]{\vspace*{20pt}\noindent\textbf{\emph{Słowa kluczowe:}} #1}

\begin{document}

\begin{abstract}

Niniejsza praca dotyczy automatyzacji jednakowej lub~podobnej konfiguracji grupy komputerów. Motywacją podjęcia tego tematu jest potrzeba automatyzacji konfiguracji np.~niektórych środowisk serwerów używanych do~obliczeń rozproszonych, serwerowni, biletomatów, bankomatów, stacji roboczych w~biurach, placówkach administracji publicznej, na~uczelniach wyższych, szkołach i~w~innych miejscach, gdzie porządane jest, aby~stanowiska komputerowe miały podobny zestaw zainstalowanego oprogramowania i~konfiguracji. Automatyzacja procesu dystrybucji oprogramowania ma~na~celu ułatwienie pracy administratorów komputerowych, minimalizując lub~zwalniając ich~z~konieczności powtarzania podobnej lub identycznej procedury instalacji i~konfiguracji wielu maszyn, w~konswekwencji zmniejszając ryzyko popełnienia błędów konfiguracji.

Celem pracy jest zaprojektowanie i~zaimplementowanie protokołu umożliwiającego propagowanie zmian w~systemie plików, pakietów oraz~elementów konfiguracji na~stacjach roboczych pracujących pod~kontrolą systemu operacyjnego \glslink{gnulinux}{Linux/GNU} lub~innego systemu \glslink{unix-like-system}{*nix}, a~także elastycznego standardu opisu zmian oraz~narzędzi dla~ich rejestrowania i~dostosowywania. Dodatkowo, na~początku pracy przedstawiono istniejące rozwiązania problemu dystrybuowania oprogramowania i~porównano ich~funkcjonalność do~funkcjonalności stworzonego w~ramach tej pracy narzędzia. Dalsza, główna część pracy została poświęcona opisowi aspektów związanych z~zaprojektowanym, alternatywnym dla~istniejących rozwiązań narzędziem dostosowania oprogramowania komputerów klienckich do~obrazu zmian przygotowanego przez administratora sieci. W~szczególności w~pracy opisano działanie oprogramowania implementującego zaproponowany protokół, aspekty bezpieczeństwa aplikacji, wykorzystane biblioteki, testy, możliwe kierunki dalszego rozwoju stworzonego oprogramowania oraz~wnioski.

\keywords{\emph{\keywordslist}}

\end{abstract}

\end{document}
