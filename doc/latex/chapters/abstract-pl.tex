\documentclass[thesis]{subfiles}

\providecommand{\keywords}[1]{\vspace*{20pt}\noindent\textbf{\emph{Słowa kluczowe:}} #1}

\begin{document}

\begin{abstract}

Niniejsza praca pt.~,,Protokół zarządzania stacjami komputerowymi pod~kontrolą systemu Linux'' dotyczy problemu automatyzacji jednakowej lub~podobnej konfiguracji grupy komputerów. Jej~celem jest zaprojektowanie i~zaimplementowanie alternatywnego, w~zamyśle wygodniejszego od~istniejących, sposobu przygotowania i~propagowania zmian konfiguracji systemu wzorcowego --- w~skrócie nazywanego serwerem --- do~stacji klienckich.% Główną cechą odróżniającą opracowany protokół od~konkurencyjnych rozwiązań jest uproszczona metoda generowania zmian zachodzących na~serwerze, tzn.~na~maszynie zawierającej wzorzec oprogramowania.

Istniejące aplikacje typu SCM (\emph{Software Configuration Manager}) rozwiązujące przedstawiony problem --- takie jak np.~omówione w~pracy \emph{Puppet}, \emph{Chef}, \emph{Ansible} i~\emph{Salt} --- korzystają dedykowanych języków metakonfiguracji (\emph{Domain Specific Language}), w~których można deklaratywnie opisać oczekiwaną konfigurację klientów, tzn.~nie specyfikując konkretnych poleceń do~wykonania, tylko odwołując się do~pożądanego stanu np.~pliku, pakietu lub~użytkownika. Dopiero na~podstawie takiego opisu, będącego warstwą abstrakcji nad~właściwą konfiguracją, klient podejmuje działania mające na~celu dostosowanie się do~wymagań w~nim zapisanych. Stworzona warstwa abstrakcji pozwala na~obsługę różnych dystrybucji i~systemów operacyjnych, jednak powstanie takiej możliwości jest okupione utratą czasu jaki administrator musi poświęcić na~naukę języka wybranego narzędzia SCM oraz~na~wyrażenie w~nim docelowych ustawień kontrolowanych przez siebie komputerów.

Rozwiązanie przedstawione w~ramach tej~pracy nie definiuje własnego, pośredniego języka metakonfiguracji, przez co, z~jednej strony maszyna klienta i~serwera muszą mieć zainstalowane dystrybucje z~co~najmniej identycznym drzewami najważniejszych katalogów (FHS --- \emph{Filesystem Hierarchy Standard}) oraz takimi samymi menadżerami pakietów. Z~drugiej strony, przyjęty niekonwencjonalny pomysł implementacji pozwolił na~stworzenie aplikacji, która ułatwia i~minimalizuje pracę administratora nad~iteracyjnym dostosowywaniem zarządzanych przez siebie systemów, w~konsekwencji zmniejszając ryzyko popełnienia błędów konfiguracji. Sposób działania zaprojektowanego oprogramowania po~stronie serwera składa się z~trzech, wykonywanych na~przemian, kroków --- modyfikowaniu systemu wzorcowego tak jakby był klientem, skanowaniu go w~celu wygenerowania migawki jego stanu oraz automatycznym utworzeniu obrazu zmian konfiguracji stacji wzorcowej na~podstawie migawek, dającym się zastosować na~maszynach klienckich. Przygotowana implementacja pozwala na~wykorzystanie szablonów plików zawierających zmienne (\emph{placeholdery}), które podczas zastosowywania obrazu zmian zostają zamienione na~konkretne wartości --- np.~na~nazwę sieciową komputera lub~wybraną zmienną środowiskową. Wymiana obrazów zmian może zachodzić przez sieć zarówno między serwerem i~klientami, jak i~między samymi klientami.

\keywords{\emph{\keywordslist}}

\end{abstract}

\end{document}
