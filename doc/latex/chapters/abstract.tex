\documentclass[thesis]{subfiles}

\providecommand{\keywords}[1]{\vspace*{20pt}\noindent\textbf{\textit{Słowa kluczowe:}} #1}

\begin{document}

\begin{abstract}

Każdego dnia administratorzy komputerowi na~całym świecie instalują, usuwają i~zmieniają oprogramowanie na~komputerach, za~które odpowiadają. W~celu usprawnienia sobie pracy, do~tego celu używają \gls{ssh}, zdalnego pulpitu lub~innych, podobnych rozwiązań, nierzadko uzależnionych od~używanego systemu operacyjnego. Często istnieje potrzeba, aby~oprogramowanie lub~jego konfiguracja była identyczna lub~zbliżona na~wielu maszynach. W~sposób szczególny potrzeba ta~jest odczuwalna w~środowiskach serwerów używanych do~obliczeń rozproszonych, serwerowniach, biletomatach, bankomatach, biurach, placówkach administracji publicznej, na~uczelniach wyższych i~szkołach, gdzie często pożądane jest, aby~stanowiska komputerowe miały podobny zestaw zainstalowanego oprogramowania i~konfiguracji. W~takim przypadku administrator jest zmuszony do~powtórzenia tej samej procedury instalacji i~konfiguracji dla wielu maszyn. Niniejsza praca magisterska ma na celu rozwiązanie tego problemu. W~pracy przedstawiono szczegółowy opis protokołu sieciowego oraz jego implementacji dla systemów \gls{gnulinux}. Protokół ten umożliwia automatyczną synchronizację oprogramowania instalowanego na~maszynach klienckich, z~serwerem zawierającym wzorzec oprogramowania i~konfiguracji. Ponadto w~pracy przedstawiono przegląd istniejących rozwiązań, aspekty bezpieczeństwa zaproponowanego rozwiązania, napotkane trudności, testy, możliwe kierunki dalszego rozwoju aplikacji oraz wnioski.

\keywords{\href{https://en.wikipedia.org/wiki/Linux}{Linux}, \href{https://en.wikipedia.org/wiki/Communications_protocol}{Communications Protocol}, \href{https://en.wikipedia.org/wiki/Software_configuration_management}{Software Configuration Management~(SCM)}, \href{https://en.wikipedia.org/wiki/Infrastructure_as_Code}{Infrastructure as Code~(IaC)}}
\end{abstract}


\end{document}
