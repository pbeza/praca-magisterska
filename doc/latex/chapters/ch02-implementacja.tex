\documentclass[thesis]{subfiles}

\begin{document}

\chapter{Implementacja}
\label{chapter:implementacja}

Kod źródłowy projektu został napisany w~języku~\texttt{C}. Kod jest zgodny ze standardem~\gls{posix}. W~rozdziale przedstawiono szczegóły implementacyjne projektu.

\section{Formatowanie kodu}

Formatowanie kodu źródłowego jest w~większości zgodne z~wytycznymi zaproponowanymi przez programistów \glslink{kernel}{jądra} systemu \gls{gnulinux}~\cite{kernel-coding-style}. Jednym z~niewielu odstępstw od~tych wytycznych jest rozmiar tabulacji równy 4~znakom, a~nie~8.

%------------------------------------------------------------------------------

\section{Wykorzystane biblioteki}
\subsection{\texttt{OpenSSL}}

Dla~zapewnienia bezpieczeństwa komunikacji między serwerem i~jego klientami, wykorzystano bibliotekę \texttt{OpenSSL}, która udostępnia wiele sprawdzonych algorytmów kryptograficznych mogących zwiększyć bezpieczeństwo komunikacji. Biblioteka ta pozwoliła na stosunkowo łatwe szyfrowanie połączenia między serwerem i klientami, co jednak nie jest wystarczającym zabezpieczeniem. % TODO

\subsection{\texttt{glibc}}

%------------------------------------------------------------------------------

\section{Utworzenie daemona systemowego}

\noindent Utworzenie \emph{\glslink{daemon}{daemona}} systemowego w~systemach \glslink{unix-like-system}{*niksowych} polega na~wykonaniu następujących kroków~\cite{creating-daemon-stackoverflow}:
\begin{itemize}
	\item Utworzenie procesu~\texttt{A}.
	\item Utworzenie procesu potomnego \texttt{B} przez proces \texttt{A}, przez wywołanie funkcji \texttt{fork} przez proces \texttt{A} i~pozwolenie na zakończenie działania procesu~\texttt{A}.
	\item Utworzenie nowej sesji przez wywołanie funkcji \texttt{setsid} przez proces~\texttt{B}. W~ten sposób proces \texttt{B}~zostaje liderem nowej sesji i~jest odłączony od~kontrolującego go~terminala~(\texttt{CTTY}).
	\item Obsłużenie lub zignorowanie sygnałów systemowych.
	\item Utworzenie procesu potomnego \texttt{C}~przez wywołanie funkcji \texttt{fork} przez proces~\texttt{B} i pozwolenie na zakończenie się procesu~\texttt{B}. W~ten sposób pozbywamy się procesów, które były liderami sesji, a~tylko takie mogą pozyskać \texttt{TTY} ponownie.
	\item Zmiana aktualnego katalogu zgodnie z~potrzebami przez wywołanie funkcji \texttt{chdir}.
	\item Zmiana \emph{umaski} zgodnie z~potrzebami przez wywołanie funkcji \texttt{umask}.
	\item Zamknięcie deskryptorów, które zostały odziedziczone po~rodzicach.
\end{itemize}

%------------------------------------------------------------------------------

\section{Procesy i wątki}

Aplikacja jest w~pełni zgodna ze~standardem \gls{posix}, dlatego do~obsługi wątków wykorzystano biblioteki \texttt{pthread}, wchodząca w~skład tego standardu~\cite{pthreads-posix-manual}.

%------------------------------------------------------------------------------

\section{Obsługa sygnałów systemowych}

%------------------------------------------------------------------------------

\section{Zmienne globalne}
\noindent Zmienne globalne zostały wykorzystane tylko tam gdzie to~konieczne, tzn.~w:
\begin{itemize}
	\item \glslink{signal-handler}{signal handlerach}\footnote{Zmienne te~muszą być typu \texttt{volatile sig\_atomic\_t}.},
\end{itemize}

%------------------------------------------------------------------------------

\section{Zmienne \texttt{volatile}}

%------------------------------------------------------------------------------

\section{Semafory, mutexy}

%------------------------------------------------------------------------------

\section{Shared memory segments and message queues}

%------------------------------------------------------------------------------

\section{Wybrane makra}

\begin{lstlisting}[language=c,numbers=none,caption={Makro wykonujące wyrażenie dopóki kod błędu jest równy \texttt{EINTR}}]
TEMP_FAILURE_RETRY(expr)
\end{lstlisting}

%------------------------------------------------------------------------------

\section{Wykorzystane narzędzia}

W~tym rozdziale opisano z~jakich narzędzi i~jakiej konfiguracji korzystano w~trakcie implementacji niniejszej pracy.

\subsection{\texttt{cmake}}

Do kompilacji kodu źródłowego wykorzystano \texttt{CMake}, który jest elastycznym, rozszerzalnym oprogramowaniem do~zarządzania procesem kompilacji i~instalacji oprogramowania na~różnych platformach i~systemach operacyjnych~\cite{cmake}. Jego konfiguracja sprowadza się do~napisania plików konfiguracyjnych o~zwyczajowej nazwie \mbox{\texttt{CMakeLists.txt}}, umieszczonych w~podkatalogach z~kodem źródłowym projektu. Wynikiem uruchomienia \texttt{cmake} na~\glslink{unix-like-system}{systemach *uniksowych} jest klasyczny plik \texttt{Makefile}.

\begin{lstlisting}[language=bash,numbers=none,caption={Uruchomienie \texttt{cmake} w~trybie \texttt{Debug}}]
cmake -D CMAKE_BUILD_TYPE=Debug ..
\end{lstlisting}

\begin{lstlisting}[language=bash,numbers=none,caption={Generowanie symboli dla \texttt{cscope}}]
make cscope
\end{lstlisting}

\begin{lstlisting}[language=bash,numbers=none,caption={Generowanie symboli dla \texttt{ctags}}]
make ctags
\end{lstlisting}

\subsubsection{Generowanie dokumentacji \texttt{Doxygen}}
\subsubsection{Generowanie paczek \texttt{deb} i innych}

\subsection{Edytor tekstu \texttt{vim}}

Do pisania kodu źródłowego projektu oraz niniejszej dokumentacji wykorzystano edytor tekstu~\texttt{vim} w~wersji~7.4, rozbudowany o wtyczki~(\emph{ang.~plugins}).

\subsubsection{Wykorzystane wtyczki \texttt{vim}}

\noindent Lista wtyczek \texttt{vim}, wykorzystanych w czasie implementacji:
\begin{itemize}
	\item\texttt{NerdTree} --- narzędzie do wyświetlania drzewa systemu plików i~nawigowania po nim~\cite{nerdtree-vimorg,nerdtree-github},
	\item\texttt{ctags} --- narzędzie do~generowania metainformacji do~wygodnego nawigowania po kodzie źródłowym~\cite{ctags},
	% http://stackoverflow.com/questions/934233/cscope-or-ctags-why-choose-one-over-the-other
	% http://cscope.sourceforge.net/cscope_maps.vim
	\item\texttt{cscope} --- narzędzie do~wygodnego przeglądania kodu źródłowego, bardziej rozbudowane od \texttt{ctags}~\cite{cscope}.
\end{itemize}

\subsubsection{Konfiguracja \texttt{.vimrc}}

% TODO dodać inne ustawienia z .vimrc

\paragraph{ctags}

W celu włączenia automatycznego ładowania pliku \texttt{tags} przez \texttt{vim}, do~pliku \mbox{\path{~/.vimrc}} dodano następującą konfigurację~\cite{ctags,ctags-tricks}:
\begin{lstlisting}[language=tex,numbers=none,caption={Konfiguracja \texttt{ctags} w~\texttt{.vimrc}}]
set tags=build/tags,tags,./tags;$HOME
\end{lstlisting}
Konfiguracja ta powoduje, że \texttt{vim} automatycznie wczytuje plik \texttt{tags}, szukając go kolejno~w:
\begin{itemize}[font=\ttfamily]
	\item katalogu \path{$PWD/build},
	\item katalogu \path{$PWD},
	\item katalogu, w którym znajduje się aktualnie otwarty plik w \texttt{vim},
	\item katalogach poniżej katalogu \path{$PWD}, ale nie niżej niż \path{$HOME}.
\end{itemize}
gdzie zmienna systemowa \gls{PWD} oznacza aktualny katalog roboczy.

\paragraph{Cscope}

Analogiczną konfigurację ustawiono, aby włączyć automatyczne szukanie plików pomocniczych dla~obsługi \texttt{Cscope} w~edytorze~\texttt{vim}~\cite{cscope,cscope-autoload}. Do~\path{~/.vimrc} dodano:
\begin{lstlisting}[language=tex,numbers=none,caption={Konfiguracja \texttt{Cscope} w~\texttt{.vimrc}}]
function! LoadCscope()
  let db = findfile("cscope.out", "build,.;$HOME")
  if (!empty(db))
    let path = strpart(db, 0, match(db, "/cscope.out$"))
    set nocscopeverbose " suppress 'duplicate connection' error
    exe "cs add " . db . " " . path
    set cscopeverbose
  endif
endfunction
au BufEnter /* call LoadCscope()
\end{lstlisting}

\subsection{\texttt{latexmk}}

\begin{lstlisting}[language=bash,numbers=none,caption={Uruchomienie ciągłej kompilacji \LaTeX do PDF}]
latexmk -pvc -pdf
\end{lstlisting}

\subsection{Menadżer plików \texttt{Midnight Commander}}

Do sprawnego poruszania się po drzewie katalogów projektu wykorzystano menadżer plików \texttt{Midnight~Commander}.

\end{document}
