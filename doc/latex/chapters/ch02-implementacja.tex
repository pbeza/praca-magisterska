\documentclass[thesis]{subfiles}

\begin{document}

\chapter{Implementacja}

Kod źródłowy projektu został napisany wyłącznie w~języku~Python.

W~niniejszym rozdziale przedstawiono szczegóły implementacyjne projektu, w~szczególności zastosowane biblioteki i~narzędzia pomocnicze.

%------------------------------------------------------------------------------

\section{Wykorzystane biblioteki}

W~niniejszym rozdziale przedstawiono biblioteki, które wykorzystano w~implementacji projektu.

\begin{itemize}
	\item \hreftt{http://www.pyopenssl.org/}{pyOpenSSL} --- \hrefemph{https://en.wikipedia.org/wiki/Language_binding}{Binding} dla~biblioteki OpenSSL. Zapewnia poufność komunikacji między klientami i~serwerem oraz~uwierzytelnienia serwera przez klienta.
%Istnieje wiele bibliotek kryptograficznych, które, pod~warunkiem poprawnego użycia, mogą istotnie poprawić bezpieczeństwo protokołu. Najpopularniejsze z~nich~to: \emph{\gls{openssl}}, \emph{GnuTLS}, \emph{LibreSSL} i~\emph{BoringSSL}. Biblioteką istniejącą najdłużej z~wymienionych jest \gls{openssl}. Biblioteka GnuTLS powstała w~odpowiedzi na~\gls{openssl} ze~względu na~to, że~\gls{openssl} nie jest opublikowana na~licencji kompatybilnej z~licencją~\glslink{gpl}{GPL}, przez co~liczne projekty korzystające licencji~\glslink{gpl}{GPL} (np.~\emph{Wireshark}), nie mogą korzystać z~\gls{openssl}. Biblioteki LibreSSL i~BoringSSL są~\emph{fork'ami} biblioteki \gls{openssl}. BoringSSL powstał w~\emph{Google} na~potrzeby użycia w~różnych produktach tej firmy, w~szczególności w~przeglądarce internetowej \emph{Chrome}. Długa historia \gls{openssl}, mnogość bibliotek pokrewnych, mających źródło w~kodzie źródłowym \gls{openssl} oraz~względnie duża społeczność kryptologów zgromadzonych wokół tego projektu, spowodowały, że~biblioteka \gls{openssl} została użyta do~zapewnienia bezpieczeństwa opracowanego protokołu sieciowego.
%\gls{openssl} udostępnia wiele metod i~algorytmów kryptograficznych, które przetrwały wiele lat kryptoanaliz (np.~\gls{aes}, \gls{rsa}) i~są powszechnie uważane za~bezpieczne, pod~warunkiem poprawnego ich~wykorzystania i~zastosowania kluczy o~odpowiednio dużej długości. Wykorzystanie tej biblioteki w~kontekście komunikacji między klientem i~serwerem zostało opisane w~rozdziale~\ref{sec:security}, dotyczącym bezpieczeństwa opracowanego protokołu.
	\item \hreftt{http://pyyaml.org/}{PyYAML} --- Służy do~parsowania plików konfiguracyjnych \texttt{YAML} serwera i~klienta.
\end{itemize}

%------------------------------------------------------------------------------

\section{Wielowątkowość}

TODO

%------------------------------------------------------------------------------

\section{Wykorzystane narzędzia}

W~tym rozdziale opisano z~jakich narzędzi i~z~jakich elementów konfiguracji korzystano w~trakcie implementacji.

%---------------------------------------
%
%\subsection{Generowanie pakietu oprogramowania}
%
%W~celu łatwej instacji powstałego oprogramowania, została przygotowana paczka instalacyjna... TODO

%---------------------------------------
%
%\subsection{Wireshark}
%
%Wireshark jest narzędziem do~analizowania pakietów i~dlatego był jednym z~najczęściej z~używanych narzędzi w~czasie implementacji projektu. Bez Wiresharka podgląd przesyłanych pakietów byłby trudny, szczególnie, że~przesyłane pakiety są~szyfrowane z~wykorzystaniem biblioteki OpenSSL~(patrz rozdział~\ref{sec:security}). Wireshark umożliwia łatwe dekodowanie pakietów po~podaniu klucza prywatnego użytego do~szyfrowania, co~czyni z~niego bardzo wygodne narzędzie do~analizy zaszyfrowanego ruchu sieciowego.

\end{document}
