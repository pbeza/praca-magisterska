\documentclass[praca_magisterska]{subfiles}

\begin{document}

\chapter{Wstęp}
%\label{chapter:intro}

\section{Temat pracy}

Tematem niniejszej pracy jest:
\begin{displayquote}
Protokół zarządzania stacjami komputerowymi pod kontrolą systemu Linux.
\end{displayquote}

\section{Cel pracy}

Ogólny opis celu niniejszej pracy jest następujący:
\blockcquote{formularz-zgloszenia-pracy}{Celem pracy jest zaprojektowanie i zaimplementowanie protokołu umożliwiającego propagowanie zmian w systemie plików, pakietów oraz elementów konfiguracji do stacji roboczych pod kontrolą systemu operacyjnego Linux/GNU lub innego systemu *nix, a także elastycznego standardu opisu zmian oraz narzędzi dla ich rejestrowania i dostosowywania. Proponowaną architekturą jest model klient-serwer. Konkretny obraz zmian może być przeznaczony dla konkretnej dystrybucji, jednakże stworzone rozwiązania praktyczne powinny dawać się w prosty sposób dostosowywać do popularnych dystrybucji systemu Linux/GNU oraz systemów *nix.}
czyli projekt i implementacja systemu synchronizowania oprogramowania zainstalowanego na grupie komputerów pracujących pod kontrolą systemu operacyjnego~\texttt{\glslink{gnulinux}{GNU/Linux}}.

\section{Motywacja}

Motywacją do napisania pracy magisterskiej na ten temat jest problem, z którym zmagają się administratorzy sieci komputerowych na całym świecie, tj. problem dystrybuowania oprogramowania i jego konfiguracji na komputerach połączonych siecią komputerową~\cite{so-problem-intro}.

\section{Istniejące rozwiązania}

Istniejące rozwiązania...

\section{Zastosowania}

Zastosowania...

\section{Cele}

\noindent Autorzy niniejszej pracy postawili przed sobą następujące cele:
\begin{itemize}
	\item \textbf{zaprojektowanie} protokołu,
	\item \textbf{zaimplementowanie} systemu,
	\item \textbf{testy} zaimplementowanego rozwiązania
\end{itemize}

\newpage

\section{Środowisko projektu}

Todo...
\newpage

\end{document}
