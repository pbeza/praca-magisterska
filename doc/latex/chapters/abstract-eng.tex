\documentclass[thesis]{subfiles}

\providecommand{\keywordseng}[1]{\vspace*{20pt}\noindent\textbf{\emph{Keywords:}} #1}

\begin{document}

% Zabronienie dzielenia wyrazów (jęz. angielski)

\begingroup
\hyphenpenalty 10000
\exhyphenpenalty 10000

{\selectlanguage{english}
\begin{abstract}

This paper entitled 'Management Protocol for Linux Workstations' concerns the~automation of~identically or~similarly configured group of~computers. The~motivation behind this topic is~the~need to~automate configuration, for~example, in~some server environments used for~distributed computing, data centers, ticket machines, ATMs, offices, public administration, universities, colleges, schools and other places where it~is desirable for~computer workstations to~have similar set of~installed software and~configuration. Automation of~the software distribution process is~intended to~facilitate the~work of~computer administrators by~minimizing or~exempting them from the need to~repeat a~similar or~identical installation and~configuration procedure for~multiple machines, thereby reducing the~risk of~configuration errors.

The~goal of~this thesis~is to~design and~implement a~protocol that allows to~automatically propagate changes in~the~file~system, packages, and configuration on~clients' workstations running under \glslink{gnulinux}{Linux/GNU} or~other \glslink{unix-like-system}{*nix} operating system, as~well as~flexible standard description of~configuration changes and~tools recording and~customizing such a~changes. In~addition, the~beginning of~the~paper includes a~description of~existing solutions of~the~software distribution problem and~comparison of~the~functionality of~the~existing solutions with the~tool created as~a~part of~this thesis. The~main part of~the~paper has been devoted to~a~description of~aspects of~the~designed alternative tool that allows applying changes prepared by~network administrator as~a~system image. In particular, the paper describes the functionality of the tools implementing the proposed protocol, application's security issues, libraries used, tests, possible directions for further development and conclusions.

\keywordseng{\emph{\keywordslist}}

\end{abstract}}

\endgroup

\end{document}
