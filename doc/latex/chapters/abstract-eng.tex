\documentclass[thesis]{subfiles}

\providecommand{\keywordseng}[1]{\vspace*{20pt}\noindent\textbf{\emph{Keywords:}} #1}

\begin{document}

% Zabronienie dzielenia wyrazów (jęz. angielski)

\begingroup
%\hyphenpenalty 10000
%\exhyphenpenalty 10000

{\selectlanguage{english}
\begin{abstract}

This paper entitled 'Management Protocol for Linux Workstations' concerns the~problem of~automating the~configuration of~a~group of~identically or~similarly configured computers. Its~purpose is to~design and implement an~alternative, more convenient than existing, solution to~prepare and~propagate changes of~the configuration made on~the reference machine --- in~short called the~server --- to~its clients.

Existing SCMs (Software Configuration Managers) that solve the referred problem, such as~those discussed in~this paper --- Puppet, Chef, Ansible and~Salt --- use dedicated meta-configuration languages (Domain Specific Languages) that make it possible to~declaratively describe the~expected configuration of the~clients, i.e.~without specifying specific commands to~execute, but only by~referring to~the~desirable state of~e.g.~file, package or~system user. Using such a~description, that may remind an~abstraction layer over the~actual configuration, client undertakes actions aimed at~adapting its system configuration to the~requirements written in~the~description. The~created abstraction layer makes it~possible to~support various Linux distributions and~different operating systems, however, this can be achieved at~the~expense of~the~time that the~administrator must devote to~learning the~language of the~chosen SCM tool and to~express the~actual settings of~the~computers controlled by~him.

The~solution presented in~this paper doesn't introduce its own meta-configuration language, therefore on one hand the client and server must have installed Linux distributions with at~least identical hierarchy of~top-level directories (FHS --- Filesystem Hierarchy Standard) and~the same package managers. On~the other hand, the adopted unconventional approach allowed for the~implementation of an~application that facilitates and minimizes the~administrator's work on~the~iterative adaptation of the systems he manages, consequently reducing the risk of~making a~mistake in~the~clients' configuration. The~idea behind the~implemented server's application consists of~three sequential steps --- modifying the~reference server system as~if it was a~client's system, scanning it to~generate a~snapshot of~its configuration state, and automatically creating an~applicable configuration snapshot image based on the~snapshots generated so far. The~prepared implementation introduces template files containing placeholders, which are replaced with client-specific values during the~application of~the~snapshot image --- for example placeholders can correspond to~the~client's network name or~its selected environment variable. The~exchange of~the~configuration snapshot images can take place over the~network both between the~server and its clients, as~well as~between the~clients themselves.

\keywordseng{\emph{\keywordslist}}

\end{abstract}}

\endgroup

\end{document}
