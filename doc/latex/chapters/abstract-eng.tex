\documentclass[thesis]{subfiles}

\providecommand{\keywordseng}[1]{\vspace*{20pt}\noindent\textbf{\emph{Keywords:}} #1}

\begin{document}

% Zabronienie dzielenia wyrazów (jęz. angielski)

\begingroup
\hyphenpenalty 10000
\exhyphenpenalty 10000

{\selectlanguage{english}
\begin{abstract}

Every day computer administrators all around the~world install, remove and~modify software on~computers that they are responsible for. To~make their job easier they use~\gls{ssh}, remote desktop and other, similar tools, often dependent on~the~installed operating system. There is~often need to~deploy identical or~similar software or~applications' configuration on~group of~computers. This need is~particularly notable in~distributed computing, data centers, ticket machines, ATMs, offices, public administration, universities, colleges and~schools where computers tend to~have similar set of~installed software and~configuration. In~such a~situation administrator need to~repeat the~same installation and~configuration procedure for~multiple computers.

This thesis aims to~solve this problem. The~paper presents a~detailed description of~the~network protocol and~its implementation for~\gls{gnulinux} operating system. Presented protocol allows automatic synchronization of~software and~configuration installed on~the clients' computers with the~corresponding reference model stored on~the server machine. This work also includes a~description of~existing solutions of~the presented problem, security issues regarding proposed protocol, encountered difficulties, tests, possible directions of~further development and~final conclusions.

\keywordseng{\keywordslist}

\end{abstract}}

\endgroup

\end{document}
