\documentclass[praca_magisterska]{subfiles}

\begin{document}

\chapter{Implementacja}

W tym rozdziale opisano z jakich narzędzi i jakiej konfiguracji korzystano w trakcie implementacji. Wszystkie narzędzia, z których korzystano w ramach ninienszego projektu, są bezpłatne i wydane na jednej z wolnych licencji takich jak \emph{GPL}, \emph{LGPL}, \emph{MIT}, \emph{BSD}, \emph{Apache2.0}.

\section{Edytor tekstu \texttt{vim}}

Do pisania kodu źródłowego projektu wykorzystano edytor tekstu~\texttt{vim} rozbudowany o wtyczki~(\emph{ang.~plugins}).

\subsection{Wykorzystane wtyczki \texttt{vim}}

Lista wtyczek \texttt{vim}, wykorzystanych w czasie implementacji:
\begin{itemize}
	\item\texttt{NerdTree} -- narzędzie do wyświetlania wygodnego drzewa systemu plików,
	\item\texttt{ctags} -- narzędzie do generowania.
\end{itemize}

\subsection{Konfiguracja}

W celu włączenia automatycznego ładowania pliku \texttt{tags} przez \texttt{vim}, do pliku \path{~/.vimrc} dodano następującą konfigurację:
\begin{lstlisting}[language=bash,numbers=none,caption={Konfiguracja \texttt{~/.vimrc}}]
set tags=build/tags,tags,./tags;$HOME
\end{lstlisting}
Konfiguracja ta powoduje, że \texttt{vim} automatycznie wczytuje plik \texttt{tags}, szukając go kolejno~w:
\begin{itemize}[font=\ttfamily]
	\item katalogu \path{$PWD/build},
	\item katalogu \path{$PWD},
	\item katalogu, w którym znajduje się aktualnie otwarty plik w \texttt{vim},
	\item katalogach poniżej katalogu \path{$PWD}, ale nie niżej niż \path{$HOME}.
\end{itemize}
Zmienna systemowa \texttt{\gls{PWD}} oznacza aktualny katalog roboczy.

\section{Menadżer plików \texttt{Midnight Commander}}

Do sprawnego poruszania się po drzewie katalogów projektu wykorzystano menadżer plików \texttt{Midnight~Commander}.

\end{document}
