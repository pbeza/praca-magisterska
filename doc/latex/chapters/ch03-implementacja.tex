\documentclass[thesis]{subfiles}

\begin{document}

\chapter{Implementacja}
\label{ch:implementacja}

W~niniejszym rozdziale przedstawiono szczegóły implementacyjne projektu, w~szczególności opisano zastosowane biblioteki i~narzędzia pomocnicze, sposób użycia aplikacji \hyperref[sec:cli-app]{klienckiej} i~aplikacji \hyperref[sec:srv-app]{serwera}, będącymi odpowiednio --- aplikacją dla~systemu dostosowującego swoją konfigurację do~konfiguracji wzorcowej i~aplikacją do~tworzenia \hyperref[sec:obraz-zmian-konfiguracji]{obrazu zmian konfiguracji} dla~systemu wzorcowego.

%------------------------------------------------------------------------------

\section{Moduły aplikacji}

Stworzona aplikacja jest \href{https://en.wikipedia.org/wiki/Command-line_interface}{aplikacją konsolową} składającą się z~dwóch zasadniczych części --- \hyperref[sec:srv-app]{serwerowej} i~\hyperref[sec:cli-app]{klienckiej} --- nazywanej w~kodzie źródłowym odpowiednio \texttt{myscm-srv} i~\texttt{myscm-cli}\footnote{\texttt{myscm} to~skrót od~roboczej nazwy stworzonego oprogramowania --- \emph{My~\href{https://en.wikipedia.org/wiki/Software_configuration_management}{Software Configuration Manager}}.}. Część serwerowa jest przeznaczona do~uruchomienia na~systemie odgrywającym rolę systemu wzorcowego (,,serwerze''), a~część kliencka na~systemach, których konfiguracja ma~zostać upodobniona do~konfiguracji systemu wzorcowego (,,kliencie''). Każda z~tych części składa się z~modułów realizujących rozłączone zadania.

Wyróżnione moduły aplikacji \hyperref[sec:srv-app]{serwera}:\mynobreakpar
\begin{itemize}
	\item moduł skanera odpowiedzialny za~skanowanie systemu wzorcowego w~poszukiwaniu zmian w~jego konfiguracji (patrz opcja \hyperlinktt{itm:srv-scan}{--scan} aplikacji \texttt{\srvappname}) --- wynikiem działania skanera jest m.in.~plik stanu, który skrótowo opisuje stan konfiguracji systemu wzorcowego; moduł ten~wykorzystuje do~działania skaner~\hyperref[sec:aide]{AIDE} (patrz rozdział~\ref{sec:aide}),
	\item moduł tworzący na~podstawie dwóch wyników skanowań systemu wzorcowego (najnowszego i~wybranego), podpisany cyfrowo \hyperref[sec:obraz-zmian-konfiguracji]{obraz zmian konfiguracji systemu wzorcowego}, będący archiwum plików dla~klienta, które później zostaje rozpakowane i~zastosowane przez aplikację kliencką (patrz opcje \hyperlinktt{itm:srv-gen-img}{--gen-img} aplikacji \texttt{\srvappname{}} i~\hyperlinktt{itm:apply-img}{--apply-img} aplikacji \texttt{\cliappname{}}),
	\item moduł umożliwiający wykorzystanie szablonów konfiguracji, tzn.~plików konfiguracyjnych zawierające zmienne (\emph{placeholdery}) dla~danych charakterystycznych dla~danego klienta i~często znanych tylko jemu --- np.~nazwa sieciowa systemu (\emph{hostname}), nazwa aktualnie zalogowanego użytkownika~itp. (patrz rozdział~\ref{sec:szablony-konfiguracji}),
	\item moduł konfiguracji aplikacji serwera składający się z~trzech plików konfiguracyjnych --- pierwszy z~nich konfiguruje zakres skanowanych (śledzonych) katalogów systemu wzorcowego, drugi zawiera ustawienia serwera, w~tym m.in.: ścieżkę do~pierwszego pliku konfiguracyjnego, ścieżkę do~klucza prywatnego certyfikatu cyfrowego serwera, używanego do~podpisywania \hyperref[sec:obraz-zmian-konfiguracji]{obrazów zmian konfiguracji}, ścieżkę do~katalogu, w~którym są~zapisywane generowane \hyperref[sec:obraz-zmian-konfiguracji]{obrazy zmian konfiguracji}~itp.; trzeci plik konfiguracyjny odpowiada za~\href{https://docs.python.org/dev/library/logging.config.html}{konfigurację logowania} zdarzeń aplikacji.
\end{itemize}

Wyróżnione moduły \hyperref[sec:cli-app]{aplikacji klienckiej}:\mynobreakpar
\begin{itemize}
	\item moduł weryfikujący podpis cyfrowy, rozpakowujący i~stosujący \hyperref[sec:obraz-zmian-konfiguracji]{obraz zmian konfiguracji wzorcowej} udostępniony przez serwer lub~przez innego klienta (patrz opcja \hyperlinktt{itm:apply-img}{--apply-img} aplikacji \texttt{\cliappname}),
	\item moduł pobierający \hyperref[sec:obraz-zmian-konfiguracji]{obraz zmian konfiguracji systemu wzorcowego} przez protokół sieciowy \sftp{}~(\emph{SSH File Transfer Protocol}); pobierany obraz nie musi być pobierany od~serwera --- może być pobrany od~innego klienta (patrz opcje \hyperlinktt{itm:cli-update}{--update} i~\hyperlinktt{itm:cli-upgrade}{--upgrade} aplikacji \texttt{\cliappname}),
	\item moduł konfiguracji aplikacji klienta --- konfiguracja ta~zawiera m.in.~szczegóły połączenia \sftp{}, ścieżkę do~katalogu, w~którym są~zapisywane i~tymczasowo rozpakowywane pobrane \hyperref[sec:obraz-zmian-konfiguracji]{obrazy zmian konfiguracji systemu wzorcowego}~itp.
\end{itemize}

Określenia \emph{klient} i~\emph{serwer} użyte w~kontekście opisu modułów aplikacji stworzonej w~ramach niniejszej pracy nie oznaczają klasycznego modelu komunikacji \href{https://en.wikipedia.org/wiki/Client\%E2\%80\%93server\_model}{klient-serwer}, w~którym serwer musi udostępniać klientom \hyperref[sec:obraz-zmian-konfiguracji]{obraz zmian swojej konfiguracji}. Termin \emph{serwer} w~dokumentacji tego projektu należy raczej utożsamiać z~systemem wzorcowym, którego oprogramowanie pozwala na~tworzenie, przegląd i~dostosowanie \hyperref[sec:obraz-zmian-konfiguracji]{obrazu zmian swojej konfiguracji}, a~nie z~systemem, z~którym klienci muszą się połączyć w~celu pobrania takiego obrazu\footnote{Z~tego powodu być może lepszymi, bardziej neutralnymi określeniami od~przyjętych, umownych nazw \emph{klient} i~\emph{serwer} byłyby np.~określenia \emph{slave} i~\emph{master}.}. Przygotowana implementacja pozwala na~pobieranie \hyperref[sec:obraz-zmian-konfiguracji]{obrazu zmian konfiguracji systemu wzorcowego} (czyli serwera) od~dowolnego systemu, który go~udostępnia za~pomocą \sftp{}, w~szczególności klient może go pobrać również od~\hyperref[sec:srv-app]{serwera}. Taki model komunikacji wydaje się być elastyczny i~skalowalny, ponieważ serwer nie jest ,,wąskim gardłem'' (\hrefemph{https://en.wikipedia.org/wiki/Single_point_of_failure}{Single Point of Failure}) oraz~nie jest tak obciążony jak mógłby być, gdyby tylko on~udostępniał obraz swojej konfiguracji.

W~kolejnych dwóch rozdziałach~\ref{sec:srv-app} i~\ref{sec:cli-app} przedstawiono instrukcję obsługi aplikacji \hyperref[sec:srv-app]{serwera} i~aplikacji \hyperref[sec:cli-app]{klienckiej}. Instrukcja ta~jest również dostępna w~postaci dwóch \emph{manuali} napisanych w~języku angielskimi~dołączonych do~pracy na~płycie~CD (patrz załącznik~\ref{ch:cd-appendix}) --- jednego dla aplikacji serwera \texttt{\srvappname} i~jednego dla~aplikacji klienckiej \texttt{\cliappname}.

W~kolejnych dwóch rozdziałach pominięto następujące wspólne --- w~tym \href{https://www.gnu.org/prep/standards/html_node/Command_002dLine-Interfaces.html}{typowe} dla~wszystkich aplikacji konsolowych --- opcje dla~obu programów:\mynobreakpar

\setlist[description]{style=nextline,font=\ttfamily}
\begin{description}
	\item[--config-check] Sprawdza poprawność konfiguracji aplikacji. Jeśli zawiera błędy, to~kod wyjścia aplikacji (\emph{exit status}) to~1. W~przeciwnym razie to~0.
	\item[--ssl-cert PATH]\hypertarget{itm:ssl-cert} Pełna ścieżka do~cyfrowego certyfikatu cyfrowego SSL \href{https://en.wikipedia.org/wiki/X.509}{X.509} zapisanego w~formacie \pem{}\footnote{Format PEM~(\emph{Privacy-enhanced Electronic Mail}) został sformalizowany przez \gls{ietf} w~2015~roku w~\href{https://tools.ietf.org/html/rfc7468}{RFC7468}.}, służącego do~cyfrowego podpisania i~zweryfikowania autentyczności \hyperref[sec:obraz-zmian-konfiguracji]{obrazów zmian konfiguracji systemu wzorcowego}, wygenerowanych za~pomocą użycia opcji \hyperlinktt{itm:srv-gen-img}{--gen-img} aplikacji \hyperref[sec:srv-app]{serwera} (domyślnie \path{/etc/ssl/private/myscm-srv.cert.pem}).
	\item[--ssl-pubkey PATH] Pełna ścieżka do~pliku zapisanego w~formacie \pem{}, przechowującego klucz publiczny certyfikatu podanego w~opcji \hyperlinktt{itm:ssl-cert}{--ssl-cert} (domyślnie \path{/etc/ssl/private/myscm-srv.cert.pub.pem}).
    \item[--verify {SIGNATURE\_PATH} {SIGNED\_PATH}] Sprawdza czy~zadany podpis cyfrowy \texttt{SIGNATURE\_PATH} pliku \texttt{SIGNED\_PATH} jest zaufany.
	\item[-c, --config FILE]\hypertarget{itm:config} Domyślnie konfiguracja aplikacji serwera jest czytana z~pliku tekstowego \myscmsrvconfig{}. Ta~opcja zmienia to~zachowanie wczytując konfigurację aplikacji serwera z~pliku tekstowego \texttt{FILE}. Konfiguracja ta~określa m.in.~ścieżkę do~certyfikatu cyfrowego serwera, którym są~podpisywane \hyperref[sec:obraz-zmian-konfiguracji]{obrazy zmian konfiguracji systemu wzorcowego}, domyślny stopień ,,gadatliwości'' logowanych zdarzeń (\emph{verbosity level})~itp.
	\item[-h, --help] Wypisuje pomoc w~języku angielskim z~informacjami o~dozwolonych opcjach aplikacji i~ich działaniu.
	\item[-v, --verbose]\hypertarget{itm:verbosity} Włącza tryb ,,gadatliwy", tzn.~włącza wypisywanie dodatkowych komunikatów mogących okazać się pomocnymi przy~diagnozowaniu ewentualnych problemów z~działaniem programu. Domyślnie logi działania programu są~zapisywane w~\path{/var/log/myscm-cli.log} dla~aplikacji \hyperref[sec:cli-app]{klienckiej} i~\path{/var/log/myscm-srv.log} dla~\hyperref[sec:srv-app]{serwera}. Błędy krytyczne i~ostrzeżenia trafiają również do~\hreftt{https://en.wikipedia.org/wiki/Syslog}{sysloga}. \href{https://docs.python.org/dev/library/logging.config.html}{Szczegółowe} ustawienia logowania są~dostępne w~\path{/etc/myscm-cli/log_config.yaml} dla~aplikacji klienta i~\path{/etc/myscm-srv/log_config.yaml} dla~aplikacji serwera.
	\item[--version] Wypisuje informację o~wersji i~licencji aplikacji.
\end{description}

Zarówno aplikacji serwera \texttt{\srvappname} jak i~aplikacji klienckiej \texttt{\cliappname} towarzyszą skrypty i~konfiguracje pomocnicze, które obsługują m.in.: tworzenie i~modyfikowanie testowego drzewa katalogów przedstawionego w~załączniku~\ref{ch:aide-example} (patrz rys.~\ref{fig:drzewo-katalogow-przyklad-aide}), wykorzystywanego do~doraźnych testów w~czasie implementacji projektu (patrz rozdział~\ref{sec:male-testy}), tworzenie \hrefemph{https://en.wikipedia.org/wiki/User_guide}{manuali} napisanych w~języku znacznikowym \hrefemph{https://en.wikipedia.org/wiki/Markdown}{Markdown} i~konwertowanych m.in.~do~\hrefemph{https://en.wikipedia.org/wiki/Groff_(software)}{Groff} za~pomocą \hrefemph{https://en.wikipedia.org/wiki/Pandoc}{Pandoc}, konfigurację tworzenia pakietów \hreftt{https://wiki.debian.org/Packaging}{deb} i~\hreftt{https://wiki.archlinux.org/index.php/creating_packages}{pkg.tar.xz} przeznaczonych odpowiednio dla~dystrybucji \glslink{gnulinux}{Linux} \debian{} i~\href{https://en.wikipedia.org/wiki/Arch_Linux}{Arch}~itp.

%------------------------------------------------------------------------------

\section{Aplikacja serwera}
\label{sec:srv-app}

Aplikacja serwera to~aplikacja przeznaczona do~użycia z~uprawnieniami \superuser{}\footnote{Uprawnienia administratora są~konieczne do~uruchomienia skanowania~\hyperref[sec:aide]{AIDE} również na~katalogach dostępnych tylko dla~administratora.} na~maszynie będącej wzorcem oprogramowania dla~klientów. Służy ona głównie do~tworzenia, przeglądu i~dostosowania \hyperref[sec:obraz-zmian-konfiguracji]{obrazów zmian konfiguracji systemu wzorcowego}, które są~generowane na~podstawie wykrycia zmian jakie zaszły między dwoma wynikami skanowań systemu wzorcowego --- najnowszego i~wybranego --- przeprowadzonymi za~pomocą uruchomienia serwera \texttt{myscm-srv} z~opcją \hyperlinktt{itm:srv-scan}{--scan}~(por.~krok~\hyperlink{itm:stworzenie-obrazu-konfiguracji}{3}, rozdział~\ref{sec:tworzenie-obrazu-konfiguracji}).

Główną funkcjonalnością aplikacji serwera jest stworzenie obrazu zmian konfiguracji, a~nie jego udostępnienie, ponieważ tworzony przez serwer \hyperref[sec:obraz-zmian-konfiguracji]{obraz zmian} może być udostępniany klientom przez innych klientów lub~może być dostarczony klientowi w~dowolny inny sposób --- np.~przez przesłanie go \hrefemph{https://en.wikipedia.org/wiki/Peer-to-peer}{peer-to-peer}, przeniesienie go~na~dysku zewnętrznym lub~przez protokół \sftp{}, który jest obsługiwany przez \hyperref[sec:cli-app]{aplikację kliencką}  (patrz opcje \hyperlinktt{itm:cli-update}{--update} i~\hyperlinktt{itm:cli-upgrade}{--upgrade} aplikacji \texttt{\cliappname}).

Szczegółowa instrukcja obsługi użycia aplikacji serwera została dostarczona z~przygotowanym pakietem oprogramowania \texttt{\srvappname{}}~(patrz załącznik~\ref{ch:cd-appendix}). Instrukcję można otworzyć po~zainstalowaniu pakietu \texttt{\srvappname{}} za~pomocą komendy \texttt{man~\srvappname{}}. Konfiguracja aplikacji serwera znajduje się w~pliku \myscmsrvconfig{}.

\begin{lstlisting}[language=,label=lst:myscm-srv-usage,numbers=none,caption={Szablon uruchomienia aplikacji serwera \texttt{myscm-srv} przeznaczonej do~użycia na~systemie odgrywającym rolę stacji z~konfiguracją wzorcową dla~klientów}]
myscm-srv [OPCJE] AKCJA
\end{lstlisting}

Listing~\ref{lst:myscm-srv-usage} przedstawia skrócony szablon użycia aplikacji serwera, gdzie \texttt{OPCJE} ujęte w~nawiasy kwadratowe \texttt{[~]} oznaczają ich~opcjonalność\footnote{Użycie argumentów opcjonalnych, które mają swoje odpowiedniki w~pliku konfiguracyjnym, powoduje zignorowanie ustawień odpowiedników z~pliku konfiguracyjnego --- dotyczy to~zarówno \hyperref[sec:srv-app]{aplikacji serwera}, jak i~\hyperref[sec:cli-app]{aplikacji klienckiej}.}. Dozwolone \texttt{OPCJE}:\mynobreakpar

\begin{description}
	\item[--aide-config]\hypertarget{itm:aide-config} Konfiguracja serwera składa się z~trzech plików --- pierwszy to~ten określony opcją \hyperlinktt{itm:config}{--config} (domyślnie \myscmsrvconfig{}), drugi jest określony przez opcję \hyperlinktt{itm:aide-config}{--aide-config} (domyślnie \aideconfig{}), a~trzeci to~plik z~\href{https://docs.python.org/dev/library/logging.config.html}{ustawieniami logowania}, do~którego ścieżka jest ustawiona w~pierwszym pliku konfiguracyjnym (domyślnie \myscmsrvlogconfig{}). Opcje \hyperlinktt{itm:config}{--config} i~\hyperlinktt{itm:aide-config}{--aide-config} różnią się tym, że~opcja \hyperlinktt{itm:aide-config}{--aide-config} określa zakres skanowanych katalogów systemu wzorcowego, na~podstawie którego tworzone są~obrazy zmian systemu konfiguracji systemu (jego składnia jest rozszerzoną składnią stosowaną w~pliku konfiguracyjnym znanym z~\hyperref[sec:aide]{AIDE}), a~\hyperlinktt{itm:config}{--config} definiuje pozostałe opcje towarzyszące aplikacji serwera, w~tym również ścieżkę do~konfiguracji, którą można nadpisać opcją \hyperlinktt{itm:aide-config}{--aide-config}.
	\item[--ssl-cert-priv] Pełna ścieżka do~pliku zapisanego w~formacie \pem{}, przechowującego klucz prywatny certyfikatu podanego w~opcji \hyperlinktt{itm:ssl-cert}{--ssl-cert} (domyślnie \path{/etc/ssl/private/myscm-srv.cert.priv.pem}).
	%\item[-d, --daemonize {[TIME\_INTERVAL]}] Uruchamia aplikację w~tle i~włącza cykliczne skanowanie systemu co~pewien czas ustawiony w~konfiguracji serwera lub~co~czas \texttt{TIME\_INTERVAL} wyrażony w~minutach. Opcja ta~może być łączona m.in.~z~opcją \hyperlinktt{itm:srv-share}{--share}. Wszystkie komunikaty programu zostają przekierowane do~\hreftt{https://en.wikipedia.org/wiki/Syslog}{sysloga} lub~do~pliku z~logami, zdefiniowanym w~konfiguracji aplikacji serwera.
\end{description}

Dozwolone wartości zmiennej \texttt{AKCJA} dla~aplikacji \texttt{myscm-srv}~to:\mynobreakpar

\begin{description}
	\item[-s, --scan]\hypertarget{itm:srv-scan} Skanuje system wzorcowy za~pomocą skanera \hyperref[sec:aide]{AIDE} korzystając z~pliku konfiguracyjnego~AIDE wskazywanego przez plik konfiguracyjny aplikacji serwera (domyślnie \aideconfig{}). Wynik skanowania zostaje zapisany w~katalogu \path{/var/lib/myscm-srv/aide.db.current}, a~jeśli taki katalog już istnieje, to~jego nazwa zostaje zmieniona na~\path{/var/lib/myscm-srv/aide.db.X}, gdzie \texttt{X} to~liczba naturalna oznaczająca liczbę skanowań dokonanych do~tej pory, nie licząc aktualnie najnowszego, zapisanego w~\path{aide.db.current} (patrz opcja \hyperlinktt{itm:list-db}{--list-db}). W~utworzonym katalogu zostaje zapisany plik będący wynikiem uruchomienia \hyperref[sec:aide]{AIDE} z~flagą \texttt{--init} oraz~pliki i~katalogi, które zostały wymienione w~konfiguracji \myscmsrvconfig{} jako te, dla~których powinna istnieć możliwość wygenerowania \hrefemph{https://en.wikipedia.org/wiki/Patch_(Unix)}{patchy} zamiast kopiowania ich w~całości do~\hyperref[sec:obraz-zmian-konfiguracji]{obrazu systemu} tworzonego za~pomocą uruchomienia aplikacji \texttt{myscm-srv} z~flagą \hyperlinktt{itm:srv-gen-img}{--gen-img}.
	\item[-g, --gen-img {SYS\_IMG\_VER}]\hypertarget{itm:srv-gen-img} Tworzy \hyperref[sec:obraz-zmian-konfiguracji]{obraz zmian konfiguracji systemu wzorcowego} na~podstawie dwóch katalogów stanów stworzonych w~następstwie wywołania programu serwera z~opcją \hyperlinktt{itm:srv-scan}{--scan} --- najnowszego dostępnego katalogu stanu \path{/var/lib/myscm-srv/aide.db.current} oraz~katalogu \path{/var/lib/myscm-srv/aide.db.X}, gdzie~\texttt{X} to~liczba naturalna \texttt{SYS\_IMG\_VER} określająca deklarowany przez klienta aktualny stan jego konfiguracji. Obraz zmian konfiguracji to~archiwum skompresowane za~pomocą \hreftt{https://en.wikipedia.org/wiki/Tar_(computing)}{tar} i~\hreftt{https://en.wikipedia.org/wiki/Gzip}{gzip} do~pliku pliku z~rozszerzeniem \targz. Domyślnie obraz zmian systemu zostaje zapisany w~pliku \path{/var/lib/myscm-srv/myscm-img.A.B.tar.gz}, gdzie \texttt{A} i~\texttt{B} to~liczby naturalne \texttt{X} w~nazwach katalogów stanów \path{/var/lib/myscm-srv/aide.db.X} odpowiadającym odpowiednio --- aktualnemu stanowi konfiguracji klienta i~stanowi do~którego klient chce aktualizować swoją konfigurację. Jeśli obraz konfiguracji o~tej samej nazwie już istnieje, to~zostaje nadpisany. Jeśli najnowszy plik stanu jest nieaktualny, tzn.~w~systemie wzorcowym zaszły zmiany w~obrębie śledzonych plików po~dokonaniu ostatniego skanowania za~pomocą opcji \hyperlinktt{itm:srv-scan}{--scan}, to~program wykryje taką sytuację i~nie dopuści do~wygenerowania obrazu dopóki nie zostanie uruchomione ponowne skanowanie systemu wzorcowego --- takie sprawdzenie jest istotne, aby do~tworzonego \hyperref[sec:obraz-zmian-konfiguracji]{obrazu} nie zostały skopiowane najnowsze pliki oznaczone starym identyfikatorem stanu systemu.
	\item[--upgrade {SYS\_IMG\_VER}] Wykonuje program tak, jakby był wywołany najpierw z~opcją \hyperlinktt{itm:srv-scan}{--scan}, a~następnie z~opcją \hyperlinktt{itm:srv-gen-img}{--gen-img} z~parametrem \texttt{SYS\_IMG\_VER}.
	\item[--list-db]\hypertarget{itm:list-db} Wyświetla listę wszystkich katalogów utworzonych w~wyniku wywołania programu z~opcją \hyperlinktt{itm:srv-scan}{--scan}. Domyślnie przeszukiwany jest katalog \path{/var/lib/myscm-srv}, w~którym plik \path{aide.db.current} jest najnowszym wynikiem skanowania systemu wzorcowego, a~\path{aide.db.1} najstarszym (tj.~pierwszym).
\item[--list-img]\hypertarget{itm:list-img} Wyświetla listę wszystkich \hyperref[sec:obraz-zmian-konfiguracji]{obrazów zmian konfiguracji systemu wzorcowego} utworzonych w~wyniku wywołania programu z~opcją \hyperlinktt{itm:srv-gen-img}{--gen-img}. Dodatkowo wyświetlana jest informacja które z~obrazów: \begin{enumerate*}[label=(\arabic*)]\item są~podpisane cyfrowo i~są~zaufane \item które nie są podpisane \item które są~podpisane, ale~nie są~uznane za~zaufane\end{enumerate*}.
    \item[--sign {[PATH]}] Tworzy podpis cyfrowy dla~zadanego pliku zdefiniowanego ścieżką \texttt{PATH}. Podpis cyfrowy zostaje zapisany w~tym samym katalogu co~zadany plik, a~jego nazwa jest taka jak nazwa zadanego pliku z~dodanym rozszerzeniem \texttt{.sig} (od~\emph{signature}). Klucz prywatny SSL~użyty do~podpisu jest definiowany w~pliku konfiguracyjnym aplikacji.
\end{description}

%------------------------------------------------------------------------------

\section{Aplikacja kliencka}
\label{sec:cli-app}

Aplikacja kliencka to~aplikacja przeznaczona do~użycia z~uprawnieniami \superuser{}\footnote{Uprawnienia administratora są~konieczne do~tworzenia, zmiany i~usuwania plików z~dowolnej części systemu plików, co~jest często potrzebne do~zastosowania obrazu zmian konfiguracji systemu wzorcowego.} na~maszynie, której konfigurację administrator chce dostosować do~konfiguracji systemu wzorcowego (tj.~serwera). Aplikacja kliencka umożliwia pobranie obrazu zmian konfiguracji np.~od~innego klienta za~pomocą \sftp{}~(\emph{SSH File Transfer Protocol}), a~następnie pozwala na~zastosowanie go na~systemie klienckim.

Szczegółowa instrukcja obsługi użycia aplikacji klienckiej, analogicznie jak w~przypadku aplikacji serwera \texttt{\srvappname}, została dostarczona z~przygotowanym pakietem oprogramowania \texttt{\cliappname{}} (patrz załącznik~\ref{ch:cd-appendix}). Instrukcję można otworzyć po~zainstalowaniu pakietu \texttt{\cliappname{}} za~pomocą komendy \texttt{man~\cliappname{}}. Konfiguracja aplikacji klienckiej znajduje się w~pliku \myscmcliconfig{}.

\begin{lstlisting}[numbers=none,language=,caption={Szablon uruchomienia aplikacji klienckiej \texttt{myscm-cli} dostosowującej konfigurację systemu klienta do~konfiguracji wczytanej z~pliku {\hyperref[sec:obraz-zmian-konfiguracji]{obrazu zmian konfiguracji systemu wzorcowego}}},label=lst:myscm-cli-usage]
myscm-cli [OPCJE] AKCJA
\end{lstlisting}

Listing~\ref{lst:myscm-cli-usage} przedstawia skrócony szablon użycia aplikacji klienckiej, gdzie \texttt{OPCJE} ujęte w~nawiasy kwadratowe \texttt{[~]} oznaczają ich~opcjonalność. Dozwolone \texttt{OPCJE}:\mynobreakpar

\begin{description}
	\item[--force]\hypertarget{itm:client-force} Wymusza stosowanie domyślnych akcji zamiast pytać użytkownika o~podjęcie decyzji w~trakcie działania programu.
	\item[-l, --list] Listuje na~standardowym wyjściu (\stdout) wszystkie lokalnie dostępne \hyperref[sec:obraz-zmian-konfiguracji]{obrazy zmian konfiguracji systemu wzorcowego} utworzone za~pomocą opcji \hyperref[sec:srv-app]{serwera} \hyperlinktt{itm:srv-gen-img}{--gen-img} wraz z~informacją które z~nich mają poprawny podpis cyfrowy serwera. Podpisy cyfrowe muszą być umieszczone w~tym samym katalogu co~odpowiadające im~obrazy, aby~były uwzględnione w~listingu (domyślnie w~katalogu \path{/var/lib/myscm-cli/downloaded}).
	\item[--print-ver]\hypertarget{itm:print-ver} Wypisuje na~standardowym wyjściu (\stdout) liczbę całkowitą oznaczającą ostatnio zastosowaną (za~pomocą opcji \hyperref[sec:cli-app]{klienta} \hyperlinktt{itm:apply-img}{--apply-img}) wersję \hyperref[sec:obraz-zmian-konfiguracji]{obrazu zmian konfiguracji systemu wzorcowego}, czyli aktualnie zastosowaną wersję konfiguracji systemu klienta. Przykładowo, \hyperref[sec:obraz-zmian-konfiguracji]{obraz zmian} zapisany w~pliku \path{myscm-img.1.2.tar.gz}, jest przeznaczony dla~klienta, dla~którego wywołanie programu z~opcją \hyperlinktt{itm:print-ver}{--print-ver} zwraca wartość \texttt{1}, a~po jego zastosowaniu wartość ta~będzie równa \texttt{2}. Jeśli klient nie zastosował jeszcze żadnego \hyperref[sec:obraz-zmian-konfiguracji]{obrazu zmian konfiguracji systemu wzorcowego}, to~zostaje wypisana wartość~\texttt{-1}. W~przeciwnym razie zostaje wypisana liczba naturalna, która jest przechowywana w~pliku tekstowym \path{/var/lib/myscm-cli/img_ver.myscm-cli}.
	\item[-p PROTO, --protocol PROTO]\hypertarget{itm:protocol} Wybiera protokół sieciowy \texttt{PROTO}, który zostanie użyty do~pobrania \hyperref[sec:obraz-zmian-konfiguracji]{obrazu zmian konfiguracji systemu wzorcowego} z~systemu wskazanego przez opcję \hyperlinktt{itm:cli-update}{--update}. Jedyna dozwolona wartość \texttt{PROTO} to~\texttt{\sftp{}}. Szczegóły połączenia, w~tym nazwa użytkownika, hasło i~port połączenia są~ustawiane w~konfiguracji \myscmcliconfig{}. Opcja ta~została wprowadzona po~to, aby~przedstawić jak mogłaby wyglądać obsługa innych protokołów sieciowych w~kolejnych wersjach \hyperref[sec:cli-app]{aplikacji klienckiej}, które mogłyby być rozbudowane o~obsługę \hreftt{https://en.wikipedia.org/wiki/Secure_copy}{SCP}, \hreftt{https://en.wikipedia.org/wiki/File_Transfer_Protocol}{FTP}, \hreftt{https://en.wikipedia.org/wiki/Rsync}{rsync} i~innych, alternatywnych protokołów pozwalających na~przesyłanie plików.
\end{description}

Dozwolone wartości zmiennej \texttt{AKCJA} dla~aplikacji \texttt{myscm-cli}~to:\mynobreakpar

\begin{description}
	\item[--update {[IP\_ADDR]}]\hypertarget{itm:cli-update} Pobiera \hyperref[sec:obraz-zmian-konfiguracji]{obraz zmian konfiguracji systemu wzorcowego} z~maszyny o~adresie IP \texttt{IP\_ADDR} lub,~jeśli adres ten nie został podany, z~komputera, którego adres~IP jest losowany\footnote{Z~jednostajnym rozkładem prawdopodobieństwa.} z~listy wczytanej z~pliku konfiguracyjnego \myscmcliconfig{} przypisanej do~zmiennej \texttt{PeersList}. Pobranie obrazu odbywa się za~pomocą protokołu zdefiniowanego za~pomocą opcji \hyperlinktt{itm:protocol}{--protocol} lub~jeśli opcja ta~nie została podana, za~pomocą protokołu podanego w~pliku konfiguracyjnym \myscmsrvconfig{}. Opcja \hyperlinktt{itm:cli-update}{--update} nie stosuje obrazu zmian ani nie weryfikuje jego podpisu cyfrowego. Pobrany obraz zmian zostaje zapisany do~pliku określonego przez konfigurację aplikacji --- domyślnie do~\path{/var/myscm-cli/myscm-img.A.B.tar.gz}, gdzie \texttt{A} i~\texttt{B}~to~liczby naturalne identyfikujące odpowiednio --- aktualną i~docelową wersję obrazu zmian\footnote{Liczba \texttt{B} jest dobierana automatycznie podczas pobierania obrazu zmian przez sprawdzenie u~klienta, z~którego zostaje pobrany obraz, istnienia plików z~wyższymi identyfikatorami od~aktualnego~\texttt{A} wczytanego z~pliku \path{/var/lib/myscm-cli/img_ver.myscm-cli} aktualizowanego systemu i~wybranie największego z~nich.}.
	\item[--apply-img {SYS\_IMG\_VER}]\hypertarget{itm:apply-img} Zastosowuje obraz konfiguracji systemu załadowany z~pliku \path{/var/lib/myscm-srv/myscm-srv.A.B.tar.gz}, gdzie \texttt{A} to~liczba naturalna określająca aktualny stan oprogramowania klienta (patrz opcja \texttt{--print-ver}), a~\texttt{B} to~zadana liczba naturalna \texttt{SYS\_IMG\_VER} określająca stan oprogramowania, do~którego klient zostanie zaktualizowany. % Jeśli system klienta nie jest w~stanie~\texttt{A}, to~aplikacja kliencka wyświetli ostrzeżenie i~informacje o~różnicach w~oczekiwanej i~podanej konfiguracji systemu. Jeśli podczas zastosowania obrazu konfiguracji systemu pojawią się konflikty, to~aplikacja umożliwia administratorowi porównać konfliktowe pliki i~rozwiązać konflikty za~pomocą ustawionego w~konfiguracji aplikacji narzędzia --- \href{https://developer.atlassian.com/blog/2015/12/tips-tools-to-solve-git-conflicts/}{np.}~za~pomocą \hrefemph{http://kdiff3.sourceforge.net/}{kdiff3}, \hrefemph{http://vimdoc.sourceforge.net/htmldoc/diff.html}{vimdiff}, \hrefemph{http://meldmerge.org/}{meld}~itp.
    \item[--upgrade {[SYS\_IMG\_VER]}]\hypertarget{itm:cli-upgrade} Wykonuje program tak, jakby był wywołany najpierw z~opcją \hyperlinktt{itm:cli-update}{--update}, a~następnie z~opcją \hyperlinktt{itm:apply-img}{--apply-img} z~parametrem \texttt{SYS\_IMG\_VER}. Jeśli opcjonalny parametr \texttt{SYS\_IMG\_VER} nie jest podany, to~stosowany jest \hyperref[sec:obraz-zmian-konfiguracji]{pobrany obraz zmian konfiguacji}. Jeśli żaden obraz zmian nie został pobrany i~parametr \texttt{SYS\_IMG\_VER} nie został podany, to~aplikacja kończy działanie bez zastosowania żadnego \hyperref[sec:obraz-zmian-konfiguracji]{obrazu zmian}.
	\item[--verify-img {PATH}] Sprawdza podpis cyfrowy \hyperref[sec:obraz-zmian-konfiguracji]{obrazu zmian konfiguracji systemu wzorcowego} \texttt{PATH} i~wypisuje komunikat \texttt{SSL signature valid} lub~\texttt{SSL signature invalid}.
\end{description}

%------------------------------------------------------------------------------

\section{Szablony konfiguracji}
\label{sec:szablony-konfiguracji}

Czasami istnieje potrzeba użycia w~plikach konfiguracyjnych programów danych specyficznych dla~systemu klienta. Zdarza się, że~dane te nie są~znane na~stacji wzorcowej, przez co~\hyperref[sec:srv-app]{aplikacja serwera} nie może ich zamieścić w~\hyperref[sec:obraz-zmian-konfiguracji]{obrazie zmian konfiguracji systemu wzorcowego}. W~takich przypadkach rozwiązaniem jest zastosowanie szablonów konfiguracji, które umożliwiają użycie w~plikach konfiguracyjnych zmiennych (\hrefemph{https://en.wikipedia.org/wiki/Placeholder_name}{placeholderów}) na~informacje charakterystyczne, a~czasami unikalne (w~pewnej skali) dla~systemu klienckiego. Przykładami takich zmiennych  mogą być np.:~nazwa sieciowa systemu (\hrefemph{https://en.wikipedia.org/wiki/Hostname}{hostname}), aktualnie zalogowany użytkownik, domyślny interfejs sieciowy, adres~IP, maska sieciowa~itp.

W~zaimplementowanym rozwiązaniu istnieje pewien zbiór obsługiwanych \emph{placeholderów} takich jak np.~\texttt{<HOSTNAME/>} czy~\texttt{<USER/>}, które są~zamieniane na~systemie klienckim na~odpowiadające im~wartości podczas zastosowania \hyperref[sec:obraz-zmian-konfiguracji]{obrazu zmian konfiguracji systemu wzorcowego} za~pomocą opcji \hyperref[sec:cli-app]{klienta} \hyperlinktt{itm:apply-img}{--apply-img} --- w~podanym przypadku odpowiednio na~nazwę sieciową systemu klienckiego oraz~na~nazwę aktualnie zalogowanego użytkownika. Jeśli plik o~pewnej nazwie \texttt{A} powinien być sparametryzowany, to~aby~\hyperref[sec:srv-app]{aplikacja serwera} śledziła szablon konfiguracji pliku \texttt{A}, administrator powinien przygotować kopię tego pliku i~nadać mu~nazwę z~dodanym rozszerzeniem \path{.myscm-template}, czyli w~tym przypadku szablon konfiguracji nosiłby nazwę \path{A.myscm-template}. \hyperref[sec:srv-app]{Aplikacja serwera} podczas tworzenia \hyperref[sec:obraz-zmian-konfiguracji]{obrazu zmian} nie skopiuje do~niego oryginalnego pliku, tylko odpowiadający mu~plik z~szablonem konfiguracji. Warto zwrócić uwagę na to, że~skasowanie pliku \path{A} z~serwera i~zachowanie jedynie jego odpowiednika zawierającego szablon konfiguracji może skończyć się tym, że~program działający na~systemie wzorcowym, który korzystał z~pliku~\path{A}, po~jego skasowaniu przestanie działać, co~w~skrajnym przypadku może uniemożliwić konfigurowanie stacji wzorcowej. Z~tego powodu należy utrzymywać oba te~pliki.

%------------------------------------------------------------------------------

\section{Kod źródłowy}

Kod źródłowy projektu został w~całości napisany w~języku \href{https://en.wikipedia.org/wiki/Python_(programming_language)}{Python} w~najnowszej dostępnej w~trakcie pisania tej pracy wersji~3.6.2. Źródła projektu znajdują się na~płycie~CD dołączonej do~tej pracy (patrz załącznik~\ref{ch:cd-appendix}) w~katalogu~\path{/src}, a~dokumentacja wygenerowana z~komentarzy kodu źródłowego w~\path{/doc/python}.

Kod źródłowy został sformatowany zgodnie z~wymogami \href{https://www.python.org/dev/peps/pep-0008/}{PEP~8} (\emph{Python Enhancement Proposals}) pt.~\emph{Style Guide for Python Code} i~\href{https://www.python.org/dev/peps/pep-0257/}{PEP~257} pt.~\emph{Docstring Conventions}. Do~sprawdzenia zgodności napisanego kodu ze~standardem~PEP~8 wykorzystano narzędzie \hreftt{https://pypi.python.org/pypi/flake8}{flake8}, który obudowuje (\emph{wrapper}) narzędzia \hreftt{https://pypi.python.org/pypi/pep8}{pep8}, \hreftt{https://pypi.python.org/pypi/pyflakes}{pyflakes}, \hreftt{https://pypi.python.org/pypi/pycodestyle}{pycodestyle} i~\hreftt{https://pypi.python.org/pypi/mccabe}{mccabe}. Edytorem kodu wykorzystanym do~implementacji projektu był początkowo \hreftt{http://www.vim.org/}{vim}, który po~pewnym czasie został zastąpiony swoim rozszerzeniem \hreftt{https://neovim.io/}{neovim} --- znanym również pod~nazwą \hreftt{https://neovim.io/}{nvim} --- rozbudowanym \href{http://vimawesome.com/}{wtyczkami}~(\emph{pluginami}).

Aplikacja serwera i~aplikacja kliencka nie są~zależne od~żadnej konkretnej dystrybucji \glslink{gnulinux}{GNU/Linux}, dlatego zasadniczo powinny one~działać niemal na~każdej dystrybucji z~zainstalowanym Pythonem w~wersji~3 i~oprogramowaniem wymienionym w~rozdziale~\ref{sec:wykorzystane-oprogramowanie}. Jak~wspomniano w~rozdziale~\ref{sec:obraz-zmian-konfiguracji}, najbezpieczniej dla~poprawności działania obu aplikacji jest wtedy, gdy~aplikacja kliencka korzysta z~obrazów konfiguracji systemu przygotowanego na~maszynie serwera z~tą samą dystrybucją \glslink{gnulinux}{GNU/Linux}, którą ma~klient\footnote{Wymaganie to~można złagodzić w~niektórych przypadkach pokrewnych dystrybucji z~tym samym schematem drzewa katalogów, pochodzących od~tej~samej dystrybucji, np.~\debian{}.}. Wymaganie to~wynika m.in.~z~tego, że~różne dystrybucje mają czasami różne schematy drzewa katalogów. Jednym z~\href{https://en.wikipedia.org/wiki/Filesystem_Hierarchy_Standard#FHS_compliance}{wielu} przykładów niezgodności są~katalogi \path{/bin} i~\path{/usr/bin}, które w~niektórych dystrybucjach istnieją oba (np.~\debian{}), w~innych też istnieją oba, przy~czym \path{/bin} jest dowiązaniem symbolicznym do~\path{/usr/bin}, a~w~pozostałych istnieje tylko jeden z~tych katalogów. Drugim zastrzeżeniem jest to,~że jeśli w~obrazie wzorcowym są~zawarte skompilowane programy lub~biblioteki, to~serwer przygotowujący obraz konfiguracji dla~klienta musi mieć tę~samą architekturę procesora\footnote{Architektura procesora, czyli w~skrócie ISA (\emph{Instruction Set Architecture}). Przykładowe architektury procesorów to~\href{https://en.wikipedia.org/wiki/List_of_instruction_sets}{np.}~\href{https://en.wikipedia.org/wiki/X86-64}{AMD64}, \href{https://en.wikipedia.org/wiki/X86}{x86}, \href{https://en.wikipedia.org/wiki/ARM_architecture}{ARM}, \href{https://en.wikipedia.org/wiki/IA-64}{IA-64}, \href{https://en.wikipedia.org/wiki/PowerPC}{PowerPC} \href{https://en.wikipedia.org/wiki/Nios_II}{Nios~II}.} co~klient --- programy skompilowane dla~jednej architektury co~do~zasady\footnote{Chyba, że~architektura procesora klienta i~serwera są~\href{https://en.wikipedia.org/wiki/X86-64\#OPMODES}{kompatybilne}.} nie uruchomią się na~innej architekturze.

%------------------------------------------------------------------------------

\section{Wykorzystane biblioteki}
\label{sec:wykorzystane-oprogramowanie}

Do~implementacji projektu wykorzystano kilkadziesiąt \href{https://docs.python.org/dev/tutorial/modules.html}{modułów} Pythona, kilka pomocniczych aplikacji i~bibliotek. Wyróżnione z~nich~to:\mynobreakpar % \hreftt{https://pypi.python.org/pypi/apt/}{apt}, \hreftt{https://pypi.python.org/pypi/python-pacman/}{python-pacman}
\begin{itemize}
	\item \hreftt{http://aide.sourceforge.net/}{AIDE} (\emph{Advanced Intrusion Detection Environment}) --- Skaner integralności plików zaprojektowany w~celu wykrywania złośliwego oprogramowania, np.~\hrefemph{https://en.wikipedia.org/wiki/Rootkit}{rootkitów}. W~niniejszym projekcie został wykorzystany do~wykrywania zmian w~konfiguracji systemu wzorcowego (patrz rozdział~\ref{sec:aide}).
	\item \hreftt{http://www.pyopenssl.org/}{pyOpenSSL} --- \hrefemph{https://en.wikipedia.org/wiki/Language_binding}{Binding} dla~biblioteki OpenSSL. Biblioteka ta~implementuje wiele protokołów i~algorytmów kryptograficznych, ale~w~ramach projektu została wykorzystana tylko w~celu umożliwia uwierzytelnienie serwera przez klienta przez użycie certyfikatów cyfrowych.
	\item \hreftt{https://github.com/GerHobbelt/google-diff-match-patch}{google-diff-match-patch} --- Moduł dla~Pythona służący m.in.~do~tworzenia i~stosowania \hrefemph{https://en.wikipedia.org/wiki/Patch_(Unix)}{patchy}. Został on~wykorzystany do~tworzenia i~stosowania \hrefemph{https://en.wikipedia.org/wiki/Patch_(Unix)}{patchy} dla~konfiguracji dołączanej do~\hyperref[sec:obraz-zmian-konfiguracji]{obrazu zmian konfiguracji systemu wzorcowego}.
\end{itemize}

Inne wybrane wykorzystane moduły Pythona to~m.in.:~\hreftt{https://pypi.python.org/pypi/argcomplete}{argcomplete}, \hreftt{https://docs.python.org/dev/library/argparse.html}{argparse}, \hreftt{https://pypi.python.org/pypi/colorlog}{colorlog}, \hreftt{https://docs.python.org/dev/library/configparser.html}{configparser}, \hreftt{https://pypi.python.org/pypi/distro}{distro}, \hreftt{https://pypi.python.org/pypi/progressbar2}{progressbar2}, \hreftt{https://pypi.python.org/pypi/pysftp}{pysftp}, \hreftt{http://pyyaml.org/wiki/PyYAMLDocumentation}{pyyaml}, \hreftt{https://pypi.python.org/pypi/termcolor}{termcolor}, moduły zależne od~wymienionych i~wiele innych.

%------------------------------------------------------------------------------

\section{AIDE}
\label{sec:aide}

Jednym z~głównych narzędzi wykorzystanych do~implementacji projektu jest~\href{http://aide.sourceforge.net/}{AIDE}~(\emph{Advanced Intrusion Detection Environment}). \href{https://wiki.archlinux.org/index.php/AIDE}{AIDE}~jest oprogramowaniem zaprojektowanym z~myślą wykrywania zmian w~systemie plików w~celu wykrycia złośliwego oprogramowania\footnote{Taki rodzaj oprogramowania jest nazywany oprogramowaniem~\href{https://en.wikipedia.org/wiki/Host-based_intrusion_detection_system_comparison}{HIDS}~(\hrefemph{https://wiki.archlinux.org/index.php/List_of_applications/Security\#Threat_and_vulnerability_detection}{Host-based Intrusion Detection System}).}, jednak sposób jego działania nie ogranicza wykorzystania go~tylko do~tego celu. AIDE~został zapoczątkowany w~1999~roku przez dwóch Finów --- \href{http://www.ipi.fi/~rammer/cv.html}{Ramiego Lehtiego}\footnote{\href{http://www.ipi.fi/~rammer/cv.html}{Będącego} w~latach 1996--1999 administratorem komputerowym w~\href{http://www.tut.fi/en}{Tampere University of~Technology~(TUT)} w~Finlandii.} oraz~\href{https://www.linkedin.com/in/pablo-virolainen-73501731/}{Pablo Virolainena} --- jest w~całości napisany w~języku~\href{https://en.wikipedia.org/wiki/C_(programming_language)}{C} i~jest objęty \glslink{gpl}{licencją~GPL}.

Działanie AIDE polega na~wykonywaniu migawek systemu~(\emph{system snapshot}) zawierających, w~zależności od~konfiguracji, np.~czas modyfikacji śledzonych plików, ilość zajmowanego przez nie miejsca, uprawnienia do~nich i~inne atrybuty plików, zdefiniowane przez administratora w~konfiguracji \href{https://linux.die.net/man/5/aide.conf}{\path{/etc/aide/aide.conf}}. Konfiguracja ta~jest zapisana w~formacie podobnym do~formatu użytego w~komercyjnym oprogramowaniu \href{https://en.wikipedia.org/wiki/Open_Source_Tripwire}{Tripwire}, stworzonym w~1992~roku przez \href{https://en.wikipedia.org/wiki/Gene_Spafford}{prof.~Gene Spafforda} z~\href{https://en.wikipedia.org/wiki/Purdue_University}{Purdue University}~(USA, \href{https://en.wikipedia.org/wiki/Indiana}{Indiana}) i~jego studenta \href{https://en.wikipedia.org/wiki/Gene_Kim}{Gene Kima}. Duże podobieństwo między konfiguracją AIDE i~Tripwire, o~którym wspomina instrukcja obsługi (\emph{\gls{manual}}) pliku konfiguracyjnego \hreftt{https://linux.die.net/man/5/aide.conf}{aide.conf}, jest przesłanką przypuszczenia, że~komercyjny Tripwire mógł być~pierwowzorem dla~\glslink{wolne-oprogramowanie}{otwartoźródłowego} AIDE. Nowe, znacznie rozbudowane wersje oprogramowania Tripwire są~sprzedawane do~dziś w~ramach działalności \href{https://en.wikipedia.org/wiki/Tripwire_(company)}{firmy} o~tej samej nazwie.

W~klasycznym zastosowaniu AIDE związanym z~zapewnieniem bezpieczeństwa systemowi operacyjnemu przez zagwarantowanie jego integralności, pierwsza, referencyjna migawka utworzona komendą \hreftt{https://linux.die.net/man/1/aide}{aide --init}, powinna zostać wykonana na~dopiero~co zainstalowanym systemie operacyjnym przed podłączeniem go~do~sieci i~zapisana na~systemie plików tylko do~odczytu~(\emph{read-only}) lub~na~nośniku niepodłączonym do~działającego systemu (\hrefemph{https://www.techopedia.com/definition/30275/cold-storage}{cold storage})\footnote{Instrukcja użytkownika (\emph{\gls{manual}}) AIDE \href{http://aide.sourceforge.net/stable/manual.html\#usage}{zaleca} przeniesienie na~bezpieczne nośniki również binaria, konfigurację i~instrukcję ~AIDE~\cite{aide-manual}.}~\cite{aide-manual}. Pierwsza migawka jest referencyjnym stanem systemu, do~którego będą porównywane kolejne migawki, a~zapisanie jej~na~partycji \href{https://linux.die.net/man/8/mount}{zamontowanej} w~trybie tylko do~odczytu i~umieszczenie jej~na~bezpiecznym nośniku danych minimalizuje ryzyko jej modyfikacji z~jakiegokolwiek powodu --- np.~z~powodu złośliwego oprogramowania, awarii, przypadkowego nadpisania~itd. Migawki wykonane podczas kolejnych skanowań, wykonane komendą \hreftt{https://linux.die.net/man/1/aide}{aide~--check} i~\hreftt{https://linux.die.net/man/1/aide}{aide~--init}, są~porównywane z~referencyjną migawką i~na~tej podstawie AIDE~tworzy tekstowy raport zmian, który zostaje odczytany i~interpretowany przez oprogramowanie stworzone w~ramach niniejszej pracy, aby~na tej~podstawie wygenerować \hyperref[sec:obraz-zmian-konfiguracji]{obraz zmian konfiguracji systemu wzorcowego}. Pierwszą migawkę referencyjną można zastąpić nową uruchamiając AIDE z~flagą \hreftt{https://linux.die.net/man/1/aide}{--update} i~nadpisując istniejący plik (domyślnie) \path{/var/lib/aide/aide.db} nowym plikiem \path{/var/lib/aide/aide.db.new} --- taka aktualizacja bazy AIDE jest szczególnie użyteczna przy~iteracyjnym dostosowywaniu konfiguracji AIDE \path{/etc/aide/aide.conf} do~własnych potrzeb.

W~drzewie katalogów systemów \glslink{unix-like-system}{*niksowych}\footnote{Struktura drzewa katalogów systemów \glslink{unix-like-system}{*niksowych} jest zdefiniowana przez \hrefemph{https://wiki.linuxfoundation.org/lsb/fhs}{Filesystem Hierarchy Standard}~(\href{http://www.tldp.org/LDP/sag/html/fs-background.html}{FHS}).} istnieją katalogi systemowe, tzn.~katalogi niebędące katalogiem domowym użytkownika ani jego podkatalogiem, których zawartość zmienia się np.~co~każde uruchomienie systemu i~takie zachowanie nie powinno zazwyczaj niepokoić administratora komputerowego, dlatego powinny być one (i~część z~nich domyślnie jest) wyłączone ze~skanowania~AIDE. Do~takich katalogów należą~m.in.~katalogi:\mynobreakpar

\begin{itemize}
	\item \hreftt{http://www.tldp.org/LDP/sag/html/var-fs.html}{/var} --- katalog, w~którym znajdują się \href{http://www.linuxpl.org/SAG/x547.html}{m.in.}~logi w~\path{/var/log}, blokady plików w~\path{/var/lock}, \emph{cache} w~\path{/var/cache}, \href{http://www.tldp.org/HOWTO/Printing-Usage-HOWTO-2.html}{kolejki wydruku} i~kolejki pocztowe w~\path{/var/spool}~itp.,
	\item \hreftt{http://www.tldp.org/LDP/Linux-Filesystem-Hierarchy/html/tmp.html}{/tmp} --- katalog z~danymi tymczasowymi programów.
\end{itemize}

 Do~katalogów, które również powinny być wyłączone ze~skanowania AIDE należą katalogi specjalne, w~tym m.in.~katalogi wirtualne, których istnienie zapewnia w~locie (\emph{on the fly}) \glslink{kernel}{jądro} systemu operacyjnego:\mynobreakpar

\begin{itemize}
	\item \hreftt{https://www.kernel.org/doc/Documentation/filesystems/sysfs.txt}{/sys} --- katalog zawierający m.in.~informacje o~podłączonych urządzeniach,
	\item \hreftt{http://www.tldp.org/LDP/sag/html/proc-fs.html}{/proc} --- katalog \hreftt{https://en.wikipedia.org/wiki/Procfs}{procfs}, zawierający informacje o~działających procesach,
	\item \hreftt{http://www.tldp.org/LDP/sag/html/dev-fs.html}{/dev} --- katalog zawierający \href{https://en.wikipedia.org/wiki/Device_file}{pliki urządzeń}, będące interfejsami sterowników urządzeń,
	\item \hreftt{https://unix.stackexchange.com/questions/13972/what-is-this-new-run-filesystem/13973}{/run} --- katalog zawierający dane dotyczące np.~aktualnie zalogowanych użytkowników i~działających \glslink{daemon}{daemonów} --- często montowany w~systemie plików \hreftt{https://wiki.archlinux.org/index.php/Tmpfs}{tmpfs}.
	\item \hreftt{http://www.tldp.org/LDP/Linux-Filesystem-Hierarchy/html/media.html}{/media} --- miejsce montowania tymczasowo podłączonych nośników danych (\emph{removable media}),
	\item \hreftt{http://www.tldp.org/LDP/Linux-Filesystem-Hierarchy/html/mnt.html}{/mnt} --- miejsce tymczasowego montowania systemów plików.
\end{itemize}

Listing~\ref{lst:aide-config} przedstawia przykładową konfigurację~AIDE. W~\hyperref[line:custom-rule]{pierwszej linii listingu} została zdefiniowana reguła, która określa jakie atrybuty śledzonego pliku lub~katalogu definiują jego stan (poza zawartością). W~przywołanym przykładzie są~to: \href{https://www.linux.com/learn/understanding-linux-file-permissions}{uprawnienia do~pliku}~(\texttt{p}), \href{https://en.wikipedia.org/wiki/Inode}{i-węzeł}~(\texttt{i}), liczba \href{https://en.wikipedia.org/wiki/Hard_link}{dowiązań twardych}~(\texttt{n}), nazwa użytkownika~(\texttt{u}), nazwa grupy~(\texttt{g}), wielkość pliku~(\texttt{s}), liczba zajmowanych \href{https://en.wikipedia.org/wiki/Block_\%28data_storage\%29}{bloków pamięci}\footnote{Rozmiar bloku pamięci np.~partycji \texttt{sda1} można wyświetlić wyrażony w~bajtach za~pomocą komendy \texttt{blockdev --getbsz /dev/sda1}. Domyślna, typowa wartość rozmiaru bloku pamięci waha się w~zależności m.in.~od~zastosowanego systemu plików --- najczęściej wynoszą one~4096, 2048 lub~1024~bajty.}~(\texttt{b}), czas modyfikacji \hreftt{http://leksykot.top.hell.pl/lx3/N/atime_mtime_ctime}{mtime}~(\texttt{m}), czas zmiany \hreftt{http://leksykot.top.hell.pl/lx3/N/atime_mtime_ctime}{ctime}~(\texttt{c})\footnote{\texttt{mtime} to~czas ostatniej zmiany zawartości pliku. \texttt{ctime} to~czas ostatniej zmiany informacji o~pliku (np.~zawartości, uprawnień~itd.). \texttt{atime} to~czas ostatniego dostępu do~pliku.}. Dwa ostatnie atrybuty --- \texttt{md5} i~\texttt{sha1} --- definiują funkcje skrótu jakie zostaną użyte do~obliczenia skrótów~(\emph{hashy}) śledzonych plików, aby~po~wywołaniu AIDE z~flagą \hreftt{https://linux.die.net/man/1/aide}{--check} lub~\hreftt{https://linux.die.net/man/1/aide}{--compare}, AIDE porównując wybrane atrybuty i~skrót pliku, mógł szybko ocenić które pliki i~katalogi zostały zmodyfikowane. Kompletna lista dostępnych atrybutów przedstawia tabela~\ref{tab:aide-file-attrs}. Reguły rozpoczynające się znakiem ukośnika (\texttt{/}) --- tj.~linie~\ref{line:etc-conf}, \ref{line:bin-conf}, \ref{line:sbin-conf} i~\ref{line:var-conf} --- odpowiadają za~dodanie do~skanowania katalogów \path{/etc}, \path{/bin}, \path{/sbin} i~\path{/var} wraz z~zawartością. Reguły~\ref{line:var-log}, \ref{line:spool} i~\ref{line:var-adm-utmp} zaczynają się~wykrzyknikiem~(\texttt{!}), co~oznacza, że~te~katalogi mają nie być skanowane przez~AIDE. Poza regułami występującymi w~przykładzie~\ref{lst:aide-config}, tj.~regułami rozpoczynającymi się znakiem ukośnika~(\texttt{/}) lub~wykrzyknika~(\texttt{!}), istnieje trzeci rodzaj reguły --- rozpoczynający się znakiem równości~(\texttt{=}) --- oznaczający, że~pliki i~katalogi pasujące do~wyrażenia następującego po~znaku równości zostaną dodane do~tworzonej bazy wynikowej --- tak samo jak ma~to~miejsce w~regule rozpoczynającej się od~ukośnika --- ale w~odróżnieniu od~niej, dzieci podkatalogów katalogów pasujących do~reguły nie zostaną dodane, a~dzieci katalogów pasujących do~danej reguły zostaną dodane tylko wtedy, gdy~reguła kończy się~znakiem ukośnika~(\texttt{/}).

\begin{lstlisting}[language=,caption={Zawartość pliku \protect\path{/etc/aide/aide.conf} z~przykładową konfiguracją~AIDE\\ (przykład z~oficjalnej strony internetowej projektu \hreftt{http://aide.sourceforge.net/stable/manual.html}{aide.sourceforge.net/stable/manual.html})},label=lst:aide-config,escapechar=|]
MyRule = p+i+n+u+g+s+b+m+c+md5+sha1|\label{line:custom-rule}|
/etc p+i+u+g     # check only permissions, inode, user and group for etc|\label{line:etc-conf}|
/bin MyRule      # apply the custom rule to the files in bin|\label{line:bin-conf}|
/sbin MyRule     # apply the same custom rule to the files in sbin|\label{line:sbin-conf}|
/var MyRule|\label{line:var-conf}|
!/var/log/.*     # ignore the log dir it changes too often|\label{line:var-log}|
!/var/spool/.*   # ignore spool dirs as they change too often|\label{line:spool}|
!/var/adm/utmp$  # ignore the file /var/adm/utmp|\label{line:var-adm-utmp}|
\end{lstlisting}

W~ogólności reguły AIDE mogą mieć jedną z~trzech postaci:\mynobreakpar
\begin{enumerate}
	\item \texttt{<regex> <file types> <group>}
	\item \texttt{!<regex> <file types>}
	\item \texttt{=<regex> <file types> <group>}
\end{enumerate}
gdzie:\mynobreakpar
\begin{itemize}
	\item \texttt{<regex>} --- wyrażenie regularne rozpoczynające się~ukośnikiem~(\texttt{/}), zgodne wyrażeniami regularnymi występującymi w~języku programowania \href{https://en.wikipedia.org/wiki/Perl}{Perl}~(\hrefemph{https://en.wikipedia.org/wiki/Perl_Compatible_Regular_Expressions}{Perl Compatible Regular Expressions}),
	\item \texttt{<file types>} --- lista będąca filtrem typów plików rozdzielonych przecinkami (może być pusta, wtedy filtrowanie jest wyłączone), do~których zostanie zastosowana dana reguła --- akceptowane typy plików przedstawiono w~tabeli~\ref{tab:aide-file-types},
	\item \texttt{<group>} --- lista atrybutów plików, które definiują stan pliku zapisany zgodnie z~następującą, prostą gramatyką:\mynobreakpar
\begin{lstlisting}[language=,numbers=none,frame=none,xleftmargin=5em,aboveskip=7pt,belowskip=0pt]
<group>| <group> + <predefined group>
       | <group> - <predefined group>
       | <predefined group>
\end{lstlisting}
	gdzie:\mynobreakpar
	\begin{itemize}
		\item \texttt{<predefined group>} to~jeden z~atrybutów z~tabeli~\ref{tab:aide-file-attrs},
		\item sumowanie atrybutów~(\texttt{<group> + <predefined group>}) i~odejmowanie atrybutów (\texttt{<group> - <predefined~group>}) oznacza odpowiednio --- teoriomnogościową sumę i~różnicę atrybutów \texttt{<group>} i~\texttt{<predefined group>}.
	\end{itemize}
\end{itemize}

Konsekwencją zastosowania wyrażeń regularnych w~regułach AIDE jest w~szczególności to, że~zapisanie reguły np.~\path{!/var/adm/utmp} oznacza zignorowanie wszystkich plików, które znajdują się w~katalogu \path{/var/adm} i~których nazwa rozpoczyna się od~napisu \path{utmp} --- np.~plik \path{/var/adm/utmp_root_kit} zostałby zignorowany i~wyłączony z~raportu skanowania~AIDE. Prawidłowa reguła ignorująca tylko plik \path{/var/adm/utmp} to~\path{!/var/adm/utmp$}. Na~początku wyrażenia regularnego dodawany jest \emph{implicite} znak ,,daszka''~(\texttt{\textasciicircum}) oznaczający w~języku wyrażeń regularnych dopasowanie do~początku linii.

\newcommand{\regularfilewiki}{\href{https://en.wikipedia.org/wiki/Unix_file_types\#Regular_file}{zwykłe pliki}}
\newcommand{\directorywiki}{\href{https://en.wikipedia.org/wiki/Unix_file_types\#Directory}{katalogi}}
\newcommand{\symlinkwiki}{\href{https://en.wikipedia.org/wiki/Symbolic_link}{dowiązania symboliczne}}
\newcommand{\chardevwiki}{\href{https://en.wikipedia.org/wiki/Device_file\#Character_devices}{urządzenie znakowe}}
\newcommand{\blockdevwiki}{\href{https://en.wikipedia.org/wiki/Device_file\#Block_devices}{urządzenie blokowe}}
\newcommand{\fifofilewiki}{\href{https://en.wikipedia.org/wiki/Named_pipe}{łącza nazwane}}
\newcommand{\unixsocketwiki}{\href{https://en.wikipedia.org/wiki/Unix_domain_socket}{gniazda UNIX}}
\newcommand{\solariswiki}{\href{https://en.wikipedia.org/wiki/Solaris_(operating_system)}{Solaris}}
\newcommand{\solarisdoorwiki}{\href{https://en.wikipedia.org/wiki/Doors_(computing)}{drzwi}}
\newcommand{\solariseventport}{\href{https://solarisrants.wordpress.com/2013/07/24/solaris-file-event-notification/}{port zdarzenia}}

\begin{table}
	\centering
	\footnotesize
	\begin{tabular}{>{\ttfamily}l|>{\itshape}l|l}
		\textnormal{Typ pliku} & \textnormal{Rozwinięcie skrótu} & Znaczenie atrybutu                     \\\hline\hline
		f                      & regular files                   & \regularfilewiki                       \\\hline
		d                      & directories                     & \directorywiki                         \\\hline
		l                      & symbolic links                  & \symlinkwiki                           \\\hline
		c                      & character devices               & \chardevwiki                           \\\hline
		b                      & block devices                   & \blockdevwiki                          \\\hline
		p                      & FIFO files                      & \fifofilewiki~(pliki FIFO)             \\\hline
		s                      & UNIX sockets                    & \unixsocketwiki                        \\\hline
		D                      & Solaris doors                   & \solarisdoorwiki~systemu \solariswiki  \\\hline
		P                      & Solaris event ports             & \solariseventport~systemu \solariswiki
	\end{tabular}
	\caption[Wszystkie atrybuty plików obsługiwane przez konfigurację AIDE, precyzujące do~jakiego typu pliku mają zastosowanie reguły z~tabeli~\ref{tab:aide-file-attrs}]{Wszystkie atrybuty plików obsługiwane przez konfigurację AIDE, precyzujące do~jakiego typu pliku mają zastosowanie reguły z~tabeli~\ref{tab:aide-file-attrs} (lista z~\emph{\glslink{manual}{manuala}} konfiguracji AIDE --- \hreftt{https://linux.die.net/man/5/aide.conf}{man aide.conf})~\cite{aide-manual}}
	\label{tab:aide-file-types}
\end{table}

Działanie AIDE polega na~tym, że~w~trakcie czytania przez niego swojej \hyperref[lst:aide-config]{konfiguracji}, konstruuje drzewo, którego krawędzie odpowiadają relacji zawierania się skanowanych katalogów i~plików, a~wierzchołki w~przybliżeniu katalogom systemu plików, który będzie przeszukiwany kiedy AIDE zostanie wywołany z~opcją \hreftt{https://linux.die.net/man/1/aide}{--init} lub~\hreftt{https://linux.die.net/man/1/aide}{--scan}~\cite{aide-manual}. W~każdym wierzchołku takiego drzewa znajdują się~trzy listy odpowiadające wszystkim trzem możliwym regułom zaczynającym się od~znaku ukośnika~(\texttt{/}), wykrzyknika~(\texttt{!}) i~znaku równości~(\texttt{=}). Jeśli wierzchołek nie ma~przypisanej którejś reguły, to~lista odpowiadająca danej regule pozostaje pusta. AIDE~umieszcza reguły tak nisko w~drzewie, tzn.~jak najbliżej liści, jak to~tylko możliwe, tak, aby~umieszczane reguły były w~drzewie wyżej niż wszystkie katalogi i~pliki, których dotyczą. Przykładowo reguła \path{!/proc} zostaje umieszczona w~korzeniu drzewa (tzn.~w~wierzchołku~\path{/}), \path{!/proc/.*} w~wierzchołku \path{/proc}, \path{!/var/log/syslog*} w~wierzchołku \path{/var/log}, a~\path{!/home/[a-z0-9]+/.bashrc$} w~wierzchołku \path{/home}. W~momencie gdy~AIDE potrzebuje ustalić czy dany plik o~ścieżce \path{filename} jest włączony do~skanowania, najpierw wyszukuje najgłębiej położony w~drzewie wierzchołek~\path{x}, którego reguły pasują do~pliku \path{filename}, a~następnie wywołuje funkcję rekurencyjną \texttt{check-node\_for\_match(x, filename, true)}\footnote{Funkcję \texttt{check-node\_for\_match(x, filename, true)} można znaleźć w~kodzie źródłowym AIDE w~wersji~\href{https://sourceforge.net/projects/aide/files/aide/0.16/aide-0.16.tar.gz/download}{0.16} --- która jest najnowszą stabilną wersją AIDE w~czasie pisania tej pracy --- w~linii~\href{https://fossies.org/linux/aide/src/gen_list.c\#l_617}{617} w~pliku \href{https://fossies.org/linux/aide/src/gen_list.c}{\path{/src/gen_list.c}}}, której pseudokod przedstawiono w~\href{http://aide.sourceforge.net/stable/manual.html\#config}{instrukcji obsługi}~(\emph{\glslink{manual}{manualu}}) AIDE i~na~listingu~\ref{lst:aide-matching-algorithm}~\cite{aide-manual}. Instrukcja warunkowa \texttt{if} znajdująca się w~\hyperref[line:aide-if-first-time]{drugiej linii listingu} warunkuje wywołanie funkcji \texttt{check} z~linii~\ref{line:aide-check-equals-list}, sprawdzającej czy~plik \texttt{filename} pasuje do~któregokolwiek z~wyrażeń regularnych znajdujących się na~liście odpowiadającej regule rozpoczynającej się od~znaku równości~(\texttt{=}), aby~niepotrzebnie nie sprawdzać spełnienia tej~reguły na~głębszych poziomach rekurencji --- reguła ta~może być tylko spełniona tylko podczas pierwszego wywołania omawianej funkcji, co~wprost wynika z~zasady jej~działania. Linia~\ref{line:aide-check-regular-list} odpowiada za~sprawdzenie czy~plik \texttt{filename} pasuje do~któregoś wyrażenia regularnego z~listy (znajdującej się w~bieżącym wierzchołku drzewa), odpowiadającej regułom rozpoczynającym się od~znaku ukośnika~(\texttt{/}). Linia~\ref{line:aide-recursive-call} to~wywołanie rekurencyjne dla~wierzchołka-rodzica, które zostaje wykonane na~każdym poziomie rekurencji poza ostatnim, kiedy rozpatrywany jest korzeń drzewa (tj.~wierzchołek~\path{/}). W~linii~\ref{line:aide-if-about-to-be-added} następuje sprawdzenie czy~po~wywołaniach rekurencyjnych dla~wierzchołków znajdujących się wyżej w~drzewie, plik \texttt{filename} ma~zostać dodany do~skanowania czy nie --- jeśli tak, to~w~linii~\ref{line:aide-negative-match} zostaje sprawdzona dla~pliku \texttt{filename} reguła rozpoczynająca się od~znaku wykrzyknika~(\texttt{!}). W~linii~\ref{line:aide-return-statement} zostaje zwrócona wartość logiczna \texttt{true} lub~\texttt{false}, która oznacza odpowiednio, że~plik zostanie dodany lub~nie zostanie dodany do~skanowania AIDE.

Oprogramowanie AIDE jest dostępne w~wielu repozytoriach oprogramowania różnych dystrybucji \glslink{unix-like-system}{*niksowych}. Pakiet oprogramowania dostępny w~repozytorium dystrybucji \debian{} tworzy dodatkowe pliki konfiguracyjne i~skrypt opakowujący \hreftt{https://www.apt-browse.org/browse/debian/jessie/main/all/aide-common/0.16~a2.git20130520-3/file/usr/bin/aide.wrapper}{aide.wrapper}, który uniemożliwia uruchomienie więcej niż~jednej instancji procesu AIDE tworząc plik (\hrefemph{https://unix.stackexchange.com/questions/12815/what-are-pid-and-lock-files-for}{lock file}) \path{/var/run/aide.lock}, a~za~pomocą skryptu \hreftt{https://www.apt-browse.org/browse/debian/jessie/main/all/aide-common/0.16~a2.git20130520-3/file/usr/sbin/update-aide.conf}{update-aide.conf} scala do~pliku \path{/var/lib/aide/aide.conf.autogenerated} konfigurację AIDE rozproszoną po~plikach \emph{rule files}, umieszczonych w~katalogu \path{/etc/aide.conf.d} oraz~uruchamia właściwy program \hreftt{https://linux.die.net/man/1/aide}{aide}. Instalacja AIDE na~Debianie domyślnie uruchamia AIDE codziennie za~pomocą \hreftt{https://en.wikipedia.org/wiki/Cron}{crona}. Konfiguracja dotycząca działań \texttt{crona} w~zakresie regularnego uruchomiania AIDE znajduje się w~plikach \path{/etc/default/aide} i~w~\texttt{/etc/cron.daily/aide}~(który z~kolei wywołuje skrypt \texttt{aide.wrapper}). Domyślnie baza stworzona w~wyniku skanowania systemu programem AIDE jest zapisywana w~pliku \path{/var/lib/aide/aide.db.new}. Logi z~działania AIDE są~domyślnie dostępne w~pliku \path{/var/log/aide/aide.log}. Szczegóły instalacyjne specyficzne dla~Debiana można znaleźć w~pliku \path{/usr/share/doc/aide-common/README.Debian.gz} --- w~szczególności można znaleźć tam wskazówki dotyczące inicjalizacji bazy AIDE za~pomocą skryptu \path{/usr/sbin/aideinit} (\emph{wrapper}) zamiast \hreftt{https://linux.die.net/man/1/aide}{aide~--init}.

W~ramach niniejszej pracy AIDE został wykorzystany do~badania zmian w~konfiguracji systemu wzorcowego w~celu wybrania plików, katalogów i~ich~atrybutów, które powinny zostać zmienione na~systemie \hyperref[sec:cli-app]{klienta}, aby~dostosował się on~do~konfiguracji narzuconej przez stację wzorcową, czyli w~przyjętej w~ramach niniejszej pracy nomenklaturze, przez \hyperref[sec:srv-app]{serwer}.

\newcommand{\inodewiki}{\href{https://en.wikipedia.org/wiki/Inode}{i-węzeł}}
\newcommand{\mdwiki}{\href{https://en.wikipedia.org/wiki/MD5}{MD5}}
\newcommand{\shawiki}{\href{https://en.wikipedia.org/wiki/SHA-1}{SHA-1}}
\newcommand{\shatwowiki}{\href{https://en.wikipedia.org/wiki/SHA-2}{SHA-256}}
\newcommand{\shafivewiki}{\href{https://en.wikipedia.org/wiki/SHA-2}{SHA-512}}
\newcommand{\ripemdwiki}{\href{https://en.wikipedia.org/wiki/RIPEMD}{RIPEMD-160}}
\newcommand{\tigerwiki}{\href{https://en.wikipedia.org/wiki/Tiger_(cryptography)}{Tiger}}
\newcommand{\havalwiki}{\href{https://en.wikipedia.org/wiki/HAVAL}{HAVAL}}
\newcommand{\crcwiki}{\href{https://en.wikipedia.org/wiki/Cyclic_redundancy_check}{CRC-32}}
\newcommand{\gostwiki}{\href{https://en.wikipedia.org/wiki/GOST_(hash_function)}{GOST}}
\newcommand{\whirlpoolwiki}{\href{https://en.wikipedia.org/wiki/Whirlpool_(cryptography)}{Whirlpool}}
\newcommand{\aclwiki}{\href{https://en.wikipedia.org/wiki/Access_control_list}{ACL}}
\newcommand{\selinuxwiki}{\href{https://en.wikipedia.org/wiki/Security-Enhanced_Linux}{SELinux}}
\newcommand{\xattrsman}{\href{http://man7.org/linux/man-pages/man7/xattr.7.html}{rozszerzone atrybuty pliku}}
\newcommand{\xattrswiki}{\href{https://en.wikipedia.org/wiki/Extended_file_attributes\#Linux}{xattrs}}
\newcommand{\extwiki}{\href{https://en.wikipedia.org/wiki/Ext2}{\texttt{ext2}}}
\newcommand{\mtime}{\href{http://leksykot.top.hell.pl/lx3/N/atime_mtime_ctime}{\texttt{mtime}}}
\newcommand{\atime}{\href{http://leksykot.top.hell.pl/lx3/N/atime_mtime_ctime}{\texttt{atime}}}
\newcommand{\ctime}{\href{http://leksykot.top.hell.pl/lx3/N/atime_mtime_ctime}{\texttt{ctime}}}

\begin{table}
	\centering
	\footnotesize
	\begin{tabular}{>{\ttfamily}l|>{\itshape}l|l}
		\textnormal{Atrybut pliku} & \textnormal{Rozwinięcie skrótu}        & Znaczenie atrybutu                      \\\hline\hline
		p                          & permissions                            & uprawnienia do~pliku                    \\\hline
		ftype                      & file type                              & typ pliku                               \\\hline
		i                          & inode                                  & \inodewiki                              \\\hline
		l                          & link name                              & nazwa pliku                             \\\hline
		n                          & number of links                        & liczba dowiązań twardych                \\\hline
		u                          & user                                   & właściciel pliku                        \\\hline
		g                          & group                                  & grupa pliku                             \\\hline
		s                          & size                                   & rozmiar                                 \\\hline
		b                          & block count                            & liczba zajmowanych bloków pamięci       \\\hline
		m                          & mtime                                  & czas modyfikacji                        \\\hline
		a                          & atime                                  & czas dostępu                            \\\hline
		c                          & ctime                                  & czas zmiany                             \\\hline
		S                          & check for growing size                 & sprawdzenie czy~plik się~powiększył     \\\hline
		I                          & ignore changed filename                & ignorowanie zmian nazwy pliku           \\\hline
		ANF                        & allow new files                        & ignorowanie w~raporcie nowych plików    \\\hline
		ARF                        & allow removed files                    & ignorowanie w~raporcie usuniętych plików\\\hline
		md5                        & md5 checksum                           & suma kontrolna \mdwiki                  \\\hline
		sha1                       & sha1 checksum                          & suma kontrolna \shawiki                 \\\hline
		sha256                     & sha256 checksum                        & suma kontrolna \shatwowiki              \\\hline
		sha512                     & sha512 checksum                        & suma kontrolna \shafivewiki             \\\hline
		rmd160                     & rmd160 checksum                        & suma kontrolna \ripemdwiki              \\\hline
		tiger                      & tiger checksum                         & suma kontrolna \tigerwiki               \\\hline
		haval                      & haval checksum                         & suma kontrolna \havalwiki               \\\hline
		crc32                      & crc32 checksum                         & suma kontrolna \crcwiki                 \\\hline
		R                          & \texttt{p+ftype+i+l+n+u+g+s+m+c+md5+X} & skrót                                   \\\hline
		L                          & \texttt{p+ftype+i+l+n+u+g+X}           & skrót                                   \\\hline
		E                          & Empty group                            & pusta grupa                             \\\hline
		X                          & \texttt{acl+selinux+xattrs+e2fsattrs}  & skrót dla rozszerzonych atrybutów pliku \\\hline
		>                          & \texttt{p+ftype+l+u+g+i+n+S+X}         & skrót dla~powiększającego się pliku     \\\hline
		% And also the following if you have mhash support enabled                                                    \\\hline
		gost                       & gost checksum                          & suma kontrolna \gostwiki                \\\hline
		whirlpool                  & whirlpool checksum                     & suma kontrolna \whirlpoolwiki           \\\hline
		%The following are available only when explicitly enabled using configure                                     \\\hline
		acl                        & access control list                    & atrybuty pliku \aclwiki                 \\\hline
		selinux                    & selinux attributes                     & atrybuty pliku \selinuxwiki             \\\hline
		xattrs                     & extended attributes                    & \xattrsman (\xattrswiki)                \\\hline
		e2fsattrs                  & file attributes on a second extended file system & atrybuty pliku na~systemie plików \extwiki
	\end{tabular}
	\caption[Wszystkie atrybuty plików dla~reguł obsługiwanych przez konfigurację~AIDE]{Wszystkie atrybuty plików dla~reguł obsługiwanych przez konfigurację~AIDE\\(lista z~\emph{\glslink{manual}{manuala}} konfiguracji AIDE --- \hreftt{https://linux.die.net/man/5/aide.conf}{man aide.conf})~\cite{aide-manual}}
	\label{tab:aide-file-attrs}
\end{table}

\begin{lstlisting}[language=,caption={Pseudokod algorytmu wykorzystywanego przez AIDE do~ustalenia czy~plik lub~katalog \protect\path{filename} powinien być dodany do~skanowania AIDE (pseudokod z~instrukcji AIDE)~\cite{aide-manual}},label=lst:aide-matching-algorithm,escapechar=|]
check_node_for_match(node,filename,first_time)
    if (first_time)|\label{line:aide-if-first-time}|
        check(equals list for this node)|\label{line:aide-check-equals-list}|
    check(regular list for this node)|\label{line:aide-check-regular-list}|
    if (node is not the root node)|\label{line:aide-if-node-not-root}|
        check_node_for_match(nodes parent,filename,false)|\label{line:aide-recursive-call}|
    if (this file is about to be added)|\label{line:aide-if-about-to-be-added}|
        check(negative list for this node)|\label{line:aide-negative-match}|
    return (info about whether this file should be added or not and how)|\label{line:aide-return-statement}|
\end{lstlisting}

%------------------------------------------------------------------------------
%
%\subsection{Generowanie pakietu oprogramowania}
%
%W~celu łatwej instacji powstałego oprogramowania, została przygotowana paczka instalacyjna... TODO

%------------------------------------------------------------------------------
%
%\subsection{Wireshark}
%
%Wireshark jest narzędziem do~analizowania pakietów i~dlatego był jednym z~najczęściej z~używanych narzędzi w~czasie implementacji projektu. Bez Wiresharka podgląd przesyłanych pakietów byłby trudny, szczególnie, że~przesyłane pakiety są~szyfrowane z~wykorzystaniem biblioteki OpenSSL~(patrz rozdział~\ref{sec:security}). Wireshark umożliwia łatwe dekodowanie pakietów po~podaniu klucza prywatnego użytego do~szyfrowania, co~czyni z~niego bardzo wygodne narzędzie do~analizy zaszyfrowanego ruchu sieciowego.

\end{document}
