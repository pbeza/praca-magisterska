\documentclass[thesis]{subfiles}

\begin{document}

\chapter{Implementacja}

W~niniejszym rozdziale przedstawiono szczegóły implementacyjne projektu, w~szczególności zastosowane biblioteki i~narzędzia pomocnicze.

%------------------------------------------------------------------------------

\section{Moduły}

Aplikacja składa się z~części serwerowej i~klienckiej. Każda z~tych części składa się z~modułów realizujących rozłączone zadania.

Główne moduły serwera:\mynobreakpar
\begin{itemize}
	\item moduł skanujący system wzorcowy i~tworzący plik stanu systemu,
	\item moduł udostępniający klientom obraz zmian konfiguracji powstałej na~podstawie skanowania systemu wzorcowego,
	\item moduł konfiguracji serwera z~ustawieniami takimi parametrami jak np.~numer portu nasłuchiwania na~połączenia klientów, czas między kolejnymi skanowaniami systemu~itp.
\end{itemize}

Główne moduły klienta:\mynobreakpar
\begin{itemize}
	\item moduł odbierający, weryfikujący podpis cyfrowy i~rozpakowujący obraz zmian konfiguracji wzorcowej, udostępnionej przez serwer,
	\item parser konfiguracji klienta z~ustawieniami takimi jak np.~numer portu nasłuchiwania serwera, czas między kolejnymi odpytaniami serwera o~aktualizację systemu~itp.
\end{itemize}

%------------------------------------------------------------------------------

\section{Kod źródłowy}

Kod źródłowy projektu został w~całości napisany w~języku \href{https://en.wikipedia.org/wiki/Python_(programming_language)}{Python} w~najnowszej dostępnej wersji~3.6.1. Źródła projektu znajdują się na~płycie~CD dołączonej do~tej pracy (patrz załącznik~\ref{ch:cd-appendix}) w~katalogu~\path{/src}, a~dokumentacja wygenerowana z~komentarzy kodu źródłowego w~\path{/doc/python}.

Kod źródłowy został sformatowany zgodnie z~wymogami \href{https://www.python.org/dev/peps/pep-0008/}{PEP~8} (\emph{Python Enhancement Proposals}) pt.~\emph{Style Guide for Python Code} i~\href{https://www.python.org/dev/peps/pep-0257/}{PEP~257} pt.~\emph{Docstring Conventions}. Do~sprawdzenia zgodności napisanego kodu ze~standardem~PEP~8 wykorzystano narzędzie \hreftt{https://pypi.python.org/pypi/flake8}{flake8}, który obudowuje (\emph{wrapper}) narzędzia \hreftt{https://pypi.python.org/pypi/pep8}{pep8}, \hreftt{https://pypi.python.org/pypi/pyflakes}{pyflakes}, \hreftt{https://pypi.python.org/pypi/pycodestyle}{pycodestyle} i~\hreftt{https://pypi.python.org/pypi/mccabe}{mccabe}. Edytorem kodu wykorzystanym do~implementacji był \texttt{vim} rozbudowany o~\href{http://vimawesome.com/}{wtyczki}~(\emph{plugins}).

%------------------------------------------------------------------------------

\section{Wykorzystane biblioteki}

Do~implementacji projektu wykorzystano wiele \href{https://docs.python.org/dev/tutorial/modules.html}{modułów} Pythona, kilka pomocniczych aplikacji i~bibliotek. Wyróżnione z~nich~to:
\begin{itemize}
	\item \href{http://aide.sourceforge.net/}{AIDE} (\emph{Advanced Intrusion Detection Environment}) --- skaner integralności plików zaprojektowany w~celu wykrywania złośliwego oprogramowania, np.~\hrefemph{https://en.wikipedia.org/wiki/Rootkit}{rootkitów}. W~niniejszym projekcie został wykorzystany do~wykrywania zmian w~konfiguracji systemu wzorcowego (patrz rozdział~\ref{sec:aide}).
	\item \href{http://www.pyopenssl.org/}{pyOpenSSL} --- \hrefemph{https://en.wikipedia.org/wiki/Language_binding}{Binding} dla~biblioteki OpenSSL. Biblioteka ta~implementuje wiele protokołów i~algorytmów kryptograficznych, ale~w~ramach projektu została wykorzystana tylko w~celu umożliwia uwierzytelnienie serwera przez klienta przez użycie certyfikatów cyfrowych.
%	\item \hreftt{http://zeromq.org/}{ØMQ} --- Znana jest również pod~nazwą \emph{ZeroMQ} --- biblioteka implementująca różne wzorce przekazywania komunikatów~(\emph{messaging library}), dedykowana przede wszystkim dla~wydajnych, rozproszonych środowisk sieciowych. Wzorcem wykorzystanym w~implementacji niniejszego projektu jest wzorzec \hrefemph{https://rfc.zeromq.org/spec:29/PUBSUB/}{Publish-Subscribe}, implementowany na~różne sposoby --- jednym ze~sposobów jest wykorzystanie protokołu~\hyperref[subsec:pgm]{PGM}, pośrednio implementowanego przez~\href{https://en.wikipedia.org/wiki/ZeroMQ}{ØMQ} przez wykorzystanie implementacji \href{https://github.com/steve-o/openpgm}{OpenPGM}. Biblioteka została napisana w~języku~C++, ale~powstały również \emph{bindingi} dla~ponad 40~innych języków programowania --- m.in.~dla~Pythona, w~którym został zaimplementowany niniejszy projekt. ØMQ~została opublikowana na~warunkach licencji~\glslink{lgpl}{LGPL}.
%	\item \hreftt{https://en.wikipedia.org/wiki/Squid_(software)}{Squid} lub\footnote{TODO --- wybór tego, który lepiej się sprawdzi --- oba da się skonfigurować do~pracy na~\emph{Debianie} i~\emph{Archu}.} \hreftt{https://www.unix-ag.uni-kl.de/~bloch/acng/html/}{Apt-Cacher-NG} jako serwer buforujący pakiety oprogramowania dla~klienta (\href{https://lwn.net/Articles/318658/}{Package Repository Proxy}).
%Istnieje wiele bibliotek kryptograficznych, które, pod~warunkiem poprawnego użycia, mogą istotnie poprawić bezpieczeństwo protokołu. Najpopularniejsze z~nich~to: \emph{\gls{openssl}}, \emph{GnuTLS}, \emph{LibreSSL} i~\emph{BoringSSL}. Biblioteką istniejącą najdłużej z~wymienionych jest \gls{openssl}. Biblioteka GnuTLS powstała w~odpowiedzi na~\gls{openssl} ze~względu na~to, że~\gls{openssl} nie jest opublikowana na~licencji kompatybilnej z~licencją~\glslink{gpl}{GPL}, przez co~liczne projekty korzystające licencji~\glslink{gpl}{GPL} (np.~\emph{Wireshark}), nie mogą korzystać z~\gls{openssl}. Biblioteki LibreSSL i~BoringSSL są~\emph{fork'ami} biblioteki \gls{openssl}. BoringSSL powstał w~\emph{Google} na~potrzeby użycia w~różnych produktach tej firmy, w~szczególności w~przeglądarce internetowej Chrome. Długa historia \gls{openssl}, mnogość bibliotek pokrewnych, mających źródło w~kodzie źródłowym \gls{openssl} oraz~względnie duża społeczność kryptologów zgromadzonych wokół tego projektu, spowodowały, że~biblioteka \gls{openssl} została użyta do~zapewnienia bezpieczeństwa opracowanego protokołu sieciowego.
%\gls{openssl} udostępnia wiele metod i~algorytmów kryptograficznych, które przetrwały wiele lat kryptoanaliz (np.~\gls{aes}, \gls{rsa}) i~są powszechnie uważane za~bezpieczne, pod~warunkiem poprawnego ich~wykorzystania i~zastosowania kluczy o~odpowiednio dużej długości. Wykorzystanie tej biblioteki w~kontekście komunikacji między klientem i~serwerem zostało opisane w~rozdziale~\ref{sec:security}, dotyczącym bezpieczeństwa opracowanego protokołu.
%	\item \hreftt{http://pyyaml.org/}{PyYAML} --- Służy do~parsowania plików konfiguracyjnych \texttt{YAML} serwera i~klienta.
\end{itemize}

%------------------------------------------------------------------------------

\section{AIDE}
\label{sec:aide}

Kluczowym narzędziem wykorzystanym w~implementacji projektu jest~\href{http://aide.sourceforge.net/}{AIDE}~(\emph{Advanced Intrusion Detection Environment}). AIDE~jest oprogramowaniem zaprojektowanym z~myślą \href{https://en.wikipedia.org/wiki/Host-based_intrusion_detection_system_comparison}{wykrywania} zmian w~systemie plików w~celu wykrycia złoźliwego oprogramowania, jednak sposób jego działania nie ogranicza wykorzystania go~tylko do~tego celu. AIDE~został zapoczątkowany przez \href{http://www.ipi.fi/~rammer/cv.html}{Ramiego Lehtiego}\footnote{\href{http://www.ipi.fi/~rammer/cv.html}{Będącego} w~latach 1996--1999 administratorem komputerowym w~\href{http://www.tut.fi/en}{Tampere University of~Technology~(TUT)} w~Finlandii.} oraz~\href{https://www.linkedin.com/in/pablo-virolainen-73501731/}{Pablo Virolainena} w~1999~roku. AIDE~jest w~całości napisany w~języku~\href{https://en.wikipedia.org/wiki/C_(programming_language)}{C} i~jest objęty \glslink{gpl}{licencją~GPL}.

Działanie AIDE polega na~wykonywaniu migawek systemu~(\emph{snapshot}) zawierającego m.in.~skróty plików~(\emph{files' \href{https://en.wikipedia.org/wiki/Hash_function}{hashes}}), czas ich~modyfikacji, uprawnienia i~inne parametry plików zdefiniowane w~konfiguracji \path{/etc/aide/aide.conf} przez administratora. Pierwsza migawka systemu wykonana komendą \texttt{aide --init}, powinna zostać zapisana na~systemie plików tylko do~odczytu~(\emph{read-only}) lub~na~nośniku niepodłączonym do~działającego systemu (\hrefemph{https://www.techopedia.com/definition/30275/cold-storage}{cold storage}), ponieważ jest ona~referencyjnym stanem systemu, będącym punktem odniesienia dla~kolejnych migawek. Zapisanie pierwszej migawki na~partycji \href{https://linux.die.net/man/8/mount}{zamontowanej} w~trybie tylko do~odczytu i/lub~umieszczenie jej~na~bezpiecznym nośniku minimalizuje ryzyko jej modyfikacji z~jakiegokolwiek powodu --- np.~z~powodu złośliwego oprogramowania, awarii, przypadkowego nadpisania~itd. Migawki wykonane podczas kolejnych skanowań, wykonane komendą \texttt{aide~-C}, są~porównywane z~pierwszą migawką i~na~tej podstawie AIDE~tworzy tekstowy raport zmian.

\begin{lstlisting}[language=,caption={Zawartość pliku \protect\path{/etc/aide/aide.conf} z~przykładową konfiguracją~AIDE (ze~strony~\hreftt{http://aide.sourceforge.net/stable/manual.html}{aide.sourceforge.net/stable/manual.html})},label=lst:aide-config,escapechar=|]
MyRule = p+i+n+u+g+s+b+m+c+md5+sha1|\label{line:custom-rule}|
/etc p+i+u+g     # check only permissions, inode, user and group for etc
/bin MyRule      # apply the custom rule to the files in bin
/sbin MyRule     # apply the same custom rule to the files in sbin
/var MyRule
!/var/log/.*     # ignore the log dir it changes too often|\label{line:var-log}|
!/var/spool/.*   # ignore spool dirs as they change too often|\label{line:spool}|
!/var/adm/utmp$  # ignore the file /var/adm/utmp|\label{line:var-adm-utmp}|
\end{lstlisting}

W~drzewie katalogów systemów \glslink{unix-like-system}{*niksowych}\footnote{Struktura drzewa katalogów systemów \glslink{unix-like-system}{*niksowych} jest zdefiniowana przez \hrefemph{https://wiki.linuxfoundation.org/lsb/fhs}{Filesystem Hierarchy Standard}~(\href{http://www.tldp.org/LDP/sag/html/fs-background.html}{FHS}).} istnieją katalogi systemowe, tzn.~katalogi niebędące katalogiem domowym użytkownika ani jego podkatalogiem, których zawartość zmienia się np.~co~każde uruchomienie systemu i~takie zachowanie nie powinno zazwyczaj niepokoić administratora, dlatego powinny być one (i~są domyślnie) wyłączone ze~skanowania~AIDE. Do~takich katalogów należą~m.in.~katalogi:

\begin{itemize}
	\item \hreftt{http://www.tldp.org/LDP/sag/html/var-fs.html}{/var} --- katalog, w~którym znajdują się \href{http://www.linuxpl.org/SAG/x547.html}{m.in.}~logi w~\path{/var/log}, blokady plików w~\path{/var/lock}, \emph{cache} w~\path{/var/cache}, \href{http://www.tldp.org/HOWTO/Printing-Usage-HOWTO-2.html}{kolejki wydruku} i~kolejki pocztowe w~\path{/var/spool}~itp.,
	\item \hreftt{http://www.tldp.org/LDP/Linux-Filesystem-Hierarchy/html/tmp.html}{/tmp} --- katalog z~danymi tymczasowymi.
\end{itemize}

 Do~katalogów, które również powinny być wyłączone ze~skanowania AIDE należą katalogi specjalne, np.~katalogi wirtualne, których istnienie zapewnia w~locie (\emph{on the fly}) \glslink{kernel}{jądro} systemu operacyjnego:

\begin{itemize}
	\item \hreftt{https://www.kernel.org/doc/Documentation/filesystems/sysfs.txt}{/sys} --- katalog zawierający m.in.~informacje o~urządzeniach podłączonych do~komputera,
	\item \hreftt{http://www.tldp.org/LDP/sag/html/proc-fs.html}{/proc} --- katalog \hreftt{https://en.wikipedia.org/wiki/Procfs}{procfs}, zawierający informacje o~działających procesach,
	\item \hreftt{http://www.tldp.org/LDP/sag/html/dev-fs.html}{/dev} --- katalog zawierający \href{https://en.wikipedia.org/wiki/Device_file}{pliki urządzeń},
	\item \hreftt{https://unix.stackexchange.com/questions/13972/what-is-this-new-run-filesystem/13973}{/run} --- katalog zawierający dane dotyczące np.~aktualnie zalogowanych użytkowników i~działających \glslink{demon}{demonów}; często montowany w~systemie plików \hreftt{https://wiki.archlinux.org/index.php/Tmpfs}{tmpfs}.
	\item \hreftt{http://www.tldp.org/LDP/Linux-Filesystem-Hierarchy/html/media.html}{/media} --- miejsce montowania np.~tymczasowo podłączonych nośników danych (\emph{removable media}),
	\item \hreftt{http://www.tldp.org/LDP/Linux-Filesystem-Hierarchy/html/mnt.html}{/mnt} --- miejsce tymczasowego montowania systemów plików.
\end{itemize}

W~ramach niniejszej pracy AIDE został wykorzystany do~badania zmian w~konfiguracji systemu wzorcowego. Listing~\ref{lst:aide-config} przedstawia przykładową konfigurację~AIDE. W~\hyperref[line:custom-rule]{pierwszej linii} została zdefiniowana reguła, która określa jakie parametry pliku definiują jego stan. W~tym przykładzie są~to \href{https://www.linux.com/learn/understanding-linux-file-permissions}{uprawnienia do~pliku}~(\texttt{p}), \href{https://en.wikipedia.org/wiki/Inode}{i-węzeł}~(\texttt{i}), liczba \href{https://en.wikipedia.org/wiki/Hard_link}{dowiązań twardych}~(\texttt{n}), nazwa użytkownika~(\texttt{u}), nazwa grupy~(\texttt{g}), wielkość pliku~(\texttt{s}), liczba zajmowanych bloków pamięci~(\texttt{b}), czas modyfikacji \hreftt{http://leksykot.top.hell.pl/lx3/N/atime_mtime_ctime}{mtime}~(\texttt{m}), czas zmiany \hreftt{http://leksykot.top.hell.pl/lx3/N/atime_mtime_ctime}{ctime}~(\texttt{c})\footnote{\texttt{mtime} to~czas ostatniej zmiany zawartości pliku. \texttt{ctime} to~czas ostatniej zmiany informacji o~pliku (np.~zawartości, uprawnień~itd.). \texttt{atime} to~czas ostatniego dostępu do~pliku. $\texttt{mtime} \leq \texttt{ctime} \leq \texttt{atime}$.}. Dwa ostatnie parametry --- \texttt{md5} i~\texttt{sha1} --- definiują funkcje skrótu jakie zostaną użyte do~obliczenia \emph{hashy} plików znajdujących się w~katalogach z~zastosowaną tą~regułą (w~przykładzie są~to~katalogi \path{/bin}, \path{/sbin} i~\path{/var}). Linie~\ref{line:var-log}, \ref{line:spool} i~\ref{line:var-adm-utmp} zaczynają się znakiem wykrzyknika, co~oznacza, że~te~katalogi mają nie być skanowane przez~AIDE.

AIDE jest dostępny w~wielu repozytoriach oprogramowania różnych dystrybucji \glslink{unix-like-system}{*niksowych}. Pakiet oprogramowania dostępny w~repozytorium Debiana tworzy dodatkowe pliki konfiguracyjne i~skrypt opakowujący \hreftt{https://www.apt-browse.org/browse/debian/jessie/main/all/aide-common/0.16~a2.git20130520-3/file/usr/bin/aide.wrapper}{aide.wrapper}, który uniemożliwia uruchomienie więcej niż~jednej instancji AIDE, scala konfigurację AIDE rozproszoną po~plikach \emph{rule files}, umieszczonych w~katalogu \path{/etc/aide.conf.d} za~pomocą skryptu \hreftt{https://www.apt-browse.org/browse/debian/jessie/main/all/aide-common/0.16~a2.git20130520-3/file/usr/sbin/update-aide.conf}{update-aide.conf} oraz~uruchamia właściwy program \hreftt{https://linux.die.net/man/1/aide}{aide}. Instalacja AIDE na~Debianie domyślnie uruchamia AIDE codziennie za~pomocą \hreftt{https://en.wikipedia.org/wiki/Cron}{crona}. Konfiguracja dotycząca działań \texttt{crona} i~AIDE znajduje się w~plikach \path{/etc/default/aide} i~w~\texttt{/etc/cron.daily/aide} (który wywołuje \texttt{aide.wrapper}). Domyślnie baza stworzona ze~skanowania systemu jest tworzona w~katalogu \path{/var/lib/aide} w~pliku \path{aide.conf.autogenerated} i~\path{aide.db.new}. Logi z~działania AIDE są~domyślnie dostępne w~pliku \path{/var/log/aide/aide.log}. Szczegóły instalacyjne specyficzne dla~Debiana można znaleźć w~pliku \path{/usr/share/doc/aide-common/README.Debian.gz} --- w~szczególności można znaleźć tam wskazówki dotyczące inicjalizacji bazy AIDE za~pomocą skryptu \path{/usr/sbin/aideinit} (\emph{wrapper}) zamiast \texttt{aide~--init}.

%------------------------------------------------------------------------------
%
%\subsection{Generowanie pakietu oprogramowania}
%
%W~celu łatwej instacji powstałego oprogramowania, została przygotowana paczka instalacyjna... TODO

%------------------------------------------------------------------------------
%
%\subsection{Wireshark}
%
%Wireshark jest narzędziem do~analizowania pakietów i~dlatego był jednym z~najczęściej z~używanych narzędzi w~czasie implementacji projektu. Bez Wiresharka podgląd przesyłanych pakietów byłby trudny, szczególnie, że~przesyłane pakiety są~szyfrowane z~wykorzystaniem biblioteki OpenSSL~(patrz rozdział~\ref{sec:security}). Wireshark umożliwia łatwe dekodowanie pakietów po~podaniu klucza prywatnego użytego do~szyfrowania, co~czyni z~niego bardzo wygodne narzędzie do~analizy zaszyfrowanego ruchu sieciowego.

\end{document}
