\documentclass[thesis]{subfiles}

\begin{document}

% Załączniki -- może lepszy: \usepackage[toc,page]{appendix} ?
% Patrz: http://tex.stackexchange.com/questions/49643/making-appendix-for-thesis

\begin{appendices}

\chapter{Płyta CD}
\label{cd-appendix}

Do~pracy dołączono płytę z~zawartością części aplikacyjnej niniejszej pracy magisterskiej. Drzewo najważniejszych\footnote{Zaprezentowane drzewo nie prezentuje całej zawartości płyty~CD, a~jedynie najważniejsze katalogi i~pliki do~głębokości drzewa równej~4. Drzewo powstało przez wybranie najważniejszych plików z~wyjścia wywołania komendy \mbox{\texttt{tree~-L~4}}~\cite{tree-manual}.} katalogów i~plików zapisanych na~płycie~CD: % TODO
\dirtree{%
.1 bin.
.1 data.
.1 doc.
.2 dziekanat.
.2 jak-pisac.
.3 wzor-pracy-mgr-mimuw.
.2 latex.
.3 chapters.
.3 img.
.1 lib.
.1 src.
.2 build.
.2 client.
.2 server.
.2 common.
.3 utils.
}

\noindent Skrót \texttt{MD5}~zawartości płyty:
\begin{center}
\texttt{TODO}
\end{center}

\chapter{Plakat}
Zgodnie z~wytycznymi Wydziału MiNI, dotyczącymi procedury związanej z~obroną pracy dyplomowej, do~niniejszej pracy dołączono poster z~podstawowymi informacjami dotyczącymi opracowanej pracy magisterskiej~\cite{informacje-dot-obron}. Plakat został załączony w~postaci wydruku na~arkuszu formatu~A3 oraz na~płycie w~pliku \mbox{\texttt{poster.svg}}~(patrz załącznik~\ref{cd-appendix}). Plakat wykonano w~grafice wektorowej w programie~Inkscape.zwięźle podsumowuje niniejszą pracę magisterską.

\end{appendices}

\end{document}
