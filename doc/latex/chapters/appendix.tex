\documentclass[thesis]{subfiles}

\begin{document}

% Załączniki -- może lepszy: \usepackage[toc,page]{appendix} ?
% Patrz: http://tex.stackexchange.com/questions/49643/making-appendix-for-thesis

\begin{appendices}

% Zamienia domyślnych tytułów 'Dodatek X' na 'Załącznik X'

\newcommand\zalacznik{Załącznik}
\renewcommand{\appendixname}{\zalacznik}

% Ustawienie, żeby załączniki były wyświetlane w Spisie Treści z wcięciem
% http://tex.stackexchange.com/questions/172496/adding-appendix-chapters-without-sections-in-table-of-contents

\makeatletter
\addtocontents{toc}{%
	\begingroup
	\let\protect\l@chapter\protect\l@section
	\let\protect\l@section\protect\l@subsection
}
\makeatother

%------------------------------------------------------------------------------

% Słownik pojęć

\chapter{Słownik pojęć}

%\printglossary[style=mcolindex,title=Słownik pojęć]
\printglossary[style=labeling,title=Słownik pojęć]
%\glsaddall % wypisuje również nieużywane pozycje słownika

%------------------------------------------------------------------------------

\chapter{Przykład działania AIDE}
\label{ch:aide-example}

Niniejszy załącznik ma~na~celu przybliżenie sposobu działania AIDE. Do~tego celu posłużył prosty przykład wykorzystania go~do~śledzenia zmian dokonanych na~demonstracyjnym drzewie katalogów \path{/tmp/demo}, utworzonym w~sposób przedstawiony na~listingu~\ref{lst:tree-create-aide-example} i~zobrazowanym na~rysunku~\ref{fig:drzewo-katalogow-przyklad-przed-aide}. Listing~\ref{lst:aide-config-example} jest przykładową konfiguracją AIDE wykorzystaną na~potrzeby prezentacji działania AIDE, listing~\ref{lst:aide-example-first-scan} zawiera komendy, które inicjują za~pomocą użycia flagi \hreftt{https://linux.die.net/man/1/aide}{--init} plik stanu AIDE \path{/tmp/aide.db.new}, odpowiadający skanowanemu drzewu katalogów, a~następnie przenoszą go~do~pliku \path{/tmp/aide.db}, ustalając w~ten sposób referencyjny plik stanu, do~którego są~później porównywane wyniki skanowania z~flagą \hreftt{https://linux.die.net/man/1/aide}{--scan} lub~\hreftt{https://linux.die.net/man/1/aide}{--compare}~(por.~\hyperref[line:aide-first-config-line]{pierwszy} wiersz konfiguracji \path{/tmp/aide-demo.conf} z~listingu~\ref{lst:aide-config-example}). Listing~\ref{lst:tree-change-aide-example} symuluje dokonanie zmian zawartości plików i~katalogów w~drzewie \path{/tmp/demo}, odzwierciedlonych w~drzewie na~rysunku~\ref{fig:drzewo-katalogow-przyklad-po-aide}. Wywołanie skanowania AIDE z~flagą \hreftt{https://linux.die.net/man/1/aide}{--check} powoduje wypisanie na~standardowe wyjście raportu dokonanych zmian w~skanowanym drzewie, co~przedstawia listing~\ref{lst:aide-example-second-scan} na~stronach~\pageref{lst:aide-example-second-scan}--\pageref{aide-example-second-scan-last-page}. Jeśli zaszłaby potrzeba zamiany starego pliku referencyjnego \path{/tmp/aide.db} na~nowy plik stanu \path{/tmp/aide.db.new}, to~należałoby wywołać AIDE z~opcją \hreftt{https://linux.die.net/man/1/aide}{--update}, co~przedstawia listing~\ref{lst:aide-example-update-output}. Zawartość nowego pliku \path{/tmp/aide.db.new} została przedstawiona na~listingu~\ref{lst:aide-example-second-scan-output}.

\begin{minipage}{\linewidth}
\begin{lstlisting}[language=,caption={Konfiguracja AIDE \protect\path{/tmp/aide-demo.conf} zastosowana do~prezentacji działania AIDE na~przykładowym drzewie katalogów~\protect\path{/tmp/demo}, utworzonym w~sposób przedstawiony na~listingu~\ref{lst:tree-create-aide-example} i~zobrazowanym na~rys.~\ref{fig:drzewo-katalogow-przyklad-przed-aide}},label=lst:aide-config-example,escapechar=|]
database = file:/tmp/aide.db|\label{line:aide-first-config-line}|
database_out = file:/tmp/aide.db.new
database_new = file:/tmp/aide.db.new
gzip_dbout = no
summarize_changes = yes
grouped = yes
verbose = 6|\label{line:aide-verbosity-line}|
report_base16 = no
MyRule = p+ftype+i+n+l+u+g+s+b+m+c+md5+crc32
/tmp/demo/.* MyRule
!/tmp/demo/dir1/subdir2/.*
\end{lstlisting}
\end{minipage}

\begin{minipage}{\linewidth}
\begin{lstlisting}[language=,caption={Utworzenie przykładowego drzewa katalogów \protect\path{/tmp/demo}, w~tym --- 9~katalogów, 11~\href{https://en.wikipedia.org/wiki/Unix_file_types\#Regular_file}{plików zwykłych}~(\emph{regular files}), 2~\href{https://en.wikipedia.org/wiki/Symbolic_link}{dowiązań symbolicznych}~(\emph{symbolic links}), 2~\href{https://en.wikipedia.org/wiki/Hard_link}{dowiązań twardych}~(\emph{hard links}) i~2~\href{https://en.wikipedia.org/wiki/Named_pipe}{łączy nazwanych}~(\emph{named pipes}/\emph{FIFOs}) na~potrzeby przykładu działania AIDE (por.~rys.~\ref{fig:drzewo-katalogow-przyklad-przed-aide})},label=lst:tree-create-aide-example]
user@host:/tmp$ mkdir demo && cd demo
user@host:/tmp/demo$ echo '0th file' > file0
user@host:/tmp/demo$ echo '1st file' > file1
user@host:/tmp/demo$ mkdir dir{0..3} dir1/subdir{0..3}
user@host:/tmp/demo$ ln -s file1 file1_symlink
user@host:/tmp/demo$ ln -s dir3 dir3_symlink
user@host:/tmp/demo$ touch dir1/file
user@host:/tmp/demo$ echo '2nd file' > dir1/file2
user@host:/tmp/demo$ echo '3rd file' > dir1/file3
user@host:/tmp/demo$ echo '4th file' > dir1/subdir1/file4
user@host:/tmp/demo$ echo '5th file' > dir1/subdir1/file5
user@host:/tmp/demo$ echo '6th file' > dir1/subdir2/file6
user@host:/tmp/demo$ echo '7th file' > dir1/subdir2/file7
user@host:/tmp/demo$ mkfifo dir2/fifo{0,1}
user@host:/tmp/demo$ echo '8th file' > dir3/file8
user@host:/tmp/demo$ echo '9th file' > dir3/file9
user@host:/tmp/demo$ ln dir3/file8 file8_hardlink
user@host:/tmp/demo$ chmod 777 dir1/file2
user@host:/tmp/demo$ ln dir1/subdir1/file4 file4_hardlink
user@host:/tmp/demo$ chgrp wireshark dir1/file3
\end{lstlisting}
\end{minipage}

%\begin{minipage}{\linewidth}
	Listing~\ref{lst:aide-example-second-scan} na~stronach~\pageref{lst:aide-example-second-scan}--\pageref{aide-example-second-scan-last-page} jest wyjściem programu AIDE, które zostaje przetworzone przez rozwiązanie zaimplementowane w~ramach niniejszej pracy, w~celu ustalenia jakie zmiany zaszły w~systemie plików od~czasu ostatniego, referencyjnego skanowania, którego wynik został zapisany w~przedstawionym przykładzie w~pliku \path{/tmp/aide.db}. Wywołanie AIDE z~flagą \hreftt{https://linux.die.net/man/1/aide}{--scan} z~listingu~\ref{lst:aide-example-second-scan} działa na~konfiguracji AIDE \path{/tmp/aide-demo.conf} przedstawionej na~listingu~\ref{lst:aide-config-example}, w~której w~linii~\ref{line:aide-verbosity-line} ustalono parametr ,,gadatliwości'' (\emph{verbosity}) na~wartość~\texttt{6} --- w~rzeczywistości, w~zaimplementowanym rozwiązaniu wartość ta~jest większa celem ułatwienia parsowania wyjścia AIDE, jednak na~potrzeby przykładu przyjęto mniejszą wartość, aby~wynik działania AIDE nie był bardzo długi i~nieczytelny.

	Sekcje \texttt{Added entries}, \texttt{Removed entries} i~\texttt{Changed entries} z~listingu~\ref{lst:aide-example-second-scan} odpowiadają odpowiednio plikom dodanym, usuniętym i~zmienionym (pod względem zawartości lub~atrybutów) od~czasu skanowania referencyjnego. Za~obecność tych sekcji na~wyjściu programu odpowiada opcja \texttt{summarize\_changes = yes} w~konfiguracji AIDE z~listingu \ref{lst:aide-config-example}. Po~wymienionych trzech sekcjach następują linie opisujące co~i~jak się zmieniło w~śledzonych plikach i~ich atrybutach. Przykładowo rozmiar pliku \path{/tmp/demo/file1} zmienił się z~9 na~11~bajtów po~dopisaniu do~niego litery \texttt{x} i~znaku \href{https://superuser.com/questions/395668/how-do-i-echo-something-file-txt-without-carriage-return}{nowej linii} (patrz linia~\ref{line:aide-append-x-char} z~listingu~\ref{lst:tree-change-aide-example}). Po~zmianie zawartości omawianego pliku zmieniły się również jego wartości skrótu \href{https://en.wikipedia.org/wiki/MD5}{MD5} i~\href{https://en.wikipedia.org/wiki/Cyclic_redundancy_check}{CRC32}. Warto zwrócić uwagę na~to, że~AIDE wykrył zmianę ilości dowiązań twardych (z~2 na~1) do~pliku \path{/tmp/demo/dir3/file8} spowodowaną usunięciem pliku \path{/tmp/demo/file8_hardlink}, ale~nie odnotował zmiany ilości dowiązań twardych do~pliku \path{/tmp/demo/dir1/subdir2/file7} po~stworzeniu dowiązania twardego \path{/tmp/demo/file7_hardlink}, którego samo stworzenie zostało odnotowane. Powodem takiego zachowania jest ustawienie w~\hyperref[lst:aide-config-example]{pliku konfiguracyjnym} linii \texttt{!/tmp/demo/dir1/subdir2/.*}, która wyłącza ten~katalog ze~skanowania AIDE.
%\end{minipage}

\begin{minipage}{\linewidth}
\begin{lstlisting}[language=,caption={Uruchomienie pierwszego skanowania AIDE z~konfiguracją \protect\path{/tmp/aide-demo.conf} i~z~flagą \hreftt{https://linux.die.net/man/1/aide}{--init}, oznaczającą inicjalizację referencyjnej bazy wynikowej \protect\path{/tmp/aide.db.new}, opisującej stan systemu},label=lst:aide-example-first-scan]
user@host:/tmp/demo$ aide --init -c /tmp/aide-demo.conf
Start timestamp: 2017-06-12 23:50:52 +0200 (AIDE 0.16)
AIDE initialized database at /tmp/aide.db.new
Verbose level: 6

Number of entries:	22

---------------------------------------------------
The attributes of the (uncompressed) database(s):
---------------------------------------------------

/tmp/aide.db.new
  MD5      : s0dlxhIPglO538vgaDr3ZQ==
  SHA1     : RDle9veNHRdgf8fZA2xWVrKrRMM=
  RMD160   : a/rNYSmrCmq4Vwk0BpocigPqCRQ=
  TIGER    : JvTaz+LrZPC7xUYXRO3wXxXVgpxhL2me
  SHA256   : oSutM/qSGsAecbTyGJrQDKFd1LwEV2IG
             t6hmcKECjWI=
  SHA512   : IugVANSKAfh0gJ+EgBuR8SNQ5m8Sxckz
             NANR2MjSZvgSB1QO9gN2Uiqnn2WW/z2I
             03/qiu4xQPx055zgYJluKA==
  CRC32    : iDo0aw==
  HAVAL    : fVpHqgYtV8qgt368sIwDE1ZNjSP5bEwP
             HUoK5RSiPkQ=
  GOST     : ybuJquCJz3rTKT/HRN+VrEAlUUtIxai0
             L8PH4FfbUDM=
  WHIRLPOOL: 9qqFSseiin1TG8P+4PVoYp7Vs3XMwZD+
             kjaisnYJ6FNj6MyOgLBfkdOCwtBM+M6s
             W7d7gKm9kmXI7FkPAi9oCw==


End timestamp: 2017-06-12 23:50:52 +0200 (run time: 0m 0s)
user@host:/tmp/demo$ mv ../aide.db.new ../aide.db
\end{lstlisting}
\end{minipage}

\vfill

\begin{lstlisting}[language=,caption={Dokonanie demonstracyjnych zmian w~śledzonym przez AIDE drzewie katalogów \protect\path{/tmp/demo}, utworzonym w~sposób przedstawiony na~listingu~\ref{lst:tree-create-aide-example}~(por.~rys.~\ref{fig:drzewo-katalogow-przyklad-przed-aide} i~\ref{fig:drzewo-katalogow-przyklad-po-aide}) przed uruchomieniem drugiego skanowania AIDE z~flagą \hreftt{https://linux.die.net/man/1/aide}{--check}~(patrz listing~\ref{lst:aide-example-second-scan})},label=lst:tree-change-aide-example,escapechar=|]
user@host:/tmp/demo$ mv dir2/fifo0 dir0
user@host:/tmp/demo$ echo 'x' >> file1|\label{line:aide-append-x-char}|
user@host:/tmp/demo$ rm file8_hardlink dir1/subdir1/file4
user@host:/tmp/demo$ mv dir3/file9 .
user@host:/tmp/demo$ chmod o-rwx dir1/file2
user@host:/tmp/demo$ chgrp user dir1/file3
user@host:/tmp/demo$ mv file0 file0_renamed
user@host:/tmp/demo$ ln -s dir1/subdir2/file6 file6_symlink
user@host:/tmp/demo$ ln dir1/subdir2/file7 file7_hardlink
user@host:/tmp/demo$ touch dir1/subdir1/file5
\end{lstlisting}

\newpage
\begin{lstlisting}[language=,caption={Drugie skanowanie drzewa katalogów \protect\path{/tmp/demo} --- skanowanie po~zmianach przedstawionych na~listingu~\ref{lst:tree-change-aide-example} i~zwizualizowanych na~rys.~\ref{fig:drzewo-katalogow-przyklad-aide}},label=lst:aide-example-second-scan]
user@host:/tmp/demo$ aide --check -c /tmp/aide-demo.conf 
Start timestamp: 2017-06-13 00:05:07 +0200 (AIDE 0.16)
AIDE found differences between database and filesystem!!
Verbose level: 6

Summary:
  Total number of entries:	22
  Added entries:		5
  Removed entries:		5
  Changed entries:		10

---------------------------------------------------
Added entries:
---------------------------------------------------

p++++++++++++++++: /tmp/demo/dir0/fifo0
f++++++++++++++++: /tmp/demo/file0_renamed
l++++++++++++++++: /tmp/demo/file6_symlink
f++++++++++++++++: /tmp/demo/file7_hardlink
f++++++++++++++++: /tmp/demo/file9

---------------------------------------------------
Removed entries:
---------------------------------------------------

f----------------: /tmp/demo/dir1/subdir1/file4
p----------------: /tmp/demo/dir2/fifo0
f----------------: /tmp/demo/dir3/file9
f----------------: /tmp/demo/file0
f----------------: /tmp/demo/file8_hardlink

---------------------------------------------------
Changed entries:
---------------------------------------------------

d =.... mc..     : /tmp/demo/dir0
f =.p.. .c...    : /tmp/demo/dir1/file2
f =...g .c...    : /tmp/demo/dir1/file3
d =.... mc..     : /tmp/demo/dir1/subdir1
f =.... mc...    : /tmp/demo/dir1/subdir1/file5
d =.... mc..     : /tmp/demo/dir2
d =.... mc..     : /tmp/demo/dir3
f =.... .c.n.    : /tmp/demo/dir3/file8
f >.... mc..C    : /tmp/demo/file1
f =.... .c.n.    : /tmp/demo/file4_hardlink

---------------------------------------------------
Detailed information about changes:
---------------------------------------------------

Directory: /tmp/demo/dir0
  Mtime    : 2017-06-12 23:43:48 +0200        | 2017-06-12 23:54:03 +0200
  Ctime    : 2017-06-12 23:43:48 +0200        | 2017-06-12 23:54:03 +0200

File: /tmp/demo/dir1/file2
  Perm     : -rwxrwxrwx                       | -rwxrwx---
  Ctime    : 2017-06-12 23:47:21 +0200        | 2017-06-12 23:59:37 +0200

File: /tmp/demo/dir1/file3
  Gid      : 127                              | 1000
  Ctime    : 2017-06-12 23:47:48 +0200        | 2017-06-13 00:00:00 +0200

Directory: /tmp/demo/dir1/subdir1
  Mtime    : 2017-06-12 23:45:42 +0200        | 2017-06-12 23:57:08 +0200
  Ctime    : 2017-06-12 23:45:42 +0200        | 2017-06-12 23:57:08 +0200

File: /tmp/demo/dir1/subdir1/file5
  Mtime    : 2017-06-12 23:45:42 +0200        | 2017-06-13 00:04:30 +0200
  Ctime    : 2017-06-12 23:45:42 +0200        | 2017-06-13 00:04:30 +0200

Directory: /tmp/demo/dir2
  Mtime    : 2017-06-12 23:46:22 +0200        | 2017-06-12 23:54:03 +0200
  Ctime    : 2017-06-12 23:46:22 +0200        | 2017-06-12 23:54:03 +0200

Directory: /tmp/demo/dir3
  Mtime    : 2017-06-12 23:46:54 +0200        | 2017-06-12 23:58:06 +0200
  Ctime    : 2017-06-12 23:46:54 +0200        | 2017-06-12 23:58:06 +0200

File: /tmp/demo/dir3/file8
  Ctime    : 2017-06-12 23:47:10 +0200        | 2017-06-12 23:56:31 +0200
  Linkcount: 2                                | 1

File: /tmp/demo/file1
  Size     : 9                                | 11
  Mtime    : 2017-06-12 23:43:11 +0200        | 2017-06-12 23:56:07 +0200
  Ctime    : 2017-06-12 23:43:11 +0200        | 2017-06-12 23:56:07 +0200
  MD5      : PhzxiZvr9xGEi3W4/w4tjw==         | 3YTsP47ztvqew1yDoSNJpw==
  CRC32    : DHu60A==                         | ELsdEQ==

File: /tmp/demo/file4_hardlink
  Ctime    : 2017-06-12 23:47:37 +0200        | 2017-06-12 23:57:08 +0200
  Linkcount: 2                                | 1


---------------------------------------------------
The attributes of the (uncompressed) database(s):
---------------------------------------------------

/tmp/aide.db
  MD5      : s0dlxhIPglO538vgaDr3ZQ==
  SHA1     : RDle9veNHRdgf8fZA2xWVrKrRMM=
  RMD160   : a/rNYSmrCmq4Vwk0BpocigPqCRQ=
  TIGER    : JvTaz+LrZPC7xUYXRO3wXxXVgpxhL2me
  SHA256   : oSutM/qSGsAecbTyGJrQDKFd1LwEV2IG
             t6hmcKECjWI=
  SHA512   : IugVANSKAfh0gJ+EgBuR8SNQ5m8Sxckz
             NANR2MjSZvgSB1QO9gN2Uiqnn2WW/z2I
             03/qiu4xQPx055zgYJluKA==
  CRC32    : iDo0aw==
  HAVAL    : fVpHqgYtV8qgt368sIwDE1ZNjSP5bEwP
             HUoK5RSiPkQ=
  GOST     : ybuJquCJz3rTKT/HRN+VrEAlUUtIxai0
             L8PH4FfbUDM=
  WHIRLPOOL: 9qqFSseiin1TG8P+4PVoYp7Vs3XMwZD+
             kjaisnYJ6FNj6MyOgLBfkdOCwtBM+M6s
             W7d7gKm9kmXI7FkPAi9oCw==


End timestamp: 2017-06-13 00:05:07 +0200 (run time: 0m 0s)
\end{lstlisting}
\label{aide-example-second-scan-last-page} % żeby móc się odwołać do ostatniej strony zajmowanej przez listing

\begin{lstlisting}[language=,caption={Zaktualizowanie pliku stanu systemu \protect\path{/tmp/aide.db.new} po~skanowaniu drzewa katalogów \protect\path{/tmp/demo} po~dokonaniu w~nim zmian przedstawionych na~listingu~\ref{lst:tree-change-aide-example} i~zwizualizowanych na~rys.~\ref{fig:drzewo-katalogow-przyklad-aide}},label=lst:aide-example-update-output]
user@host:/tmp/demo$ aide --update -c /tmp/aide-demo.conf 
Start timestamp: 2017-06-13 00:11:49 +0200 (AIDE 0.16)
AIDE found differences between database and filesystem!!
New AIDE database written to /tmp/aide.db.new

(...) Skipped fragment identical to above command's output (...)

/tmp/aide.db.new
  MD5      : eORTPtiPLrkyWoezwMCLug==
  SHA1     : Zqn8oDZlMMVmnbBplX045dlEcf8=
  RMD160   : 9ddczy9kGt8BfEuDbJb1XrkmpGA=
  TIGER    : FdC9kqp9fiMhLxifEtzqI27gScXEbhni
  SHA256   : EbD/BuWkqf+IFpuDKm10KDSJ66VSL2K+
             rwcTtiO9siw=
  SHA512   : cQTUtyiOf9tVhDg+nxnGts4/nXeP62tg
             OTgAYVwxqapOco8oXIlXWiAPmBi6iUbA
             Gb5rqKkL2jjn5gN+E4iJKA==
  CRC32    : 65CBGg==
  HAVAL    : J33vR7P/PWSC30sA16E1yE2L/gNfWI/p
             JLHhajaQNY4=
  GOST     : cQ/9GpeTXXI7fMbyFpFGa/P4i2pG+6h9
             nz62FI6U4kY=
  WHIRLPOOL: zae+ZMAF13z08+d264FaeZXY7vxWjLWY
             A7Hzao29+EksBihJ2n4qJb85sSRR7hwB
             dHssv+Z0QFO/k1InWBpeCQ==


End timestamp: 2017-06-13 00:11:49 +0200 (run time: 0m 0s)
\end{lstlisting}

\begin{minipage}{\linewidth}
\begin{lstlisting}[language=,caption={Zaktualizowany plik stanu systemu \protect\path{/tmp/aide.db.new} po~skanowaniu drzewa katalogów \protect\path{/tmp/demo} po~dokonaniu w~nim zmian przedstawionych na~listingu~\ref{lst:tree-change-aide-example} i~zwizualizowanych na~rys.~\ref{fig:drzewo-katalogow-przyklad-aide}},label=lst:aide-example-second-scan-output,escapechar=|]
user@host:/tmp/demo$ cat ../aide.db.new
@@begin_db
# This file was generated by Aide, version 0.16
# Time of generation was 2017-06-13 00:11:49
@@db_spec name lname attr perm inode bcount uid gid size mtime ctime lcount md5 crc32|\label{line:aidedbnew-columns-names}|
/tmp/demo/dir0 0 34359742397 40755 22 8 1000 1000 4096 MTQ5NzMwNDQ0Mw== MTQ5NzMwNDQ0Mw== 2 0 0|\label{line:aidedbnew-first-from-range}|
/tmp/demo/dir0/fifo0 0 34359742397 10644 40 0 1000 1000 0 MTQ5NzMwMzk4Mg== MTQ5NzMwNDQ0Mw== 1 0 0
/tmp/demo/dir1 0 34359742397 40755 23 8 1000 1000 4096 MTQ5NzMwMzkxMA== MTQ5NzMwMzkxMA== 6 0 0
/tmp/demo/dir1/file 0 34359812029 100644 33 0 1000 1000 0 MTQ5NzMwMzg3Ng== MTQ5NzMwMzg3Ng== 1 1B2M2Y8AsgTpgAmY7PhCfg== AAAAAA==
/tmp/demo/dir1/file2 0 34359812029 100770 34 8 1000 1000 9 MTQ5NzMwMzg5NA== MTQ5NzMwNDc3Nw== 1 fPzlv5u5G40YOOqm53vNkQ== fae7bA==
/tmp/demo/dir1/file3 0 34359812029 100644 35 8 1000 1000 9 MTQ5NzMwMzkxMA== MTQ5NzMwNDgwMA== 1 w6pbLq3ndYjJReSudThjtQ== cRmpjQ==
/tmp/demo/dir1/subdir0 0 34359742397 40755 27 8 1000 1000 4096 MTQ5NzMwMzgyOA== MTQ5NzMwMzgyOA== 2 0 0
/tmp/demo/dir1/subdir1 0 34359742397 40755 28 8 1000 1000 4096 MTQ5NzMwNDYyOA== MTQ5NzMwNDYyOA== 2 0 0
/tmp/demo/dir1/subdir1/file5 0 34359812029 100644 37 8 1000 1000 9 MTQ5NzMwNTA3MA== MTQ5NzMwNTA3MA== 1 DDmr9IRrE2rSn6A3nfDZMg== Vc6IPw==
/tmp/demo/dir1/subdir3 0 34359742397 40755 30 8 1000 1000 4096 MTQ5NzMwMzgyOA== MTQ5NzMwMzgyOA== 2 0 0
/tmp/demo/dir2 0 34359742397 40755 25 8 1000 1000 4096 MTQ5NzMwNDQ0Mw== MTQ5NzMwNDQ0Mw== 2 0 0
/tmp/demo/dir2/fifo1 0 34359742397 10644 41 0 1000 1000 0 MTQ5NzMwMzk4Mg== MTQ5NzMwMzk4Mg== 1 0 0
/tmp/demo/dir3 0 34359742397 40755 26 8 1000 1000 4096 MTQ5NzMwNDY4Ng== MTQ5NzMwNDY4Ng== 2 0 0
/tmp/demo/dir3/file8 0 34359812029 100644 42 8 1000 1000 9 MTQ5NzMwNDAwMw== MTQ5NzMwNDU5MQ== 1 jXL6coUzcqJ4eO2B9wVhKA== R+iLhg==
/tmp/demo/dir3_symlink dir3 34359742399 120777 32 0 1000 1000 4 MTQ5NzMwMzg3MA== MTQ5NzMwMzg3MA== 1 0 0
/tmp/demo/file0_renamed 0 34359812029 100644 15 8 1000 1000 9 MTQ5NzMwMzc4Mw== MTQ5NzMwNDg3Nw== 1 Xk/ZsDDB3q6NEc1SJSVpCQ== cuvJCA==
/tmp/demo/file1 0 34359812029 100644 20 8 1000 1000 11 MTQ5NzMwNDU2Nw== MTQ5NzMwNDU2Nw== 1 3YTsP47ztvqew1yDoSNJpw== ELsdEQ==
/tmp/demo/file1_symlink file1 34359742399 120777 31 0 1000 1000 5 MTQ5NzMwMzg1OQ== MTQ5NzMwMzg1OQ== 1 0 0
/tmp/demo/file4_hardlink 0 34359812029 100644 36 8 1000 1000 9 MTQ5NzMwMzkzMg== MTQ5NzMwNDYyOA== 1 qX9IDI1OOYA5OfoKfNHY9Q== M2SIzQ==
/tmp/demo/file6_symlink dir1/subdir2/file6 34359742399 120777 45 0 1000 1000 18 MTQ5NzMwNDk0MA== MTQ5NzMwNDk0MA== 1 0 0
/tmp/demo/file7_hardlink 0 34359812029 100644 39 8 1000 1000 9 MTQ5NzMwMzk1Ng== MTQ5NzMwNDk2Mg== 2 cgHdmt/roeP+P+HwBdK83g== LodI3w==
/tmp/demo/file9 0 34359812029 100644 43 8 1000 1000 9 MTQ5NzMwNDAxNA== MTQ5NzMwNDY4Ng== 1 1zYCmaRbg4+M57I7ICdgWw== IUKLdA==|\label{line:aidedbnew-last-from-range}|
@@end_db
\end{lstlisting}
\end{minipage}
%user@host:/tmp/demo$ aide --compare -c /tmp/aide-demo.conf 
%Start timestamp: 2017-06-13 00:16:26 +0200 (AIDE 0.16)
%AIDE found differences between the two databases!!
%
%(...) Wynik na stdout taki jak dla opcji --update (...)
%

%\begin{minipage}{\linewidth} % https://tex.stackexchange.com/questions/73231/avoid-page-breaks-in-lstlistings
\begin{figure}[h!]
	\centering
	\begin{subfigure}[t]{0.49\textwidth}
			\mybox{dirs}{%
				\dirtree{%
				.1 /.
				.2 \vdots.
				.2 tmp.
				.3 \vdots.
				.3 demo.
				.4 dir0.
				.4 dir1.
				.5 file.
				.5 file2.
				.5 file3.
				.5 subdir0.
				.5 subdir1.
				.6 file4.
				.6 file5.
				.5 subdir2.
				.6 file6.
				.6 file7.
				.5 subdir3.
				.4 dir2.
				.5 fifo0.
				.5 fifo1.
				.4 dir3.
				.5 file8.
				.5 file9.
				.4 dir3\_symlink -> dir3.
				.4 file0.
				.4 file1.
				.4 file1\_symlink -> file1.
				.4 file4\_hardlink.
				.4 file8\_hardlink.
				}
				\tcblower
				\caption{Drzewo katalogów \path{/tmp/demo} utworzone w~sposób przedstawiony na~listingu~\ref{lst:tree-create-aide-example} przed~dokonaniem w~nim zmian z~listingu~\ref{lst:tree-change-aide-example}}
				\label{fig:drzewo-katalogow-przyklad-przed-aide}
			}
	\end{subfigure}
	\begin{subfigure}[t]{0.49\textwidth}
			\mybox{dirs}{%
				\dirtree{%
				.1 /.
				.2 \vdots.
				.2 tmp.
				.3 \vdots.
				.3 demo.
				.4 dir0.
				.5 fifo0.
				.4 dir1.
				.5 file.
				.5 file2.
				.5 file3.
				.5 subdir0.
				.5 subdir1.
				.6 file5.
				.5 subdir2.
				.6 file6.
				.6 file7.
				.5 subdir3.
				.4 dir2.
				.5 fifo1.
				.4 dir3.
				.5 file8.
				.4 dir3\_symlink -> dir3.
				.4 file0\_renamed.
				.4 file1.
				.4 file1\_symlink -> file1.
				.4 file4\_hardlink.
				.4 file6\_symlink -> dir1/subdir2/file6.
				.4 file7\_hardlink.
				.4 file9.
				}
				\tcblower
				\caption{Drzewo katalogów \path{/tmp/demo} utworzone w~sposób przedstawiony na~listingu~\ref{lst:tree-create-aide-example} po~dokonaniu w~nim zmian z~listingu~\ref{lst:tree-change-aide-example}}
				\label{fig:drzewo-katalogow-przyklad-po-aide}
			}
	\end{subfigure}
	\caption[Drzewo katalogów \protect\path{/tmp/demo} utworzone na~potrzeby demonstracji działania skanera~AIDE]{Drzewo katalogów \protect\path{/tmp/demo} utworzone na~potrzeby demonstracji działania skanera AIDE w~sposób przedstawiony na~listingu~\ref{lst:tree-create-aide-example} --- widok przed i~po~dokonaniu zmian przedstawionych na~listingu~\ref{lst:tree-change-aide-example}}
	\label{fig:drzewo-katalogow-przyklad-aide}
\end{figure}
%\end{minipage}

Każdy z~wierszy \ref{line:aidedbnew-first-from-range}--\ref{line:aidedbnew-last-from-range} listingu~\ref{lst:aide-example-second-scan-output} odpowiada pojedynczemu plikowi lub~katalogowi śledzonemu przez AIDE, a~każda z~kolumn, których nazwy znajdują się w~linii~\ref{line:aidedbnew-columns-names}, odpowiada wybranym w~konfiguracji atrybutom śledzonych plików~(por.~listing~\ref{lst:aide-config-example}). Kolumna \texttt{name} oznacza pełną ścieżkę do~śledzonego pliku, kolumna \texttt{lname} to~\texttt{0} jeśli plik nie jest dowiązaniem symbolicznym, a~w~przeciwnym wypadku to~pełna ścieżka do~pliku docelowego, \texttt{attr} to~\href{https://unix.stackexchange.com/questions/342693/understanding-aide-db/343020}{zakodowane kody} żądanych w~konfiguracji AIDE atrybutów śledzonych plików. Pozostałe kolumny odpowiadają atrybutom o~kodach opisanych w~tabeli~\ref{tab:aide-file-attrs}.

%------------------------------------------------------------------------------

% Płyta CD z pracą magisterską

\chapter{Płyta CD}
\label{ch:cd-appendix}

Do~pracy dołączono płytę z~zawartością części aplikacyjnej niniejszej pracy magisterskiej. Drzewo najważniejszych\footnote{Zaprezentowane drzewo katalogów nie prezentuje całej zawartości płyty~CD, a~jedynie wybrane, najistotniejsze katalogi i~pliki do~głębokości drzewa równej~4. Drzewo powstało przez wybranie najważniejszych plików z~wyjścia wywołania komendy \mbox{\texttt{tree~-L~4}}~\cite{tree-manual}.} katalogów i~plików zapisanych na~płycie~CD: % TODO

{\footnotesize
\dirtree{%
.1 /.
.2 bin.
.3 packages.
.4 debian.
.5 myscm-srv.deb.
.5 myscm-cli.deb.
.4 arch.
.5 myscm-srv.tar.xz.
.5 myscm-cli.tar.xz.
.2 config.
.3 server.
.4 certificate.pem.
.4 request.cnf.
.4 request.csr.
.4 rsa\_aes256\_4096.key.
.4 rsa\_aes256\_4096.key.pkcs12.
.2 data.
.2 doc.
.3 python.
.3 latex.
.4 chapters.
.4 img.
.3 poster.
.2 lib.
.2 src.
.3 client.
.3 server.
}
}

\noindent Skrót \texttt{MD5}~zawartości płyty:\mynobreakpar
\begin{center}
\texttt{TODO}
\end{center}

%------------------------------------------------------------------------------

% Plakat (poster)

\chapter{Poster}

Zgodnie z~wytycznymi Wydziału MiNI, dotyczącymi procedury związanej z~obroną pracy dyplomowej, do~niniejszej pracy dołączono poster z~podstawowymi informacjami dotyczącymi opracowanej pracy magisterskiej~\cite{informacje-dot-obron}.

Poster został wykonany w~grafice wektorowej w~programie~\texttt{Inkscape} i~został załączony w~dwóch formach, w~postaci:
\begin{enumerate}
	\item wydruku na~arkuszu formatu~A3,
	\item pliku \mbox{\texttt{poster.svg}}~(patrz załącznik~\ref{ch:cd-appendix}), nagranym na~załączonej płycie~CD.
\end{enumerate}

\addtocontents{toc}{\endgroup}

\end{appendices}

\end{document}
