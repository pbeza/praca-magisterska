\documentclass[12pt,a4paper]{article}
\usepackage[main=english]{babel}
\usepackage{graphicx}
\usepackage[hidelinks]{hyperref}
\usepackage[parfill]{parskip}[2001/04/09]

\usepackage[utf8]{inputenc}
\usepackage[T1]{fontenc}
\usepackage{mathptmx}

\linespread{1.25}
\setlength{\parindent}{0cm}

\pagestyle{empty} % no page numbers!
\usepackage[left=35mm, right=35mm, top=15mm, bottom=20mm, noheadfoot]{geometry}

\begin{document}

\thispagestyle{empty}
\title{\textbf{Management Protocol\\for~Linux Workstations}}
\author{Patryk Bęza\\[-8pt]
\small \texttt{\href{mailto:bezap@student.mini.pw.edu.pl}{bezap@student.mini.pw.edu.pl}}\\[6pt]
\small Faculty of~Mathematics and~Information Science\\[-8pt]
\small Warsaw University of Technology\\[-8pt]
\small Koszykowa 75, 00-662~Warsaw
}
\date{}
\maketitle\thispagestyle{empty}

This paper entitled 'Management Protocol for Linux Workstations' concerns the~automation of~identically or~similarly configured group of~computers. The~motivation behind this topic is~the~need to~automate configuration, for~example, in~some server environments used for~distributed computing, data centers, ticket machines, ATMs, offices, public administration, universities, colleges, schools and other places where it~is desirable for~computer workstations to~have similar set of~installed software and~configuration. Automation of~the software distribution process is~intended to~facilitate the~work of~computer administrators by~minimizing or~exempting them from the need to~repeat a~similar or~identical installation and~configuration procedure for~multiple machines, thereby reducing the~risk of~configuration errors.

The~goal of~this thesis~is to~design and~implement a~protocol that allows to~automatically propagate changes in~the~file~system, packages, and configuration on~clients' workstations running under Linux/GNU or~other *nix operating system, as~well as~flexible standard description of~configuration changes and~tools recording and~customizing such a~changes. In~addition, the~beginning of~the~paper includes a~description of~existing solutions of~the~software distribution problem and~comparison of~the~functionality of~the~existing solutions with the~tools created as~a~part of~this thesis. The~main part of~the~paper has been devoted to~a~description of~aspects of~the~designed alternative tool that allows applying changes prepared by~network administrator as~a~system image. In particular, the paper describes the functionality of the tools implementing the proposed protocol, application's security issues, libraries used, tests, possible directions for further development and conclusions.

\end{document}
