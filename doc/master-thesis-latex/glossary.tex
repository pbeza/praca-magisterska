%
% Ze względu na custom'owy styl zastosowany w glossaries zamiast '\\' używamy '\\&' (bo słownik to tabelka dwukolumnowa).
% Na końcu każdego wpisu w słowniku dodaje się automatycznie znak kropki '.' i nowej linii '\\'.
%

\newglossaryentry{gnulinux}
{
	name={GNU/Linux},
	description={System operacyjny \emph{GNU} napisany przez Richarda Stallmana, z jądrem \emph{Linux}, napisanym przez \emph{Linusa Torvalds'a}}
}
\newglossaryentry{RAM}
{
	name={RAM},
	description={\textit{Random Access Memory}.\\&Podstawowy  rodzaj  pamięci  cyfrowej
   wykorzystywana w komputerach}
}
\newglossaryentry{RSA}
{
	name={RSA},
	description={Jeden z pierwszych i obecnie najpopularniejszych asymetrycznych algorytmów kryptograficznych z kluczem publicznym, zaprojektowany w 1977 przez Rona Rivesta, Adi Shamira oraz Leonarda Adlemana. Pierwszy algorytm, który może być stosowany zarówno do szyfrowania jak i do podpisów cyfrowych. Bezpieczeństwo szyfrowania opiera się na trudności faktoryzacji dużych liczb złożonych. Jego nazwa pochodzi od pierwszych liter nazwisk jego twórców}
}
\newglossaryentry{ASCII}
{
	name={ASCII},
	description={\textit{American Standard Code for Information Interchange}.\\&7-bitowy kod przyporządkowujący liczby z zakresu 0-127: literom (alfabetu angielskiego), cyfrom, znakom przestankowym i innym symbolom oraz poleceniom sterującym}
}
\newglossaryentry{kernel}
{
	name={kernel},
	description={Jądro \gls{OpenCL}.\\&Program uruchamiany w środowisku OpenCL na urządzeniu docelowym, a kompilowany na urządzeniu host}
}
\newglossaryentry{UTF8}
{
	name={UTF-8},
	description={\textit{UCS Transformation Format 8-bit}.\\&System kodowania \emph{Unicode}, wykorzystujący od 8 do 32 bitów do zakodowania pojedynczego znaku, w pełni kompatybilny z \gls{ASCII}}
}
\newacronym{SHA-0}{SHA-0}{\textit{Secure Hash Algorithm}.\\&Rodzina powiązanych ze sobą kryptograficznych funkcji skrótu zaprojektowanych przez NSA (\emph{National Security Agency})}
\newacronym{DES}{DES}{\textit{Data Encryption Standard}.\\&Symetryczny szyfr blokowy zaprojektowany w 1975 roku przez \emph{IBM} na zlecenie ówczesnego Narodowego Biura Standardów USA}
\newacronym{CPU}{CPU}{\textit{Central Processing Unit}.\\&Procesor}
\newacronym{API}{API}{\textit{Application Programming Interface}.\\&Interfejs Programowania Aplikacji}
\newacronym{BOM}{BOM}{\textit{Byte Order Mark}.\\&Znacznik kolejności bajtów}
\newacronym{IDE}{IDE}{\textit{Integrated Development Environment}.\\&Zintegrowane Środowisko Programistyczne}
